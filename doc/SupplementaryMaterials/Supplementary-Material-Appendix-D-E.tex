% Preamble
\documentclass{article}

%Dependencies
\usepackage[left]{lineno}
%\usepackage{indentfirst}
\usepackage{titlesec}
\usepackage[utf8]{inputenc}
\usepackage{amsmath}
\usepackage{amsfonts}
\usepackage{amssymb}
\usepackage{color,soul}
%\usepackage{times}
\usepackage[sc]{mathpazo} %Like Palatino with extensive math support
\linespread{1.25}
% Default margins are too wide all the way around. I reset them here
\usepackage{fullpage}
% Force figures to be in correct section
\usepackage{placeins}


% Fonts and language
\RequirePackage[utf8]{inputenc}
\RequirePackage[english]{babel}
\RequirePackage{amsmath,amsfonts,amssymb}
\RequirePackage{mathpazo}
\RequirePackage[scaled]{helvet}
\RequirePackage[T1]{fontenc}
\RequirePackage{url}
\RequirePackage[colorlinks=true, allcolors=black]{hyperref}
%\RequirePackage[colorlinks=true, allcolors=blue,draft=true]{hyperref}
\RequirePackage{lettrine}
\RequirePackage{cleveref}

% Graphics, tables and other formatting
\RequirePackage{graphicx,xcolor}
\RequirePackage{colortbl}
\RequirePackage{booktabs}
\RequirePackage{tikz}
\RequirePackage{algorithm}
\RequirePackage[noend]{algpseudocode}
\RequirePackage{changepage}
\RequirePackage[labelfont={bf,sf},%
                labelsep=space,%
                figurename=Figure,%
                singlelinecheck=off,%
                justification=RaggedRight]{caption}
%\setlength{\columnsep}{24pt} % Distance between the two columns of text
\setlength{\parindent}{12pt} % Paragraph indent

% Bibliography
\usepackage{natbib} \bibpunct{(}{)}{;}{author-year}{}{,}
\bibliographystyle{amnatnat}
\addto{\captionsenglish}{\renewcommand{\refname}{Literature Cited}}
\setlength{\bibsep}{0.0pt}

% Running headers
\usepackage{fancyhdr}
\setlength{\headheight}{28pt}

% Graphics package
\usepackage{graphicx}
\graphicspath{{../../output/figures/}.pdf}

% New commands: fonts
%\newcommand{\code}{\fontfamily{pcr}\selectfont}
%\newcommand*\chem[1]{\ensuremath{\mathrm{#1}}}
\newcommand\numberthis{\addtocounter{equation}{1}\tag{\theequation}}

%% Other Options

% Change subsection numbering
\renewcommand\thesubsection{\arabic{subsection})}
\renewcommand\thesubsubsection{}

% Subsubsection Title Formatting
\titleformat{\subsubsection}    
{\normalfont\fontsize{12pt}{17}\itshape}{\thesubsubsection}{12pt}{}


%%%%%%%%%%%%%%%%%%%%%%%%%%%%%%%%%%%%%%%%%%%%
\title{Supplementary Material (Appendices D--E) for: Sexually antagonistic variation and the evolution of dimorphic sexual systems. \textit{The American Naturalist}}

\author{Colin Olito$^{\ast,1,2}$ \& Tim Connallon$^{1}$}
\date{\today}

\begin{document}
\maketitle


\noindent{} $^{1}$ Centre for Geometric Biology, School of Biological Sciences, Monash University, Victoria 3800, Australia.

\noindent{} $^{2}$ \textit{Current address}: Department of Biology, Section for Evolutionary Ecology, Lund University, Lund 223 62, Sweden.

\noindent{} $^{\ast}$ Corresponding author e-mail: \url{colin.olito@gmail.com}

\newpage
%%%%%%%%%%%%%%%%%%%%%%%%%%%%%%%%%%%%%%%%%%%%
% Running Header
\pagestyle{fancyplain}
\makeatother
\lhead{\textit{Supplement to Olito \& Connallon. SA variation and dimorphic sexual systems. \textit{Am.~Nat.}.\\}}
\lhead{\textit{Supplement to Olito \& Connallon (2019). \textit{Am.~Nat.}\\ }}
\rhead{\textit{Evolution of dimorphic sexual systems}\\ }
\renewcommand{\headrulewidth}{0pt}
\renewcommand{\footrulewidth}{0pt}
\addtolength{\headheight}{0pt}


%%%%%%%%%%%%%%%%%%%%%%%%%%%%%%%%%%%%%%%%%%%%%%%%%%%%
%%%%%%%%%%%%%%%%%%%%%%%%%%%%%%%%%%%%%%%%%%%%%%%%
\section*{Appendix D: Supplementary figures}
\renewcommand{\theequation}{D\arabic{equation}}
\setcounter{equation}{0}
\renewcommand{\thefigure}{D\arabic{figure}}
\setcounter{figure}{0}



\begin{figure}[htbp]
\centering
\includegraphics[scale=0.75]{./DblMutExpansionKHatFig}
\caption{Effect of linkage between the $\mathbf{A}$ and $\mathbf{M}$ loci on the threshold level of reproductive compensation for successful invasion of unisexual females (panel A) and males (panel B) into hermaphrodite populations. Plots show the decrease in reproductive compensation for the invasion of 'double mutants' (i.e., based on $\lambda_{\mathbf{AM}}$) compared with the single locus invasion criteria (i.e., based on $\lambda_{\mathbf{M}}$; see Eq(3) and Eq(6) for the models of gyno- and androdioecy respectively) for different selection intensities, selfing rates ($C$) and inbreeding depression ($\delta$) as a function of the recombination rate between the two loci. We vary the recombination rate between $0$ and the threshold where the one- and two-locus invasion conditions become equivalent, Eq(4). For the model of gynodioecy, we first solved $\lambda_{\mathbf{AM}} > 1$ for $k$ (denoted $\hat{k}_{\mathbf{AM}}$ for convenience), and then calculated the proportional decrease in reproductive compensation by subtracting $\hat{k}_{\mathbf{AM}}$ from $\hat{k}$ (the right side of Eq(3)) and dividing by the latter for the specificed values of $C$, $\delta$, $r$, and $s_j$ (where $j \in \{f,m\}$). These results confirm the result presented in fig.~1 of the main text, that the parameter space where unisexual females can invade is increased in the two-locus model because linkage reduces $\hat{k}_{\mathbf{AM}}$ relative to $\hat{k}$ across biologically plausible values of $C$ and $\delta$. Because the integral of Eq(6) is undefined when $C = 1$, we restricted the integration to include high, but not complete, selfing rates $C = [0, 0.9]$ for the model of androdioecy; comparisons between the models should therefore be made with caution.}
\label{fig:dblMutKHatFig}
\end{figure}
\newpage{}


\begin{figure}[ht!]
\centering
\includegraphics[scale=0.58]{./FigD1-Gyno-obOut-funnel}
\caption{Invasion of dominant male-sterility mutations into populations with segregating SA variation under obligate outcrossing. Plots show the fraction of parameter conditions maintaining single-locus SA polymorphism (within the range $0 < s_f,s_m \leq 0.5$) where a dominant male-sterility allele at $\mathbf{M}$ can invade populations initially at single-locus equilibrium frequencies for $\mathbf{A}$ with additive fitness effects ($h_f=h_m=1/2$). Results were obtained by evaluating the two relevant candidate leading eigenvalues ($\lambda_{\mathbf{M}}$,$\lambda_{\mathbf{AM}}$) of the Jacobian matrix of the genotype $\times$ transmission mode recursions for populations at the above initial conditions for $1000$ points uniformly distributed throughout the relevant $s_f \times s_m$ paramter space. Blue points indicate parameter sets where $1 - \lambda_{\mathbf{M}} > 0$, and/or $\lambda_{\mathbf{AM}} - 1 > 0$. Solid black lines represent the corresponding single-locus invasion criteria for SA alleles.}
\label{fig:GynObOutFunnel}
\end{figure}
\newpage{}

\begin{figure}[ht!]
\centering
\includegraphics[scale=0.58]{./FigD2-Gyno-C25-d80-funnel}
\caption{Invasion of dominant male-sterility mutations into populations with segregating SA variation under conditions of low selfing ($C = 0.25$) and high inbreeding depression ($\delta = 0.8$). Plots show the fraction of parameter conditions maintaining single-locus SA polymorphism (within the range $0 < s_f,s_m \leq 0.5$) where a dominant male-sterility allele at $\mathbf{M}$ can invade populations initially at single-locus equilibrium frequencies for $\mathbf{A}$ with additive fitness effects ($h_f=h_m=1/2$). Results were obtained by evaluating the two relevant candidate leading eigenvalues ($\lambda_{\mathbf{M}}$,$\lambda_{\mathbf{AM}}$) of the Jacobian matrix of the genotype $\times$ transmission mode recursions for populations at the above initial conditions for $1000$ points uniformly distributed throughout the relevant $s_f \times s_m$ paramter space. Blue points indicate parameter sets where $1 - \lambda_{\mathbf{M}} > 0$, and/or $\lambda_{\mathbf{AM}} - 1 > 0$. Solid black lines represent the corresponding single-locus invasion criteria for SA alleles.}
\label{fig:GynC25d80Funnel}
\end{figure}
\newpage{}

\begin{figure}[ht!]
\centering
\includegraphics[scale=0.58]{./FigD3-Gyno-C75-d20-funnel}
\caption{Invasion of dominant male-sterility mutations into populations with segregating SA variation under conditions of high selfing ($C = 0.75$) and low inbreeding depression ($\delta = 0.2$). Plots show the fraction of parameter conditions maintaining single-locus SA polymorphism (within the range $0 < s_f,s_m \leq 0.5$) where a dominant male-sterility allele at $\mathbf{M}$ can invade populations initially at single-locus equilibrium frequencies for $\mathbf{A}$ with additive fitness effects ($h_f=h_m=1/2$). Results were obtained by evaluating the two relevant candidate leading eigenvalues ($\lambda_{\mathbf{M}}$,$\lambda_{\mathbf{AM}}$) of the Jacobian matrix of the genotype $\times$ transmission mode recursions for populations at the above initial conditions for $1000$ points uniformly distributed throughout the relevant $s_f \times s_m$ paramter space. Blue points indicate parameter sets where $1 - \lambda_{\mathbf{M}} > 0$, and/or $\lambda_{\mathbf{AM}} - 1 > 0$. Solid black lines represent the corresponding single-locus invasion criteria for SA alleles.}
\label{fig:GynC75d20Funnel}
\end{figure}
\newpage{}

\begin{figure}[ht!]
\centering
\includegraphics[scale=0.58]{./FigD4-Andro-obOut-funnel}
\caption{Invasion of dominant female-sterility mutations into populations with segregating SA variation under obligate outcrossing. Plots show the fraction of parameter conditions maintaining single-locus SA polymorphism (within the range $0 < s_f,s_m \leq 0.5$) where a dominant female-sterility allele at $\mathbf{M}$ can invade populations initially at single-locus equilibrium frequencies for $\mathbf{A}$ with additive fitness effects ($h_f=h_m=1/2$). Results were obtained by evaluating the two relevant candidate leading eigenvalues ($\lambda_{\mathbf{M}}$,$\lambda_{\mathbf{AM}}$) of the Jacobian matrix of the genotype $\times$ transmission mode recursions for populations at the above initial conditions for $1000$ points uniformly distributed throughout the relevant $s_f \times s_m$ paramter space. Blue points indicate parameter sets where $1 - \lambda_{\mathbf{M}} > 0$, and/or $\lambda_{\mathbf{AM}} - 1 > 0$. Solid black lines represent the corresponding single-locus invasion criteria for SA alleles.}
\label{fig:AndroObOutFunnel}
\end{figure}
\newpage{}

\begin{figure}[ht!]
\centering
\includegraphics[scale=0.58]{./FigD5-Andro-C25-d80-funnel}
\caption{Invasion of dominant female-sterility mutations into populations with segregating SA variation under conditions of low selfing ($C = 0.25$) and high inbreeding depression ($\delta = 0.8$). Plots show the fraction of parameter conditions maintaining single-locus SA polymorphism (within the range $0 < s_f,s_m \leq 0.5$) where a dominant female-sterility allele at $\mathbf{M}$ can invade populations initially at single-locus equilibrium frequencies for $\mathbf{A}$ with additive fitness effects ($h_f=h_m=1/2$). Results were obtained by evaluating the two relevant candidate leading eigenvalues ($\lambda_{\mathbf{M}}$,$\lambda_{\mathbf{AM}}$) of the Jacobian matrix of the genotype $\times$ transmission mode recursions for populations at the above initial conditions for $1000$ points uniformly distributed throughout the relevant $s_f \times s_m$ paramter space. Blue points indicate parameter sets where $1 - \lambda_{\mathbf{M}} > 0$, and/or $\lambda_{\mathbf{AM}} - 1 > 0$. Solid black lines represent the corresponding single-locus invasion criteria for SA alleles.}
\label{fig:AndC25d80Funnel}
\end{figure}
\newpage{}

\begin{figure}[ht!]
\centering
\includegraphics[scale=0.58]{./FigD6-Andro-C75-d20-funnel}
\caption{Invasion of dominant female-sterility mutations into populations with segregating SA variation under conditions of high selfing ($C = 0.75$) and low inbreeding depression ($\delta = 0.2$). Plots show the fraction of parameter conditions maintaining single-locus SA polymorphism (within the range $0 < s_f,s_m \leq 0.5$) where a dominant female-sterility allele at $\mathbf{M}$ can invade populations initially at single-locus equilibrium frequencies for $\mathbf{A}$ with additive fitness effects ($h_f=h_m=1/2$). Results were obtained by evaluating the two relevant candidate leading eigenvalues ($\lambda_{\mathbf{M}}$,$\lambda_{\mathbf{AM}}$) of the Jacobian matrix of the genotype $\times$ transmission mode recursions for populations at the above initial conditions for $1000$ points uniformly distributed throughout the relevant $s_f \times s_m$ paramter space. Blue points indicate parameter sets where $1 - \lambda_{\mathbf{M}} > 0$, and/or $\lambda_{\mathbf{AM}} - 1 > 0$. Solid black lines represent the corresponding single-locus invasion criteria for SA alleles.}
\label{fig:AndC75d20Funnel}
\end{figure}
\newpage{}



\begin{figure}[htbp]
\centering
\includegraphics[scale=0.7]{./FigD7R1}
\caption{Invasion of unisexuals into populations with pre-existing SA polymorphism. Plots show the fraction of parameter conditions maintaining single-locus SA polymorphism (within the range $0 < s_f,s_m \leq 0.5$) where a dominant sex-specific sterility allele at $\mathbf{M}$ can invade, assuming completely recessive SA fitness effects at the SA $\mathbf{A}$ locus ($h_f=h_m=0$), plotted as a function of the recombination rate $r$. Panels A--C show results from the model of gynodioecy via invasion of a male-sterility allele, while planels D--F show results for the model of androdioecy via invasion of a female-sterility allele. For each panel, results are shown for different values of reproductive compensation, $k$, chosen as a fraction of the single-locus invasion threshold for $M_2$ ($\hat{k}$, which is equal to the right-hand side of Eq(3) or Eq(6) in the main text for the models of gynodioecy and androdioecy respectively). Hence, the orange, green, and dark blue lines show scenarios where unisexuals experience a decrease in gamete production relative to hermaphrodites of $1$, $5$, and $10\%$. Results were obtained by evaluating the two relevant candidate leading eigenvalues ($\lambda_{\mathbf{A}}$,$\lambda_{\mathbf{M}}$,$\lambda_{\mathbf{AM}}$) of the Jacobian matrix of the genotype $\times$ transmission mode recursions for populations at the above initial conditions for $1000$ points uniformly distributed throughout the relevant $s_f \times s_m$ parameter space.}
\label{fig:PrInvDomRev}
\end{figure}

\begin{figure}[ht!]
\centering
\includegraphics[scale=0.58]{./FigD8-Gyno-domRev-obOut-funnel}
\caption{Invasion of dominant male-sterility mutations into populations with segregating SA variation under obligate outcrossing ($C=0$) and a complete dominance reversal. Plots show the fraction of parameter conditions maintaining single-locus SA polymorphism (within the range $0 < s_f,s_m \leq 0.5$) where a dominant male-sterility allele at $\mathbf{M}$ can invade populations initially at single-locus equilibrium frequencies for $\mathbf{A}$ with completely recessive fitness effects ($h_f=h_m=0$). Results were obtained by evaluating the two relevant candidate leading eigenvalues ($\lambda_{\mathbf{M}}$,$\lambda_{\mathbf{AM}}$) of the Jacobian matrix of the genotype $\times$ transmission mode recursions for populations at the above initial conditions for $1000$ points uniformly distributed throughout the relevant $s_f \times s_m$ paramter space. Blue points indicate parameter sets where $1 - \lambda_{\mathbf{M}} > 0$, and/or $\lambda_{\mathbf{AM}} - 1 > 0$. Solid black lines represent the corresponding single-locus invasion criteria for SA alleles.}
\label{fig:GynObOutFunnelDomRev}
\end{figure}
\newpage{}

\begin{figure}[ht!]
\centering
\includegraphics[scale=0.58]{./FigD9-Gyno-domRev-C25-d80-funnel}
\caption{Invasion of dominant male-sterility mutations into populations with segregating SA variation under conditions of low selfing ($C = 0.25$), high inbreeding depression ($\delta = 0.8$), and a complete dominance reversal. Plots show the fraction of parameter conditions maintaining single-locus SA polymorphism (within the range $0 < s_f,s_m \leq 0.5$) where a dominant male-sterility allele at $\mathbf{M}$ can invade populations initially at single-locus equilibrium frequencies for $\mathbf{A}$ with completely recessive fitness effects ($h_f=h_m=0$). Results were obtained by evaluating the two relevant candidate leading eigenvalues ($\lambda_{\mathbf{M}}$,$\lambda_{\mathbf{AM}}$) of the Jacobian matrix of the genotype $\times$ transmission mode recursions for populations at the above initial conditions for $1000$ points uniformly distributed throughout the relevant $s_f \times s_m$ paramter space. Blue points indicate parameter sets where $1 - \lambda_{\mathbf{M}} > 0$, and/or $\lambda_{\mathbf{AM}} - 1 > 0$. Solid black lines represent the corresponding single-locus invasion criteria for SA alleles.}
\label{fig:GynC25d80FunnelDomRev}
\end{figure}
\newpage{}

\begin{figure}[ht!]
\centering
\includegraphics[scale=0.58]{./FigD10-Gyno-domRev-C75-d20-funnel}
\caption{Invasion of dominant male-sterility mutations into populations with segregating SA variation under conditions of high selfing ($C = 0.75$), low inbreeding depression ($\delta = 0.2$), and a complete dominance reversal. Plots show the fraction of parameter conditions maintaining single-locus SA polymorphism (within the range $0 < s_f,s_m \leq 0.5$) where a dominant male-sterility allele at $\mathbf{M}$ can invade populations initially at single-locus equilibrium frequencies for $\mathbf{A}$ with completely recessive fitness effects ($h_f=h_m=0$). Results were obtained by evaluating the two relevant candidate leading eigenvalues ($\lambda_{\mathbf{M}}$,$\lambda_{\mathbf{AM}}$) of the Jacobian matrix of the genotype $\times$ transmission mode recursions for populations at the above initial conditions for $1000$ points uniformly distributed throughout the relevant $s_f \times s_m$ paramter space. Blue points indicate parameter sets where $1 - \lambda_{\mathbf{M}} > 0$, and/or $\lambda_{\mathbf{AM}} - 1 > 0$. Solid black lines represent the corresponding single-locus invasion criteria for SA alleles.}
\label{fig:GynC75d20FunnelDomRev}
\end{figure}
\newpage{}

\begin{figure}[ht!]
\centering
\includegraphics[scale=0.58]{./FigD11-Andro-domRev-obOut-funnel}
\caption{Invasion of dominant female-sterility mutations into populations with segregating SA variation under obligate outcrossing ($C=0$) and a complete dominance reversal. Plots show the fraction of parameter conditions maintaining single-locus SA polymorphism (within the range $0 < s_f,s_m \leq 0.5$) where a dominant female-sterility allele at $\mathbf{M}$ can invade populations initially at single-locus equilibrium frequencies for $\mathbf{A}$ with completely recessive fitness effects ($h_f=h_m=0$). Results were obtained by evaluating the two relevant candidate leading eigenvalues ($\lambda_{\mathbf{M}}$,$\lambda_{\mathbf{AM}}$) of the Jacobian matrix of the genotype $\times$ transmission mode recursions for populations at the above initial conditions for $1000$ points uniformly distributed throughout the relevant $s_f \times s_m$ paramter space. Blue points indicate parameter sets where $1 - \lambda_{\mathbf{M}} > 0$, and/or $\lambda_{\mathbf{AM}} - 1 > 0$. Solid black lines represent the corresponding single-locus invasion criteria for SA alleles.}
\label{fig:AndObOutFunnelDomRev}
\end{figure}
\newpage{}

\begin{figure}[ht!]
\centering
\includegraphics[scale=0.58]{./FigD12-Andro-domRev-C25-d80-funnel}
\caption{Invasion of dominant female-sterility mutations into populations with segregating SA variation under conditions of low selfing ($C = 0.25$), high inbreeding depression ($\delta = 0.8$), and a complete dominance reversal. Plots show the fraction of parameter conditions maintaining single-locus SA polymorphism (within the range $0 < s_f,s_m \leq 0.5$) where a dominant female-sterility allele at $\mathbf{M}$ can invade populations initially at single-locus equilibrium frequencies for $\mathbf{A}$ with completely recessive fitness effects ($h_f=h_m=0$). Results were obtained by evaluating the two relevant candidate leading eigenvalues ($\lambda_{\mathbf{M}}$,$\lambda_{\mathbf{AM}}$) of the Jacobian matrix of the genotype $\times$ transmission mode recursions for populations at the above initial conditions for $1000$ points uniformly distributed throughout the relevant $s_f \times s_m$ paramter space. Blue points indicate parameter sets where $1 - \lambda_{\mathbf{M}} > 0$, and/or $\lambda_{\mathbf{AM}} - 1 > 0$. Solid black lines represent the corresponding single-locus invasion criteria for SA alleles.}
\label{fig:AndC25d80FunnelDomRev}
\end{figure}
\newpage{}

\begin{figure}[ht!]
\centering
\includegraphics[scale=0.58]{./FigD13-Andro-domRev-C75-d20-funnel}
\caption{Invasion of dominant female-sterility mutations into populations with segregating SA variation under conditions of high selfing ($C = 0.75$), low inbreeding depression ($\delta = 0.2$), and a complete dominance reversal. Plots show the fraction of parameter conditions maintaining single-locus SA polymorphism (within the range $0 < s_f,s_m \leq 0.5$) where a dominant female-sterility allele at $\mathbf{M}$ can invade populations initially at single-locus equilibrium frequencies for $\mathbf{A}$ with completely recessive fitness effects ($h_f=h_m=0$). Results were obtained by evaluating the two relevant candidate leading eigenvalues ($\lambda_{\mathbf{M}}$,$\lambda_{\mathbf{AM}}$) of the Jacobian matrix of the genotype $\times$ transmission mode recursions for populations at the above initial conditions for $1000$ points uniformly distributed throughout the relevant $s_f \times s_m$ paramter space. Blue points indicate parameter sets where $1 - \lambda_{\mathbf{M}} > 0$, and/or $\lambda_{\mathbf{AM}} - 1 > 0$. Solid black lines represent the corresponding single-locus invasion criteria for SA alleles.}
\label{fig:AndC75d20FunnelDomRev}
\end{figure}
\newpage{}


\begin{figure}[ht!]
\centering
\includegraphics[scale=0.75]{./EqFreqAtKHat}
\caption{Effect of linkage on equilibrium frequencies of unisexual (A) females, and (B) males when reproductive compensation is nearly equal to the single-locus threshold for invasion of unisexual sterility allele (i.e., $k = \hat{k} \times 1.001$; where $\hat{k}$ is equal to the right-hand side of Eq(3) or Eq(6) as appropriate). Results are shown for the models of gyno- and androdioecy via invasion of recessive unisexual sterility alleles, additive fitness effects at $\mathbf{A}$ ($h_f = h_m = 0.5$, using selection coefficients of $s_m = 0.1$ (for the model of gynodioecy) and $s_f = 0.1$ (for the model of androdioecy), and inbreeding depression that follows $\delta = \delta^\ast(1 - C/2)$ (see Models section in the main text). Plots illustrate the increase in equilibrium frequencies of unisexuals predicted by our two-locus models (dashed greyscale lines) relative to the corresponding exact single-locus equilibrium frequencies, which are very small at this level of reproductive compensation (solid black line; $\hat{Z}$ predicted by \citealt{Charlesworth1978a}). Results are shown for four different levels of recombination, highlighting that with weaker linkage, the two-locus predictions converge on those of the single-locus model.}
\label{fig:eqFreq2v1Loc}
\end{figure}
\newpage{}

\begin{figure}[ht!]
\centering
\includegraphics[scale=0.75]{./FigD14-Eq-Unisexual-Frequency-DomRev}
\caption{Equilibrium frequencies of unisexual (A--B) females and (C--D) males across a gradient of reproductive compensation. Results are shown for the models of gyno- and androdioecy via invasion of recessive unisexual sterility alleles, recessive SA fitness effects at $\mathbf{A}$ ($h_f = h_m = 1/4$, using selection coefficients of $s_m = 0.1$ (for the model of gynodioecy) and $s_f = 0.1$ (for the model of androdioecy), and inbreeding depression that decreases linearly with the selfing rate: $\delta = \delta^\ast(1 - C/2)$ (see Models section in the main text). Plots show the equilibrium frequencies of unisexuals predicted by our models for five different levels of reproductive compensation, calculated as a fraction of the single-locus invasion criterion for $M_2$ defined by Eq(3) and Eq(6) in the main text. Note that the single-locus equilibrium frequency of unisexuals always equals $0$ when $k < \hat{k}$. Hence, the lines corresponding to $k < \hat{k}$ illustrate how linkage among SA loci expands the parameter space where unisexual sterility alleles can invade beyond the predictions of the single-locus models.}
\label{fig:eqFreqDomRev}
\end{figure}
\newpage{}

\FloatBarrier

%%%%%%%%%%%%%%%%%%%%%%%%%%%%%%%%%%%%%%%%%%%%%%%%
\section*{Appendix E: Alternative relations between the selfing rate and inbreeding depression}
\renewcommand{\theequation}{E\arabic{equation}}
\setcounter{equation}{0}
\renewcommand{\thefigure}{E\arabic{figure}}
\setcounter{figure}{0}

As explained in the main text, we attempted to account for negative covariance between $C$ and $\delta$ in our deterministic simulations, as might be expected if inbreeding depression is caused by recessive deleterious mutations \citep{Charlesworth1987, Charlesworth2009}. We did this by constraining inbreeding depression to follow a simple linear function of the selfing rate, $\delta = \delta^\ast(1 - C/2)$, where $\delta^\ast$ represents the hypothetical severity of inbreeding depression if selfing were enforced on a completely outcrossing population ($\delta^\ast \in [0,1]$). We then used a value of $\delta^{\ast}$ for our simulations that yielded predicted values for $\delta$ that were roughly consistent with empirical data for partially selfing plant populations \citep{HusbandSchemske1996}. Here we briefly address the consequences of relaxing this assumption. However, we reiterate that our goal is not to accurately model inbreeding depression, but to capture the basic effect of purging deleterious recessives in partially selfing populations. 

In the simplest case, the mutation load due to deleterious recessive mutations at a single locus in a completely selfing population should be roughly half that of a randomly mating outcrossing population ($\mu$ versus $2 \mu$, where $\mu$ is the genome-wide mutation rate to recessive deleterious alleles; \citealt{OhtaCockerham1974}). Moreover, a variety of detailed genetic models of inbreeding depression predict that inbreeding depression should follow a negative decelerating curve as a function of the population selfing rate (see, for example, \citealt{OhtaCockerham1974, LandeSchemske1985, Charlesworth1985, Garcia-Dorado2017, LandePorcher2017}). To capture these main features, first we define a simple function describing the mutation load due to recessive deleterious mutations,

\begin{equation}\label{eq:Load}
    L = \frac{a (1 - C)}{C + a (1 - C)},
\end{equation}

\noindent where $C$ is the population selfing rate, and $a$ is a shape parameter determining the curvature of the line. When $a = 1$, Eq(\ref{eq:Load}) yields a straight line. For $a < 1$ the function is concave up (Fig.~\ref{fig:Cdelta}A), and for $a > 1$ it is concave downward. We then incorporate Eq(\ref{eq:Load}) into a simple expression for inbreeding depression,

\begin{equation}\label{eq:ID}
    \delta = \delta^{\ast} - \delta^{\ast} b (1 - L),
\end{equation}

\noindent where $\delta^\ast$ is defined as above, and $b$ is a shape parameter determining how far $\delta$ will decline under complete selfing ($C = 1$) relative to $\delta^{\ast}$. This flexible function allows us to define an initial level of inbreeding depression under obligate outcrossing ($\delta^\ast$), how far inbreeding depression will decline under complete selfing ($b$), and the curvature of the line connecting these two endpoints ($a$). Given that in a completely selfing population $L$ is expected to be roughly half that in an obligately outcrossing one, we constrain $b$ to equal $1/2$. Note that when $a=1$ and $b=1/2$, Eq(\ref{eq:ID}) yields $\delta = \delta^\ast(1 - C/2)$, as defined in the main text. To illustrate the effect of altering the shape of Eq(\ref{eq:Load}) to be concave upward rather than linear, we examine the equilibrium female frequencies predicted by the model of gynodioecy via invasion of a recessive male-sterility allele when reproductive compensation exceeds the single locus threshold for invasion ($k > \hat{k}$). A non-linear relation for $\delta$ has only a minor effect on the equilibrium female frequencies, and does not alter the main conclusions from the simulations (fig.~\ref{fig:Cdelta}B).

\begin{figure}[ht!]
\centering
\includegraphics[width=\linewidth]{FigC1-2-compareCdelta}
\caption{Comparison of linear versus non-linear relations between the hermaphrodite selfing rate, $C$, and inbreeding depression, $\delta$. Panel (A) shows the linear ($a=1$; solid line) vs.~non-linear ($a=0.2$; dashed line) functions for $\delta$. In both cases, we assume that inbreeding depression in a completely selfing population will be half that of an obligately outcrossing one ($b=0.5$ for both lines). Panel (B) shows similar simulation results to those presented in Fig.~2 of the main text -- equilibrium frequencies of unisexual females compared with single-locus predictions when reproductive compensation is above the single-locus threshold (i.e., $k > \hat{k}$). Results are shown for the model of gynodioecy via invasion of a recessive male-sterility allele, with additive fitness effects at $\mathbf{A}$ ($h_f = h_m = 0.5$), using a selection coefficient of $s_m = 0.4$., and the same linear (solid lines) vs.~nonlinear (dashed lines) functions of inbreeding depression as shown in panel (A).}
\label{fig:Cdelta}
\end{figure}



\clearpage

%%%%%%%%%%%%%%%%%%%%%
% Bibliography
%%%%%%%%%%%%%%%%%%%%%
\bibliography{dioecySA-Supplements-bibliography}



\end{document}
