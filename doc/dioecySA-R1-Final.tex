%%%%%%%%%%%%%%%%%%%%%%%%%%%%%
%Preamble
\documentclass{article}

%Dependencies
\usepackage[left]{lineno}
\usepackage{titlesec}
\usepackage{xcolor}

\newcommand\hl[1]{%
  \bgroup
  \hskip0pt\color{blue!80!black}%
  #1%
  \egroup
}
\usepackage{ogonek}
\usepackage{float}


% Other Packages
%\usepackage{times}
\RequirePackage{fullpage}
\linespread{1.5}
\RequirePackage[colorlinks=true, allcolors=black]{hyperref}
\RequirePackage[english]{babel}
\RequirePackage{amsmath,amsfonts,amssymb}
\RequirePackage[sc]{mathpazo}
\RequirePackage[T1]{fontenc}
\RequirePackage{url}

% Bibliography
%\usepackage[authoryear,sectionbib,sort]{natbib}
\usepackage{natbib} \bibpunct{(}{)}{;}{author-year}{}{,}
\bibliographystyle{amnatnat}
\addto{\captionsenglish}{\renewcommand{\refname}{Literature Cited}}
\setlength{\bibsep}{0.0pt}

% Graphics package
\usepackage{graphicx}
\graphicspath{{../output/figures/}.pdf}

% New commands: fonts
%\newcommand{\code}{\fontfamily{pcr}\selectfont}
%\newcommand*\chem[1]{\ensuremath{\mathrm{#1}}}
\newcommand\numberthis{\addtocounter{equation}{1}\tag{\theequation}}
\titleformat{\subsubsection}[runin]{\bfseries\itshape}{\thesubsubsection.}{0.5em}{}


%%%%%%%%%%%%%%%%%%%%%%%%%%%%%
% Title Page

\title{Sexually antagonistic variation and the evolution of dimorphic sexual systems}
\author{Colin Olito$^{\ast,1,2}$ \& Tim Connallon$^{1}$}
\date{\today}

\begin{document}
\maketitle


\noindent{} $^{1}$ Centre for Geometric Biology, School of Biological Sciences, Monash University, Victoria 3800, Australia.

\noindent{} $^{2}$ \textit{Present address}: Department of Biology, Section for Evolutionary Ecology, Lund University, Lund 223 62, Sweden.

\noindent{} $^{\ast}$ Corresponding author e-mail: \url{colin.olito@gmail.com}

\bigskip

\noindent{} \textit{Manuscript elements}: Figure~1, Figure~2, Figure~3, Table~1; Online Supplementary Material: Appendix A -- Development of the recursions; Appendix B -- Expressions for single-locus SA polymorphism; Appendix C -- Results for dominance reversals; Appendix D -- Supplementary figures; Appendix E -- Alternative relations between the selfing rate and inbreeding depression; Appendix F -- Mathematica code to reproduce analytic results.

\bigskip
\noindent{} \textit{Running Head}: SA variation and dimorphic sexual systems

\bigskip

\noindent{} \textit{Keywords}: Androdioecy; Dioecy; Gynodioecy; Intralocus sexual conflict; Linkage disequilibrium; Recombination; Sexual system; Sexual dimorphism; Sex chromosome evolution

\bigskip

\noindent{} \textit{Manuscript type}: Major Article

\bigskip


% Set line number options
\linenumbers
\modulolinenumbers[1]
\renewcommand\linenumberfont{\normalfont\small}

%%%%%%%%%%%%%%%%%%%%%%%%%%%%%
% Main Text

\newpage{}
\section*{Abstract}

\noindent{} Multicellular Eukaryotes use a broad spectrum of sexual reproduction strategies, ranging from simultaneous hermaphroditism to complete dioecy (separate sexes). The evolutionary pathway from hermaphroditism to dioecy involves the spread of "sterility alleles" that eliminate female or male reproductive functions, producing unisexual individuals. Classical theory predicts that evolutionary transitions to dioecy are feasible when female and male sex functions genetically trade-off with one another (allocation to sex functions is "sexually antagonistic"), and rates of self-fertilization and inbreeding depression are high within the ancestral hermaphrodite population. We show that genetic linkage between sterility alleles and loci under sexually antagonistic selection significantly alters these classical predictions. We identify three specific consequences of linkage for the evolution of dimorphic sexual systems. First, linkage broadens conditions for the invasion of unisexual sterility alleles, facilitating transitions to sexual systems that are intermediate between hermaphroditism and dioecy (andro- and gynodioecy). Second, linkage elevates the equilibrium frequencies of unisexual individuals within andro- and gynodioecious populations, which promotes subsequent transitions to full dioecy. Third, linkage dampens the role of inbreeding during transitions to andro- and gynodioecy, making these transitions feasible in outbred populations. We discuss implications of these results for the evolution of dimorphic reproductive systems and sex chromosomes.
\newpage{}


%%%%%%%%%%%%%%%%%%%%%%%%
\section*{Introduction} \label{sec:Introduction}
%%%%%%%%%%%%%%%%%%%%%%%%
Multicellular Eukaryotes use a diverse array of strategies for sexual reproduction \citep{Bachtrog2014}. At one end of the spectrum, simultaneous hermaphrodites express both female and male reproductive structures (sex functions) and reproduce by outcrossing, self-fertilization, or a combination of both (mixed mating). At the other end of the spectrum, dioecious (or gonochoristic) species include separate females and males -- usually at similar frequencies -- which typically show pronounced sex differences in morphology, behavior, physiology and life-history \citep{Andersson1994}. Flowering plants are particularly diverse in the range of sexual systems that they exhibit, which includes hermaphroditism, dioecy, and nearly every possible state in between (\citealt{Darwin1877, Westergaard1958, Bawa1980, SakaiWeller1999, Barrett2010, Renner2014, Bachtrog2014}; see \citealt{JarneAuld2006} for animals). 

Dioecy has evolved repeatedly in plants from an ancestral state of functional hermaphroditism -- an observation that has inspired much research into the evolutionary mechanisms that might drive such transitions \citep{Westergaard1958, SakaiWeller1999, Charlesworth2006, Bachtrog2014, Renner2014, GoldbergOtto2017, KaferPannell2017}. Theory and data agree that transitions from hermaphroditism to dioecy are likely to follow a two-step evolutionary pathway. In the first step, a nuclear unisexual sterility mutation spreads within a hermaphrodite population, resulting in the evolution of gynodioecy (a population composed of hermaphrodites and unisexual females) or androdioecy (hermaphrodites and males; \citealt{Charlesworth1978a, Charlesworth1978b}). During the second step, the alternative sterility allele type -- a female-sterility allele in a gynodioecious population, or a male-sterility allele in an androdioecious population -- spreads to complete the transition to full dioecy \citep{Westergaard1958, Charlesworth1978a, Charlesworth1978b, Charlesworth2006, Charlesworth2009, KaferPannell2017}. In these scenarios, unisexuality can arise through the invasion of mutations that individually cause the complete loss of an ancestral sex function in hermaphrodite individuals, or through the spread of multiple alleles that cumulatively cause unisexuality \citep{Charlesworth1978a, Charlesworth1978b, Morgan1992a, Morgan1992b, SegerEckhart1996, Charlesworth1999}. Current data on the evolution of plant sex determination systems are largely consistent with this two-step scenario \citep{Westergaard1958, Charlesworth2002, Charlesworth2006, Renner2014, Ashman2015}.

Classical theory has shown that transitions to dioecy are most likely to occur when two conditions are met \citep{Lewis1941, Lloyd1975, Lloyd1976, Charlesworth1978a, Morgan1992a, SegerEckhart1996}. First, female and male sex functions in hermaphrodites must genetically trade-off against one another (e.g., through allocation trade-offs). Second, rates of self-fertilization and inbreeding depression in the ancestral hermaphrodite population must be high enough to promote the spread of unisexual sterility alleles that initiate the transition to dioecy. This theory predicts that gynodioecy, rather than androdioecy, represents the more likely intermediate state between hermaphroditism and dioecy (\citealt{Lloyd1975, Charlesworth1978a, SegerEckhart1996, KaferPannell2017}; see \citealt{Charlesworth1999, Charlesworth2006} for comprehensive reviews of relevant theory). Finally, the second sterility mutation to invade the population is predicted to establish in tight linkage with the locus segregating for the first sterility allele (the "linkage constraint" of \citealt{Charlesworth1978a}). The establishment of complementary sterility mutations, under tight linkage and in repulsion phase, sets the stage for the subsequent evolution of heteromorphic sex chromosomes (e.g., a gene-rich X and degenerate Y; \citealt[pp.265--269]{Bull1983}; \citealt{Rice1987, Qiuetal2013}; reviewed in \citealt{Bachtrog2006,Charlesworth2002}). 

One strength of the theory outlined above is that it makes clear and testable predictions about the types of species that are most likely to transition to dioecy. According to the theory, hermaphrodite lineages with high rates of self-fertilization are particularly likely to transition to dioecy through an intermediate gynodioecious state \citep{Charlesworth1978a, Charlesworth1999, KaferPannell2017}. Yet, while some studies of single species support this longstanding prediction (e.g., reviewed in \citealt{Webb1999, DufayBillard2012}), empirical evidence for an association between dioecy and selfing across taxa remains equivocal \citep{Charlesworth1985, Charlesworth2006, Renner2014}. The mismatch between the theoretical predictions and empirical patterns suggests that additional biological factors are likely to influence evolutionary transitions to dioecy, potentially overriding the facilitating effects of self-fertilization and inbreeding depression for the evolution of dioecy.

Here, we extend population genetics theory for the evolution of dioecy by showing that genetic linkage of unisexual sterility alleles to a sexually antagonistic (or "SA") locus facilitates the initial step in the pathway to full dioecy (i.e., transitions from hermaphroditism to andro- or gynodioecy), and inverts the relation between self-fertilization and dioecy relative to predictions of classical theory. SA loci refer to sites in the genome where fitness effects of genetic variation trade off between females and males or between hermaphrodite sex functions (see \citealt{Gregorious1982, Morgan1992a, Morgan1992b, JordanConnallon2014, Olito2017}). Our models differ from the classical theories for the evolution of dioecy \citep{Charlesworth1978a} by allowing unisexual sterility mutations to essentially hitchhike with genetically linked SA alleles, and vice versa. Linkage to an SA locus expands the range of conditions where unisexuals will spread, elevates the equilibrium frequencies of unisexuals in the population, and promotes evolutionary transitions to dioecy most readily in predominantly outcrossing populations. These findings have parallels in recent theory on the evolutionary consequences of multi-locus SA genetic variation \citep{ConnallonClark2010, Patten2010, UbedaPatten2010, ConnallonJordan2016, Olito2017}, and they suggest a new role for SA genetic variation in the evolutionary origins of new sex-chromosome systems.

%Genetically-based trade-offs between sex functions are special cases of the much broader phenomenon of sexually antagonistic selection (hereafter, "SA" selection), where alleles that benefit one sex are deleterious for the other \citep{Gregorious1982, Rice1992, BondurianskyChenoweth2009, Mank2017}. In a hermaphrodite population, an SA allele could, for example, increase ovule production yet decrease pollen production or viability \citep{Gregorious1982, Morgan1992a, JordanConnallon2014, Olito2017}. \hl{In fact, a variety of earlier theoretical models have explored such 'SA' allocation trade-offs, and the conditions that favour extreme allocation strategies resulting in effectively separate sexes (e.g.,} \citealt{Gregorious1982, Morgan1992a, Morgan1992b, SegerEckhart1996}). \hl{However, these models did not examine the effects of such allocation trade-offs on the evolution of sterility alleles, which ultimately form the genetic basis of separate sexes} \citep{Charlesworth1978a,Charlesworth2006}. During the transition to dioecy, male-sterility alleles that improve female sex function, and female-sterility alleles that improve male sex function, are essentially SA alleles, with SA effects potentially arising via reproductive compensation \citep{Lewis1941, Lloyd1975, Charlesworth1978a, Gregorious1982}. The evolutionary dynamics of unisexual sterility alleles can therefore be understood through the broader perspective of SA selection theory. For example, SA alleles spread within a population when the benefit to one sex offsets the cost to the other, with the critical balance between benefits and costs determined by the level of inbreeding in the population \citep{Kidwell1977, Gregorious1982, JordanConnallon2014}. Similarly, unisexual sterility alleles -- which initiate the transition to dioecy -- can spread when the loss of fitness from unisexual sterility is offset by fitness gains through the remaining sex function, and once again, the balance between these factors is \hl{influenced} by inbreeding \citep{Charlesworth1978a}. 

%\hl{Despite the longstanding theoretical prediction of a positive correlation between the extent of self-fertilization and the rate of evolutionary transitions to dioecy, empirical evidence for this association remains equivocal} \citep{Charlesworth1985, Charlesworth2006, Renner2014}\hl{, suggesting that classical single-locus theory misses a key factor influencing the likelihood of such transitions}. Recent multi-locus population genetics theory has shown that a locus under SA selection can substantially \hl{affect} the evolutionary dynamics at linked loci that segregate for deleterious mutations, meiotic drive elements, or SA alleles \citep{ConnallonClark2010, Patten2010, UbedaPatten2010, ConnallonJordan2016, Olito2017}. For example, SA selection at one locus facilitates the spread of \hl{genetically} linked SA alleles that would otherwise be lost from a population (i.e., \hl{under free recombination}). This consequence of linkage to an SA locus applies in both dioecious populations \citep{Patten2010,ConnallonClark2010} and hermaphrodite populations with fixed rates of self-fertilization \citep{Olito2017}. These multi-locus models suggest a mechanism that might facilitate transitions between hermaphrodite and andro- or gynodioecious sexual systems. Linkage of unisexual sterility alleles to an SA locus could, in principle, promote the spread of unisexuals within hermaphrodite populations, and relax evolutionary constraints that would otherwise hinder evolutionary transitions between sexual systems.

%\hl{Here, we extend classical theory for the evolution of dioecy by considering the role of linkage between SA loci during the initiating step: transitions from hermpahroditism to andro- and gynodioecy. Our models differ from the classic single-locus models of }\citet{Charlesworth1978a} \hl{by introducing a second SA locus that may be genetically linked to the site of the unisexual sterility mutation initiating the transition.} We focus on three key questions in our analysis\hl{: How does linkage between unisexual sterility alleles and SA loci affect (1) conditions for the evolutionary invasion of unisexual sterility alleles, (2) frequencies of unisexual individuals in andro- and gynodioecious populations, and (3) the evolutionary dynamics and conditions for polymorphism at SA loci?} Our results show that linkage to an SA locus expands the parameter conditions where unisexuals invade, dampens the constraints associated with outcrossing on the invasion of sterility alleles, and elevates the equilibrium frequencies of unisexuals in andro- and gynoodioecious populations beyond single-locus predictions. \hl{Finally, our results suggest} a hitherto unrecognized role for SA genetic variation \hl{prior to} the initial stages of sex-chromosome evolution.


%%%%%%%%%%%%%%%%%%%%%%%%
\section*{Models} \label{sec:Models}
%%%%%%%%%%%%%%%%%%%%%%%%

Building upon previous theory, we model evolutionary transitions from hermaphroditism to andro- and gynodioecy via the invasion of unisexual nuclear sterility alleles that cause complete male or complete female sterility, respectively (\citealt{Charlesworth1978a}). We separately consider the invasion of completely dominant and completely recessive sterility alleles during transitions to andro- and gynodioecy, leading to a total of four population genetic models. These scenarios mirror classical theory for the invasion of unisexuals in hermaphrodite populations \citep{Charlesworth1978a}, providing clear points of contrast. Below, we provide a full description of the simplest model: the evolution of gynodioecy via invasion of a dominant male-sterility mutation. We then briefly outline essential differences in the other three models, with reference to the first. Full details for all four models are provided in Appendix A of the Online Supporting Information, and simulation code is available at \url{https://github.com/colin-olito/dioecySA}.

%%%%%%%%%%%%%%%%%%%%%%%%
\subsection*{Gynodioecy}

Consider a genetic system involving two diallelic autosomal loci, $\mathbf{A}$ (with alleles $A$, $a$) and $\mathbf{M}$ (with alleles $M_1$, $M_2$), that recombine at rate $r$ per meiosis. The $\mathbf{A}$ locus is under sexually antagonistic selection, with the $A$ allele female-beneficial and male-deleterious, and the $a$ allele male-beneficial and female-deleterious. At the $\mathbf{M}$ locus, the $M_1$ allele has a relative fitness of $1$ in both sexes, while the $M_2$ allele causes sterility through the male sex function, and is completely dominant to the $M_1$ allele. Compared to hermaphrodites, female unisexuals re-allocate resources that are normally invested in the male sex function (e.g., to pollen production) towards ovule production (\citealt{Lloyd1975,Lloyd1976,Charlesworth1978a}), with $k$ representing the proportional increase in ovule production by unisexual females relative to hermaphrodites. The population is initially hermaphroditic (fixed for $M_1$); evolution of gynodioecy requires the invasion of the $M_2$ allele into the population. Cytoplasmic male sterility may also play a role in the evolution of gynodioecy, but transitions to full dioecy require the subsequent invasion of nuclear sterility alleles (e.g., \citealt{Lewis1941,Frank1989,Charlesworth2006,Charlesworth2002}).

Among offspring of hermaphrodites, a fixed proportion $C$ are produced by self-fertilization, and the remainder ($1 - C$) are produced by outcrossing. Unisexual females cannot self-fertilize. Individuals produced by self-fertilization suffer from inbreeding depression, with $\delta$ representing the decreased probability of survival to reproductive maturity of self-fertilized zygotes relative to those produced by outcrossing. Generations are discrete, and the life-cycle proceeds as follows: birth $\rightarrow$ differential survival due to inbreeding depression $\rightarrow$ selection on reproductive success through male and female sex functions $\rightarrow$ random fertilization.

Let $x_i$ and $y_i$ denote the frequencies of the four possible haplotypes $[AM_1,AM_2,aM_1,aM_2]$ in female and male gametes, respectively. An individual carrying haplotypes $i$ and $j$ has fitness of $w^f_{ij}$ through the female sex function, and $w^m_{ij}$ through the male sex function; fitness per sex function is multiplicative between the $\mathbf{A}$ and $\mathbf{M}$ loci (Table~\ref{tab:fitness}). We assume that there are no parent-of-origin effects on fitness.

To track evolutionary changes in genotype frequencies, we partitioned genotypes by whether they were produced by self-fertilization or by outcrossing. The approach yields a system of $20$ general recursion equations ($10$ genotypes $\times$ two modes of reproduction; see Appendix A). Although more compact approaches have been used for generating recursions in previous models (e.g., \citealt{Charlesworth1978a, Charlesworth2010}), we used the following expanded expressions to clarify the transmission pathway and assumptions underlying each of our models. As shown below, our models reproduce the classical results in special cases where there is only selection at loci segregating for sterility alleles.

Let $F_{ij}$ represent the zygotic frequency of outcrossed genotypes carrying haplotypes $i$ and $j$, and $G_{ij}$ represent the frequency of the same genotype among zygotes produced by self-fertilization. When $M_2$ is completely dominant to $M_1$, the recursions for outcrossed zygotes in the next generation simplify to:
\begin{linenomath}\begin{align*} \label{eq:FprGyn}
    F'_{11} &= (1 - S) (x_1 y_1)            & F'_{23} &= (1 - S) (x_2 y_3)  \\
    F'_{12} &= (1 - S) (x_2 y_1)            & F'_{24} &= 0 \\
    F'_{13} &= (1 - S) (x_1 y_3 + x_3 y_1)  & F'_{33} &= (1 - S) (x_3 y_3)  \\
    F'_{14} &= (1 - S) (x_4 y_1)            & F'_{34} &= (1 - S) (x_4 y_3)  \\
    F'_{22} &= 0                            & F'_{44} &= 0, \numberthis
\end{align*}\end{linenomath}

\noindent where $S$ is the proportion of all ovules in the population that are self-fertilized ($S$ is proportional to the selfing rate, $C$, and takes into account selection on ovule production and the fact that not all genotypes produce pollen; see Appendix A for details). Note that $y_2=y_4=0$ because these male gametic haplotypes cannot be produced when $M_2$ is a dominant male sterility allele. Among zygotes formed by self-fertilization, the genotypic frequencies in the next generation are: 
\begin{linenomath}\begin{align*} \label{eq:GprGyn}
    G'_{11} &= S (o^S_{11} + o^S_{13}/4) \\
    G'_{13} &= S (o^S_{13}/2) \\
    G'_{33} &= S (o^S_{33} + o^S_{13}/4), \numberthis
\end{align*} \end{linenomath}

\noindent where $o^S_{ij}$ are functions describing the proportional contribution of each genotype to self-fertilized ovules (see Appendix A). All $G'_{ij} = 0$ where $ij \neq [11,13,33]$. 

The basic form of the recursions does not change when $M_2$ is recessive, but there are two notable differences. First, because only $M_2M_2$ individuals are unisexual females, none of the recursions reduce to zero. Second, in contrast to the gynodioecy model with dominant $M_2$ allele, the recombination rate parameter ($r$) enters into the $G'_{ij}$ recursions and the functions $x_i$ and $y_i$ in the recessive $M_2$ model (see Appendix A).


%%%%%%%%%%%%%%%%%%%%%%%%
\subsection*{Androdioecy}

Androdioecy evolves when a $M_2$ allele causing female sterility invades a hermaphrodite population. In the androdioecy model, the reproductive compensation term, $k$, now describes the increase in pollen production by males relative to hermaphrodites. When the $M_2$ allele is dominant (i.e., $M_2$ carriers are male), the fitness expressions, $w^f_{ij}$ and $w^m_{ij}$ resemble those described in Table~\ref{tab:fitness}, except that the $\mathbf{M}$ locus affects the female rather than the male sex function. 

Recursions for the androdioecy models are similar to those for gynodioecy, with a few key differences. When $M_2$ is dominant, the recombination rate drops out of the expressions for haplotype frequencies in ovules ($x_i$; $x_2=x_4=0$, as in the gynodioecy model with dominant $M_2$ allele), whereas haplotype frequencies in pollen are partly dependent on recombination (functions for $y_i$ include $r$ terms). The genotypic frequencies in the next generation among zygotes formed by outcross fertilization, $F'_{ij}$, are otherwise identical to Eq(\ref{eq:FprGyn}), except that heterozygotes at the $\mathbf{M}$ locus do not produce ovules. The form of the $G'_{ij}$ recursions remains unchanged from Eq(\ref{eq:GprGyn}) (see Appendix A).

When the $M_2$ (female-sterility) allele is recessive, only $M_2 M_2$ homozygotes are unisexual males. The form of the recursions is similar to the case of gynodioecy with a recessive $M_2$ allele, but with the $\mathbf{M}$ locus now affecting female sterility. Expressions for the ovule and pollen haplotype frequencies, $x_i$ and $y_i$, partially depend on the recombination rate (see Appendix A).


%%%%%%%%%%%%%%%%%%%%%%%%
\subsection*{Analyses} \label{subsec:analyses}

Because they are analytically tractable for linear stability analyses, we used the dominant unisexual sterility allele models to determine conditions that favour the spread of sterility and SA alleles (see \hyperref[sec:Introduction]{Introduction}).  

Following previous SA theory (e.g., \citealp{Kidwell1977, Prout2000, JordanConnallon2014}), we focused our analyses on the simplest case of codominant (i.e., additive) expression of SA alleles at locus $\mathbf{A}$ ($h_m = h_f = 1/2$ in Table~\ref{tab:fitness}). Additional results exploring the effects of dominance reversals at the $\mathbf{A}$ locus ($h_m, h_f < 1/2$) are presented in Appendix C of the Online Supporting Information. These additional results are similar to those of the additive scenario, except that the invasion conditions for both SA and sterility alleles are more permissive (\citealt{Fry2010, JordanConnallon2014, Olito2017}).

We first examined the case where both loci are initially monomorphic ($M_1$ is fixed at the $\mathbf{M}$ locus; $A$ or $a$ is fixed at the $\mathbf{A}$ locus). To identify conditions for the invasion of derived alleles at each locus (individually or in combination), we carried out a linear stability analysis of each monomorphic condition corresponding to $F_{11} = G_{11} = 1$ when $A$ is initially fixed, and to $F_{33} = G_{33} = 1$ when $a$ is initially fixed. An equilibrium is unstable, and a derived haplotype will invade the population, when the leading eigenvalue of the Jacobian matrix of the system of recursions is greater than one ($\lambda_L > 1$) \citep{OttoDay2007}. For the models involving dominant unisexual sterility alleles, each analysis yields three analytically tractable candidate leading eigenvalues. The first two candidate eigenvalues ($\lambda_{\mathbf{A}}$ and $\lambda_{\mathbf{M}}$) describe conditions for invasion of the derived allele at each locus singly. The third candidate eigenvalue describes the condition for invasion of a haplotype carrying both derived alleles ($\lambda_{\mathbf{AM}}$). Mathematica code for our analytic results are presented in Appendix F of the online supplementary material.

We next explored how SA polymorphism influences the invasion of unisexual sterility alleles into a population of hermaphrodites. For this scenario, we evaluated whether the $M_2$ allele can invade a population at deterministic polymorphic equilibrium at the $\mathbf{A}$ locus, with $M_1$ initially fixed at the $\mathbf{M}$ locus. As we note in the discussion, the assumption of polymorphic equilibrium at the SA locus is made for convenience in the stability analyses, and is not a requirement for invasion of the unisexual sterility alleles. For the initial conditions of SA polymorphisms, we used the exact single-locus equilibria from \citet{Kidwell1977} for outcrossing populations with additive SA effects. We used approximate single-locus equilibria (see \citealt{JordanConnallon2014,ConnallonJordan2016}) for cases of partial selfing and/or nonadditive effects of SA alleles (the expressions for single-locus SA polymorphism are provided in Appendix B in the Online Supporting Information). The approximations compare well against exact numerical results when selection at the SA locus is modest to weak ($s_f,s_m \leq 0.5$; \citealt{JordanConnallon2014, ConnallonJordan2016, Olito2017}). 

Linear stability analyses are inconclusive when sterility alleles are recessive. Yet the evolutionary dynamics of recessive sterility alleles are of interest because such alleles are probably more common than dominant ones (\citealt{Charlesworth1978a}). 
We therefore carried out deterministic simulations of the exact recursions to find the equilibrium frequencies for each model of recessive unisexual sterility. To assess the impact of linkage on the polymorphic frequencies of unisexuals and hermaphrodites, we compared equilibrium frequencies from our models with the corresponding single-locus equilibria from \citet{Charlesworth1978a} (denoted $\hat{Z}$). For these comparisons, we focus on relatively tight linkage between $\mathbf{A}$ and $\mathbf{M}$ ($r\leq 0.1$). Predictions from one- and two-locus models converge under loose linkage (high $r$). 

In single-locus models of gynodioecy and androdioecy, the equilibrium frequency of unisexuals is determined by the reproductive compensation term, $k$, and the compound parameter $C \delta$ \citep{Charlesworth1978a}, where $C$ and $\delta$ refer to the rate of selfing and the severity of inbreeding depression, respectively. If inbreeding depression is caused primarily by recessive deleterious mutations, as current data suggest \citep{Charlesworth2009}, $C$ and $\delta$ should negatively covary, as increased selfing purges deleterious recessives (we note, however, that other processes also contribute to observed patterns of inbreeding depression, e.g.: \citealt{CrnokrakBarrett2002, Charlesworth-etal-2007, Charlesworth2009, HedrickGarcia-Dorado2016}). For simplicity, we incorporated negative covariance into our simulations by constraining inbreeding depression to follow a linear declining function of the selfing rate: $\delta = \delta^\ast(1 - C/2)$, where $\delta^\ast$ represents the hypothetical severity of inbreeding depression for an outcrossing population. We set $\delta^\ast = 0.8$ for all simulations, which resulted in levels of inbreeding depression that are consistent with empirical data (e.g., Fig. 2 in \citealt{HusbandSchemske1996}). We emphasize that our intention is not to formally model the evolution of inbreeding depression, but to capture effects of purging deleterious recessives in partially selfing populations, and thereby explore biologically meaningful parameter space for the evolution of andro- and gynodioecy. In fact, a variety of detailed genetic models give rise to a nonlinear relation between inbreeding depression and selfing (e.g., \citealt{OhtaCockerham1974, LandeSchemske1985, Charlesworth1985, Roze2015, Garcia-Dorado2017, LandePorcher2017}). As we show in Appendix E in the Online Supporting Information, such non-linear expressions for $\delta$ as a function of $C$ yield similar results to the linear expression described above. 

To take into account the fact that $C$ and $\delta$ also influence the maintenance of SA polymorphism \citep{JordanConnallon2014,Olito2017}, we ran simulations using values of $s_f$ and $s_m$ that correspond to equal equilibrium frequencies for the two SA alleles ($p = q = 1/2$). Thus, our simulations explore whether unisexual sterility alleles will invade hermaphroditic populations at polymorphic equilibrium with equal frequencies of male- and female-beneficial SA alleles. We reiterate, however, that such balanced SA polymorphisms are not strictly necessary for sterility alleles to invade (see \hyperref[sec:Discussion]{Discussion}).


%%%%%%%%%%%%%%%%%%%%%%%%
\section*{Results}
%%%%%%%%%%%%%%%%%%%%%%%%

%%%%%%%%%%%%%%%%%%%%%%%%
\subsection*{Invasion of unisexuals into monomorphic populations}

%%%%%%%%%%%%%%%%%%%%%%%%
\subsubsection*{Gynodioecy:} We begin with the simplest scenario for the evolution of gynodioecy: invasion of a dominant male-sterility allele, $M_2$, into a hermaphrodite population that is initially monomorphic for the SA locus (fixed for the $AAM_1M_1$ or the $aaM_1M_1$ genotype, which includes cases where selection favors the fixation of one of the SA alleles).  

A new haplotype can spread within the population under three conditions. First, when the single-locus invasion criteria for the rare SA allele is met ($\lambda_{\mathbf{A}} > 1$; \citealt{Kidwell1977, JordanConnallon2014, Olito2017}), selection will favour its invasion. This single-locus invasion condition applies broadly to cases of additive and non-additive effects of SA alleles.

Second, a male-sterile allele can invade in the absence of linkage to an SA allele when the classic single-locus criterion for the evolution of gynodioecy is met ($\lambda_{\mathbf{M}} > 1$):

\begin{equation}\label{eq:1LocGyn}
	k > 1 - 2 C \delta.
\end{equation}

\noindent (see Eq(4) of \citealt{Charlesworth1978a}). Recall that $k$ is the increased ovule production in females relative to hermaphrodites, and $\delta$ represents the strength of inbreeding depression in self-fertilized relative to outcrossed individuals. For convenience, we define $\hat{k}$ as the threshold level of reproductive compensation for invasion of a unisexual in the single-locus model; $\hat{k}$ is equal to the right-hand side of Eq(\ref{eq:1LocGyn}), and $k > \hat{k}$ is the single-locus condition for invasion.

Third, a mutant haplotype bearing both the male-sterile allele and the derived allele at the SA locus can invade when $\lambda_{\mathbf{AM}} > 1$. This condition for invasion takes the same basic form as Eq(\ref{eq:1LocGyn}), in which the minimum value of $k$ that is required for invasion is a decreasing function of $C \delta$ (see Appendix F). When the population is initially fixed for the female-beneficial SA allele ($A$), the haplotype invasion condition is more restrictive (requiring larger $k$) than the single-locus invasion condition (Eq(\ref{eq:1LocGyn})). In contrast, when the population is initially fixed for the male-beneficial allele ($a$), the condition for invasion of the derived haplotype is more permissive, provided the recombination rate is low relative to the strength of selection at the SA locus: 

\begin{equation}\label{eq:2LocGyn}
	r < \frac{s_f}{2 + s_f}.
\end{equation}

\noindent Eq(\ref{eq:2LocGyn}) shows that tight linkage, strong SA selection, or both, facilitates the invasion of unisexual females and the evolution of gynodioecy. Nevertheless, strong selection is not required for linkage to have a pronounced effect on the conditions for invasion of unisexuals. For example, when $s_f \leq 0.1$, the threshold $k$ for unisexual invasion decreases by up to $\approx 15 \%$ relative to single-locus predictions (fig.~\ref{fig:dblMutExpFig}A; see also Appendix D, fig.~D1). Stronger SA selection decreases the threshold for invasion of a unisexual sterility allele. Importantly, the above result holds whether the linked female-beneficial SA allele is ultimately fixed, or is maintained as a balanced polymorphism.

Linkage to a male-sterility allele also expands conditions for the invasion of female-beneficial SA alleles. Under obligate outcrossing, conditions for invasion of a derived haplotype with the female-benefit SA allele and male-sterility allele ($AM_2$; corresponding to $\lambda_{\mathbf{AM}} > 1$) are more permissive than the single-locus SA invasion condition when Eq(\ref{eq:2LocGyn}) is satisfied. Under partial selfing, the invasion condition for the same derived haplotype becomes more permissive than the single-locus prediction when:

\begin{equation}\label{eq:2LocGynSApartSelf}
	r < \frac{s_m (1 - C)}{2 + s_m + C (2 - s_m - 4 \delta)}.
\end{equation}

\noindent Eq(\ref{eq:2LocGynSApartSelf}) shows that selfing dampens the linkage-induced expansion of parameter space for invasion of female-beneficial SA alleles. This dampening effect occurs because increased selfing biases evolution in favour of the female-beneficial alleles \citep{Olito2017}; the bias leaves less parameter space available for linkage to further expand the conditions where female-beneficial SA alleles can invade. Nevertheless, tight linkage can still aid invasion of female-beneficial alleles provided there is some degree of outcrossing among hermaphrodites. 


%%%%%%%%%%%%%%%%%%%%%%%%
\subsubsection*{Androdioecy:} We identified conditions for the evolution of androdioecy by analyzing the stability of a population initially fixed for the $AAM_1M_1$ or the $aaM_1M_1$ genotype. We now follow the evolution of a dominant female sterility allele ($M_2$) and SA alleles with additive effects on the sex functions. Analysis of the first eigenvalue ($\lambda_{\mathbf{A}}$) recovers the general, single-locus invasion criterion for a derived SA allele (see above). Analysis of the second eigenvalue ($\lambda_{\mathbf{M}}$) yields the familiar single-locus condition for invasion of unisexual males into a hermaphrodite population:

\begin{equation}\label{eq:1LocAndro}
	k > \frac{1 + C (1 - 2 \delta)}{(1 - C)}.
\end{equation}

\noindent (see Eq(8) of \citealt{Charlesworth1978a}). As before, we define $\hat{k}$ as the threshold level of reproductive compensation for unisexual invasion in the single-locus model, with $\hat{k}$ equal to the right-hand side of Eq(\ref{eq:1LocGyn}) (for the gynodioecy model) or Eq(\ref{eq:1LocAndro}) (for the androdioecy model), as appropriate.

Haplotypes bearing derived alleles at both loci can invade when $\lambda_{\mathbf{AM}} > 1$, where invasion conditions are once again dependent upon the SA genotype that is initially fixed in the ancestral hermaphrodite population (see Appendix F). Mirroring the results of the gynodioecy model, the conditions for invasion are more restrictive than the single-locus condition (Eq(\ref{eq:1LocAndro})) when the population is initially fixed for the male-beneficial allele ($a$). Conditions for invasion are more permissive when the female-beneficial allele ($A$) is initially fixed. Replacing $s_f$ with $s_m$ in Eq(\ref{eq:2LocGyn}) yields the condition under which linkage to the SA locus promotes the invasion of unisexual males (i.e., relative to Eq(\ref{eq:1LocAndro})). The expansion of parameter space for invasion of the two-locus derived haplotype is smaller in the androdioecy model compared to the gynodioecy model (fig.~\ref{fig:dblMutExpFig}B). Again, this condition holds whether or not the invading male-beneficial allele will ultimately fix.

Linkage to a female-sterility allele also facilitates the invasion of male-beneficial SA alleles. With $s_m$ substituted for $s_f$, Eq(\ref{eq:2LocGyn}) provides the condition where linkage to a female-sterile allele expands the invasion condition of a male-beneficial SA allele in an obligate outcrossing population. Under arbitrary selfing in hermaphrodites, linkage to a female-sterility locus expands the invasion conditions of male-beneficial SA alleles when:

\begin{equation}\label{eq:2LocAndroSApartSelf}
	r < \frac{s_f(1 + C (1 - 2 \delta))}{2 - 2 C + s_f(1 + C (1 - 2 \delta))}.
\end{equation}

\noindent In contrast to the gynodioecy model, where selfing dampens the expansion of parameter space for the linked invasion of female-beneficial SA alleles (Eq(\ref{eq:2LocGynSApartSelf})), self-fertilization expands conditions for invasion of male-beneficial SA alleles that are linked to female-sterility alleles in the androdioecy model. For example, the right hand side of Eq(\ref{eq:2LocAndroSApartSelf}) increases with both $C$ and $s_f$ such that, for $C > 1/3$, there is always some expansion of the parameter space where male-beneficial alleles can invade relative to the single-locus expectation, even under free-recombination ($r = 1/2$) (see also \citealt{Olito2017}).



%%%%%%%%%%%%%%%%%%%%%%%%
\subsection*{Invasion of unisexuals into polymorphic populations}

Reproductive compensation, recombination, and selfing rates have similar effects on the invasion of sterility mutations into populations maintaining polymorphic SA alleles compared to populations where the SA locus is initially monomorphic (above). Linkage to a polymorphic SA locus expands the parameter conditions for the invasion of unisexuals relative to the classical single-locus conditions (fig.~\ref{fig:PrInv}). In cases where reproductive compensation falls below the single-locus invasion threshold, unisexuals can still invade when the sterility mutation and SA locus are sufficiently linked. For example, under obligate outcrossing and tight linkage ($r = 0$), unisexuals can invade across $\approx 69\%$ of the parameter conditions where SA polymorphism is maintained (within the range $s_f,s_m \leq 0.5$), even if they suffer a  $10\%$ reduction in gamete production relative to the single-locus invasion criterion for unisexual sterility alleles (the right hand side of Eq(\ref{eq:1LocGyn}); see fig.~\ref{fig:PrInv}A,D). With smaller reductions in gamete production relative to the single locus invasion criterion, unisexuals can invade across a greater fraction of polymorphic SA parameter space, even when linkage is relatively loose (e.g.,  when $r = 0.2$, and $k = 0.95 \times \hat{k}$, unisexuals will invade across ~$\approx 38\%$ of parameter conditions maintaining SA polymorphism). 

With a small amount of selfing and high inbreeding depression in hermaphrodites ($C = 1/4,~\delta = 0.8$), the breadth of parameter space where unisexuals can invade remains similar to the obligate outcrossing scenario (fig.~\ref{fig:PrInv}B,E). However, under high selfing rates and low inbreeding depression ($C = 3/4,~\delta = 0.2$), tighter linkage is required for invasion of male-sterility alleles during transitions to gynodioecy than for female-sterility alleles during transitions to androdioecy, when reproductive compensation falls below the single locus invasion threshold (fig.~\ref{fig:PrInv}C,F). That is, effect of linkage to an SA locus has a stronger facilitating effect on transitions to androdioecy than for transitions to gynodioecy. As before, this contrasting effect of selfing between the gynodioecy and androdioecy models arises because selfing causes asymmetry in the evolutionary dynamics of SA alleles, which favours female-benefit relative to male-benefit alleles (\citealt{ Charlesworth1978a, JordanConnallon2014}; see Appendix D, figs.~D2--D7 for examples). The evolutionary bias towards female-beneficial alleles creates greater scope for linkage to a male-beneficial SA allele to aid invasion of a female-sterility allele (resulting in androdioecy) than for linkage to a female-beneficial allele to aid the spread of a male-sterility allele (resulting in gynodioecy).

SA genetic variation will facilitate the spread of loosely-linked unisexual sterility alleles when selection at the SA locus is strong (larger values of $s_m$ and $s_f$), and selection is biased towards the sex of the invading unisexual type. In other words, with incomplete linkage the invasion of male-sterility alleles requires female-biased selection ($s_f > s_m$), while the invasion of female-sterility requires male-biased selection ($s_f < s_m$; see Appendix D, figs.~D2--D7 in the Online Supporting Information). 

%%%%%%%%%%%%%%%%%%%%%%%%
\subsection*{Equilibrium frequencies of unisexuals when sterility alleles are recessive}

Relative to classical single-locus predictions, linkage with an SA locus elevates the equilibrium frequencies of recessive sterility alleles and unisexuals, consistent with results of the dominant sterility allele models. Indeed, when reproductive compensation matches or exceeds the threshold for unisexual invasion in the single-locus model (i.e., $k \geq \hat{k}$; where $\hat{k}$ is equal to the right-hand side of Eq(\ref{eq:1LocGyn}) or Eq(\ref{eq:1LocAndro}) as appropriate), linkage to an SA locus can greatly increase the equilibrium frequencies of unisexuals relative to single-locus predictions (\citealt{Charlesworth1978a}; figs.~\ref{fig:eqFreq2v1Loc} and D15, greyscale lines). The effect is strongest in predominantly outcrossing populations, and weakens with increasing rates of selfing in hermaphrodites (recall that inbreeding depression also declines with selfing in our models; see \hyperref[sec:Models]{Models}). As the recombination rate between loci increases, unisexual equilibrium frequencies converge to predictions of the single-locus models.

When reproductive compensation falls below the single-locus threshold for invasion ($k < \hat{k}$), the combination of tight linkage and outcrossing can still elevate the equilibrium frequencies of unisexuals (fig.~\ref{fig:eqFreq}A,C, with the effect declining with the recombination rate, the rate of outcrossing, and the magnitude of reproductive compensation: fig.~\ref{fig:eqFreq}B,D). In these cases, the frequencies of males in the androdioecy model are more sensitive to reductions in $k$ than for the equilibrium frequencies of females in the gynodioecy model. With lower reproductive compensation, equilibrium unisexual frequencies are highest for populations with intermediate selfing rates, reflecting a balance between two opposing factors that influence the evolution of recessive sterility alleles: whereas selfing increases both the expression of unisexual phenotypes (by increasing the frequency of $M_2 M_2$ homozygotes) and the opportunity for selection for rare unisexual sterility alleles, it also places a limit on the reproductive success of unisexuals, which can only reproduce by outcrossing.





%%%%%%%%%%%%%%%%%%%%%%%%
\section*{Discussion} \label{sec:Discussion}
%%%%%%%%%%%%%%%%%%%%%%%%

Our theoretical predictions have three implications for the evolution of mating systems and the genetic basis of separate sexes. First, compared to classical single-locus models of andro- and gynodioecy, linkage to an SA locus expands the range of conditions where unisexual sterility alleles are favoured, and elevates the equilibrium frequencies of invading unisexual individuals within andro- and gynodioecious populations. Second, the facilitating effect of linkage on the invasion of unisexuals is most pronounced in outcrossing hermaphrodite populations, suggesting that correlations between the ancestral selfing rate and the evolution of separate sexes may be weaker than predicted by earlier theory, in which selfing promotes invasion of unisexuals (see \citealt{Charlesworth1978a}). Third, successfully invading unisexual sterility alleles, which initiate transitions to dioecy, are expected to preferentially accumulate in genomic regions harboring SA polymorphism -- a prediction at odds with the prevailing view that sexual antagonism is more likely to follow rather than precede the evolution of separate sexes and of genetic sex determination (reviewed in \citealt{CharlesworthEtAl2005,Bachtrog2006}). Below, we elaborate on the predictions of our models and suggest ways to test them.

%%%%%%%%%%%%%%%%%%%%%%%%
\subsection*{SA Polymorphism and the evolution of dimorphic sexual systems} 

%Although a small fraction of angiosperms are dioecious ($5-6\%$ of all species; \citealt{Renner2014}), they nevertheless exhibit a fantastic diversity of sexual systems. Moreover, the repeated and independent evolutionary transitions from hermaphroditism to androdioecy, gynodioecy, and full dioecy, as observed across the phylogeny of flowering plants, \hl{demands} a genetical explanation \citep{Renner2014,KaferPannell2017}. Several evolutionary pathways and genetic mechanisms may lead from hermaphroditism to separate sexes, but all ultimately involve the evolutionary invasion of at least two unisexual sterility alleles with complementary effects on sex determination: at least one leading to unisexual females and the other leading to unisexual males \citep{Charlesworth1978a, Charlesworth1978b, Renner2014, Ashman2015}. 

Classical theory predicts that conditions for the evolution of dioecy are likely to be restrictive \citep{Lloyd1975,Lloyd1976,Charlesworth1978a,KaferPannell2017}. Unisexuals can invade a population when they compensate for the loss of one ancestral sex function by increasing gamete production through the other (e.g., unisexual females can compensate for the loss of the male sex function by producing more ovules than hermaphrodites do). We find that when sterility mutations arise on haplotypes that carry complementary SA alleles, the conditions for the spread of unisexual sterility requires less compensation than predicted by previous theory (e.g., a male sterility allele is more likely to spread when it arises on a haplotype carrying a female-beneficial SA allele). Intuitively, linkage to a complementary SA allele helps to offset the loss of a sex function caused by the sterility allele, thereby reducing the amount of reproductive compensation required for unisexual invasion relative to single-locus predictions \citealt{Charlesworth1978a}). These results parallel recent multilocus models of sex-specific selection, which show that predominantly female-harming mutations tend to accumulate on haplotypes carrying male-benefit SA alleles, and male-harming mutations preferentially accumulate in linkage with female-beneficial SA alleles \citep{ConnallonJordan2016, Patten2010, UbedaPatten2010, BlackburnOtto2010}. A corollary of this result is that conditions for the maintenance of SA polymorphism are also expanded by linkage to sex-specific sterility mutations. Hence, the invasion of new unisexual sterility mutations (i.e., during the initial step in the evolution of dioecy) should also promote the maintenance of polymorphism at nearby SA loci. 

Balanced SA polymorphisms are not a prerequisite for linkage between an SA locus and a unisexual sterility mutation to facilitate transitions to andro- or gynodioecy. However, the amount of standing SA genetic variation should influence the potential for unisexuals to invade hermaphrodite populations lacking balanced polymorphisms. In hermaphrodite populations that are initially monomorphic at SA loci, selection may favour the invasion of haplotypes that carry unisexual sterility alleles and a rare complementary SA allele. Yet the waiting time until such "double-mutant" haplotypes arise may be long, particularly within small populations where mutational variation is limited \citep{WeinreichChao2005,ConnallonClark2010}. However, while the origin and invasion of double-mutant haplotypes is improbable on generational timescales, such events become probable over the longer time intervals over which reproductive systems evolve. Consequently, such events may play a meaningful role in the evolution of dimorphic sexual systems. In addition, unisexual sterility alleles should readily invade hermaphrodite populations that are polymorphic for SA alleles. Such SA polymorphism could potentially segregate in hemaphrodite populations as balanced polymorphisms or as transient polymorphisms evolving under recurrent mutation, purifying selection and genetic drift (see below).

Although we currently know very little about the extent of SA genetic variation in hermaphroditic species (e.g., \citealt{Abbott2011,Olito-etal-2018}), several features of SA selection suggest that such variation may be common, particularly in large populations. First, theory suggests that both outcrossing and linkage between SA loci should promote the evolution of balanced SA polymorphisms in hermaphrodite populations, especially in heterogeneous environments \citep{JordanConnallon2014, Olito2017, Olito-etal-2018}. Second, SA alleles that are destined to eventually become fixed or lost are expected to segregate for many generations compared to unconditionally deleterious or beneficial alleles \citep{ConnallonClark2012}. The long transit times for SA alleles provide an opportunity for unisexual sterility mutations to arise on haplotypes bearing SA alleles that may ultimately be lost (e.g., \citealt{WeinreichChao2005}). However, empirically identifying SA loci is challenging, even within dioecious species under strong SA selection, where few candidate loci have been identified (e.g., \citealt{Barson2015, LeeKelly2015}; reviewed in \citealt{ConnallonHall2018, Mank2017}). Further empirical attention to SA genetic variation in hermaphroditic species would help clarify the potential for linked SA selection to facilitate transitions to dioecy. An important caveat to our models of andro- and gynodioecy is that several mechanisms, including the evolution of sex-specific gene expression \citep{Vicoso2013}, may resolve the SA selection before the appearance of linked unisexual sterility mutations.


%%%%%%%%%%%%%%%%%%%%%%%%
\subsection*{Mating systems and the evolution of dioecy}

The interplay between hermaphrodite mating systems and reproductive compensation is a key factor influencing the evolution of separate sexes in flowering plants, especially via the gynodieocy pathway \citep{Darwin1877,Charlesworth1978a}. Previous theory predicts that the evolution of gynodioecy is driven by the combination of reproductive compensation by unisexuals and inbreeding avoidance, and is therefore most likely to occur in partially selfing populations \citep{Lewis1942, Lloyd1975, Charlesworth1978a, KaferPannell2017}. This prediction follows directly from the structure of Eq(\ref{eq:1LocGyn}) where $\hat{k}$ depends entirely on the product of the selfing rate and inbreeding depression ($C \delta$). In contrast, the evolution of androdioecy is predicted to require significantly higher reproductive compensation, especially in partially selfing populations, because invading males must still compete with selfing hermaphrodites to fertilize ovules \citep{Charlesworth1978b, KaferPannell2017}. 

The population selfing rate plays a critical role in our models as well, though our predictions differ from those of classical theory. The facilitating effect of linkage to an SA locus on the invasion of unisexual individuals is most prominent in outcrossing populations, and the effect of linkage weakens with increased selfing. Moreover, outcrossing hermaphrodite populations are more likely than selfing ones to maintain SA genetic variation \citep{JordanConnallon2014,Olito2017}. Both factors should promote evolutionary transitions from outcrossing hermaphrodite populations to andro- and gynodioecious systems. Thus, single-locus theory for the initial steps towards the evolution of dioecy may significantly underestimate the potential for evolutionary transitions to dimorphic sexual systems in predominantly outcrossing ancestral species. On the other hand, our models agree with previous theory that the conditions favouring transitions to androdioecy will remain difficult to satisfy when the ancestral rate of self-fertilization is high (large $k$; \citealt{Charlesworth1978a}, Eq(\ref{eq:1LocAndro})). Hence, our results comport with prior theory in predicting that the evolution of androdioecy -- and of dioecy via the androdioecy pathway -- will remain more restrictive than transitions to gynodioecy \citep{Charlesworth1978a, Charlesworth2006, KaferPannell2017, Renner2014}.

Although the classic single-locus predictions for the evolution of andro- and gynodioecy are widely accepted, one of the key predictions -- an association between the ancestral selfing rate and evolutionary transitions to dioecy -- is not well supported empirically \citep{Charlesworth1985, Charlesworth2006, Renner2014}. Our models can account for the weak association between the ancestral selfing rate and transitions to dioecy. In species where unisexual sterility alleles arise in linkage with an SA locus, evolutionary transitions to dimorphic sexual systems should be elevated in predominantly outcrossing, hermaphrodite taxa. If transitions to dioecy involve a mixture of the traditional (i.e., single-locus) path and the alternative that we have proposed,  then the net effect will be a low (or no) correlation between the ancestral hermaphrodite selfing rate and transitions to dieocy. In light of our results, it would be worth re-examining the association between the inferred pattern of ancestral self-fertilization and the transition rate between sexual systems among angiosperm species, using modern phylogenetic comparative methods. Such a study would further help to identify species where our proposed mechanism for the evolution of gynodioecy is most likely to have played a role (e.g., gynodioecious and dioecious species that appear to have evolved from outcrossing ancestors).


%%%%%%%%%%%%%%%%%%%%%%%%
\subsection*{The population genetic basis of sex chromosomes and separate sexes}

Our results also have implications for the role of SA genetic variation during the early stages of sex chromosome evolution. The process of sex chromosome evolution into highly-differentiated X and Y (or Z and W) chromosomes is thought to proceed by the following series of steps. First, a new sex-determination gene (or linked gene cluster) originates on an ordinary pair of autosomes. Second, SA alleles benefitting the heterogametic sex are predicted to accumulate in tight linkage with the sex-determining locus on the proto-Y or proto-W chromosome. Third, this linked SA variation generates selection for suppressed recombination between the proto sex-chromosomes. Fourth, the non-recombining sex chromosome degenerates as a consequence of recombination suppression. Finally, mechanisms of dosage compensation may evolve in response to the loss of functional genes on the degenerate sex chromosome \citep{Rice1987, Charlesworth2002, CharlesworthEtAl2005, Bachtrog2006, Qiuetal2013, Bachtrog2014}. 

In the classical theory for the evolution of dioecy \citep{Charlesworth1978a}, the initial step of sex chromosome evolution, as outlined above, coincides with the transition to full dioecy via the sequential invasion of complementary gender modifiers in a tightly-linked region of the genome, and complete repulsion linkage disequilibrium with one another (e.g., {proto-Y and proto-X chromosomes carry female-sterility and male-sterility alleles, respectively; \citealt{Charlesworth1978a}). The tightly coupled pair of loci segregate, effectively, as a single sex-determining locus, which sets the stage for further sex chromosome differentiation by the steps outlined above.

Our theoretical results suggest an alternative -- and previously unrecognized -- sex-chromosome evolutionary pathway. In contrast to the standard view, where sexual antagonism emerges after the origin of the sex-determination locus, our models suggest that SA polymorphism may represent the first step, rather than an intermediate step, in the evolution of heteromorphic sex chromosome systems. This role of SA genetic variation during the origin of sex chromosomes is reminiscent of recent theory for the turnover of established sex chromosomes \citep{vanDoornKirkpatrick2007,vanDoornKirkpatrick2010}. However, in our models SA polymorphism precedes the origin of the sex determination system. Following the transition to dioecy, the proto sex chromosomes are already primed for the evolution of recombination suppression and further differentiation \citep{Charlesworth1978a,Rice1987,Bachtrog2006,Qiuetal2013}. Given the crucial role of genetic linkage in nearly every step of sex chromosome evolution, it seems plausible that the amount of SA genetic variation present prior to the origin of distinct sexes may help to explain the extensive variation in sex-chromosome differentiation that has been observed in angiosperms \citep{Charlesworth2002,Renner2014,Bachtrog2014}.


%%%%%%%%%%%%%%%%%%%%%%%%
\subsection*{Acknowledgements}
This research was supported by a Monash University Dean's International Postgraduate Scholarship, Postgraduate Publication Award, and a Wenner-Gren Postdoctoral Fellowship to C.O., and Australian Research Council funds to T.C. We thank C.~Venables, M.~Patten, H.~Spencer, the SexGen group at Lund Unviersity, L.~Rowe, and two anonymous reviewers for valuable feedback on earlier versions of the manuscript. C.O. conceived the study, developed the models, performed the analyses, and wrote the manuscript. T.C. assisted in developing the models and writing the manuscript.


%%%%%%%%%%%%%%%%%%%%%
% Bibliography
%%%%%%%%%%%%%%%%%%%%%
\bibliography{dioecySA-bibliography}

\newpage


%%%%%%%%%%%%%%%%%%%%%%%%%%%%%%%%%%%%%%%%%%%%%%%%%%%%%%%%%%%%%%%%%%
%  Tables 

\begin{table}[htbp]
\centering
\caption{\bf Fitness expressions for diploid adults prior to reproduction for the model of a dominant male-sterility mutation ($w^f_{ij}$ denotes fitness effects through the female sex function , $w^m_{ij}$ for male sex function).}
\begin{tabular}{l c c c c} \hline
Haplotype & $ AM_1$ & $ AM_2$ & $ aM_1$ & $ aM_2$ \\
\hline
Female sex function & & & & \\
$ AM_1$ & $1$ & $(1 + k)$ & $(1 - h_f s_f)$        & $(1 - h_f s_f)(1 + k)$ \\
$ AM_2$ & $-$ & $(1 + k)$ & $(1 - h_f s_f)(1 + k)$ & $(1 - h_f s_f)(1 + k)$ \\
$ aM_1$ & $-$ & $-$       & $(1 - s_f)$            & $(1 - s_f)(1 + k)$ \\
$ aM_2$ & $-$ & $-$       & $-$                    & $(1 - s_f)(1 + k)$ \\
Male sex function & & & & \\
$ AM_1$ & $(1 - s_m)$ & $0$ & $(1 - h_m s_m)$ & $0$ \\
$ AM_2$ & $-$         & $0$ & $0$             & $0$ \\
$ aM_1$ & $-$         & $-$ & $1$             & $0$ \\
$ aM_2$ & $-$         & $-$ & $-$             & $0$ \\
\hline
\end{tabular}
\label{tab:fitness}\\
{\footnotesize Note: Rows and columns indicate the \textit{i}th and \textit{j}th gametic haplotype respectively. The lower triangle of each matrix is the reflection of the upper triangle, and is omitted for simplicity and consistency with the $i \geq j$ row/column indexing used throughout the article.}
\end{table}
\newpage{}



%%%%%%%%%%%%%%%%%%%%%%%%%%%%%%%%%%%%%%%%%%%%%%%%%%%%%%%%%%%%%%%%%%
%  Figures 


\begin{figure}[htbp]
\centering
\includegraphics[width=\linewidth]{./DblMutExpansionFig}
\caption{Effect of linkage between the $\mathbf{A}$ and $\mathbf{M}$ loci on the invasion of unisexual females (panel A) and males (panel B) into hermaphrodite populations. Plots show the increase in parameter space (calculated as a proportion of the parameter space where Eq(\ref{eq:1LocGyn}) is satisfied) for the invasion of 'double mutants' (i.e., based on $\lambda_{\mathbf{AM}}$) compared with the single locus invasion criteria (i.e., based on $\lambda_{\mathbf{M}}$; see Eq(\ref{eq:1LocGyn}) and Eq(\ref{eq:1LocAndro}) for the models of gynodioecy and androdioecy respectively) for different selection intensities as a function of the recombination rate between the two loci. We vary the recombination rate between $0$ and the threshold where the one- and two-locus invasion conditions become equivalent, Eq(\ref{eq:2LocGyn}). For the model of gynodioecy, we first solved $\lambda_{\mathbf{AM}} > 1$ for $k$ (denoted $\hat{k}_{\mathbf{AM}}$ for convenience), and then integrated both $\hat{k}_{\mathbf{AM}}$ and $\hat{k}$ (the right side of Eq(\ref{eq:1LocGyn})) over $C = [0,1]$ and $\delta = [0,1]$ for the specificed values of $r$ and $s_j$ (where $j \in \{f,m\}$). The increase in parameter space (decrease in $\hat{k}_{\mathbf{AM}}$ relative to $\hat{k}$) was then calculated by subtracting the resulting volume for $\hat{k}_{\mathbf{AM}}$ from that of $\hat{k}$ and dividing by the latter. Thus, the parameter space where unisexual females can invade is increased in the two-locus model because linkage reduces $\hat{k}_{\mathbf{AM}}$ relative to $\hat{k}$ across biologically plausible values of $C$ and $\delta$ (see Fig.~D1 in the online supplementary material). Because the integral of Eq(\ref{eq:1LocAndro}) is undefined when $C = 1$, we restricted the integration to include high, but not complete, selfing rates $C = [0, 0.9]$ for the model of androdioecy; comparisons between the models should therefore be made with caution.}
\label{fig:dblMutExpFig}
\end{figure}
\newpage{}

\begin{figure}[htbp]
\centering
\includegraphics[scale=0.57]{./EQInvFig-R1}
\caption{Invasion of unisexuals into populations with pre-existing SA polymorphism. Plots show the fraction of parameter conditions maintaining single-locus SA polymorphism (within the range $0 < s_f,s_m \leq 0.5$) where a dominant sex-specific sterility allele at $\mathbf{M}$ can invade, assuming additive SA fitness effects at the SA $\mathbf{A}$ locus ($h_f=h_m=1/2$), plotted as a function of the recombination rate $r$. Panels A--C show results from the model of gynodioecy via invasion of a male-sterility allele, while planels D--F show results for the model of androdioecy via invasion of a female-sterility allele. For each panel, results are shown for different values of reproductive compensation, $k$, chosen as a fraction of the single-locus invasion threshold for the unisequal sterility allele ($\hat{k}$, which is equal to the right-hand side of Eq(\ref{eq:1LocGyn}) or Eq(\ref{eq:1LocAndro}) for the models of gynodioecy and androdioecy respectively). Note the different scale for the x-axis in panel C. To calculate the fraction of parameter space where unisexuals can invade populations initially polymorphic at the SA locus we first drew $1000$ points uniformly distributed throughout polymorphic $s_f \times s_m$ parameter space (i.e., where the single-locus equilibrium frequency of the female-beneficial SA allele, $\hat{q}$, satisfied $0 < \hat{q} < 1$). The fraction of parameter space was calculated as the fraction of these $1000$ points at which either of the candidate leading eigenvalues associated with invasion of unisexuals ($\lambda_{\mathbf{M}}$, $\lambda_{\mathbf{AM}}$) evaluated in excess of one.}
\label{fig:PrInv}
\end{figure}
\newpage{}

\begin{figure}[htbp]
\centering
\includegraphics[width=\linewidth]{./EqFreqBigK}
\caption{Effect of linkage on equilibrium frequencies of unisexual (A) females, and (B) males when reproductive compensation is above the single-locus threshold for invasion of unisexual sterility allele (i.e., $k > \hat{k}$; where $\hat{k}$ is equal to the right-hand side of Eq(\ref{eq:1LocGyn}) or Eq(\ref{eq:1LocAndro}) as appropriate). Results are shown for the models of andro- and gynodioecy via invasion of recessive unisexual sterility alleles, additive fitness effects at $\mathbf{A}$ ($h_f = h_m = 0.5$, using selection coefficients of $s_m = 0.1$ (for the model of gynodioecy) and $s_f = 0.1$ (for the model of androdioecy), and inbreeding depression that follows $\delta = \delta^\ast(1 - C/2)$ (see \hyperref[sec:Models]{Models}). Plots illustrate the increase in equilibrium frequencies of unisexuals predicted by our two-locus models (dashed greyscale lines) relative to the corresponding exact single-locus equilibrium frequencies (solid black line; $\hat{Z}$ predicted by \citealt{Charlesworth1978a}). Results are shown for four different levels of recombination, highlighting that with weaker linkage, the two-locus predictions converge on those of the single-locus model.}
\label{fig:eqFreq2v1Loc}
\end{figure}
\newpage{}

\begin{figure}[htbp]
\centering
\includegraphics[scale=0.75]{./EqFreqAllKFig}
\caption{Equilibrium frequencies of unisexual (A--B) females and (C--D) males across a gradient of reproductive compensation (with values chosen as a fraction of the single-locus invasion threshold for $M_2$ ($\hat{k}$, which is equal to the right-hand side of Eq(\ref{eq:1LocGyn}) or Eq(\ref{eq:1LocAndro}) for the models of gynodioecy and androdioecy respectively). Results are shown for the models of andro- and gynodioecy via invasion of recessive unisexual sterility alleles, additive fitness effects at $\mathbf{A}$ ($h_f = h_m = 0.5$, using selection coefficients of $s_m = 0.1$ (for the model of gynodioecy) and $s_f = 0.1$ (for the model of androdioecy), and inbreeding depression that decreases linearly with the selfing rate: $\delta = \delta^\ast(1 - C/2)$ (see \hyperref[sec:Models]{Models}). Plots show the equilibrium frequencies of unisexuals predicted by our models for five different levels of reproductive compensation, calculated as a fraction of the single-locus invasion criterion for $M_2$ defined by Eq(\ref{eq:1LocGyn}) and Eq(\ref{eq:1LocAndro}). Note that the single-locus equilibrium frequency of unisexuals always equals $0$ when $k < \hat{k}$. Hence, the lines corresponding to $k < \hat{k}$ illustrate how linkage among SA loci expands the parameter space where unisexual sterility alleles can invade beyond the predictions of the single-locus models.}
\label{fig:eqFreq}
\end{figure}


\end{document}
