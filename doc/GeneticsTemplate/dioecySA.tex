\documentclass[9pt,twocolumn,twoside,lineno]{gsajnl}
% Use the documentclass option 'lineno' to view line numbers

% Other useful packages
\usepackage{color,soul}

% Equation numbering
\newcommand\numberthis{\addtocounter{equation}{1}\tag{\theequation}}

\articletype{inv} % article type
% {inv} Investigation 
% {gs} Genomic Selection
% {goi} Genetics of Immunity 
% {gos} Genetics of Sex 
% {mp} Multiparental Populations

\title{Sexually antagonistic polymorphism and the evolution of dimorphic sexual systems from hermaphroditism}

\author[$\ast$,1]{Colin Olito}
\author[$\ast$]{Tim Connallon}
% \author[$\dagger$]{Author Two}
% \author[$\ddagger$]{Author Three}
% \author[$\S$]{Author Four}
% \author[$\ast\ast$]{Author Five}

\affil[$\ast$]{Department of Biological Sciences, Monash University, Melbourne, VIC 3800, Australia}
% \affil[$\dagger$]{Author two affiliation}
% \affil[$\ddagger$]{Author three affiliation}
% \affil[$\S$]{Author four affiliation}
% \affil[$\ast\ast$]{Author five affiliation}

\keywords{Androdioecy; Gynodioecy; Intralocus sexual conflict; linkage disequilibrium; recombination; sexual system; sexual dimorphism}

\runningtitle{GENETICS Journal Template on Overleaf} % For use in the footer 

\correspondingauthor{Colin Olito}

\begin{abstract}
Although the majority of flowering plants are hermaphroditic, where individuals perform both male and female sex-functions, plant sexual systems form a diverse spectrum encompassing simultaneous hermaphroditism, to physically separate sexes (dioecy), and nearly every possible intermediate state. One major evolutionary pathway from hermaphroditism towards separate sexes is via the sequental invasion of a sex-specific sterility mutations, which ultimately become linked to form a sex-determining-locus. Classic population genetic theory underpinning this evolutionary pathway predicts (1) that for the first sterility mutation to invade a population of hermaphrodites there must be accompanying compensation in sex-function for unisexuals, (2) invasion of the sterility mutation is facilitated by higher self-fertization rates and stronger inbreeding depression, and (3) that invasion of male-sterility mutations is more permissive than for female-sterility mutations. However, sterility mutations can be viewed as a class of strongly deleterious sexually antagonistic (SA) alleles. Recent theory predicts that linkage and 'genetic architecture' can strongly affect the evolution of SA alleles. Here, we explore the theoretical consequences of physical linkage between a 'standard' SA locus and a second 'sterility' locus for the invasion of sex-specific sterility mutations into hermaphrodite populations. We show that linkage results in significantly more permissive parameter conditions for the invasion of unisexuals, and elevated equilibrium frequencies of unisexuals, relative to the classic single-locus predictions. Overall, our results suggest that sterility mutations driving the evolution of dimorphic sexual systems may evolve in tight linkage with pre-existing polymorphic SA loci, with potentially interesting implications for the process of early sex-chromosome evolution.
\end{abstract}

\setboolean{displaycopyright}{true}

\begin{document}

\maketitle
\thispagestyle{firststyle}
\marginmark
\firstpagefootnote
\correspondingauthoraffiliation{Department of Biological Sciences, Monash University, Melbourne, VIC 3800, Australia. Email: colin.olito@gmail.com.}
\vspace{-11pt}%

\lettrine[lines=2]{\color{color2}A}{}lthough the majority of flowering plants are hermaphroditic, where individuals perform both male and female sex-functions, plant sexual systems form a diverse spectrum ranging from simultaneous hermaphroditism, to physically separate sexesdioecy (dioecy) \citep{Darwin1877,Westergaard1958,Bachtrog2014}. Moreover, this diversity of sexual dimorphism encompasses nearly every possible intermediate state, from hermaphrodites bearing various combinations of perfect and/or single-sex flowers (different forms of monoecy), to mixed populations of hermaphrodites and unisexual females (gynodioecy) or males (androdioecy) \citep{Bawa1980,SakaiWeller1999}.

Dimorphic sexual systems have evolved from hermaphrodite ancestors repeatedly, and independently, in numerous plant lineages (\citealt{Westergaard1958,SakaiWeller1999,Charlesworth2006,Bachtrog2014,Renner2014}; although reversals may also be common \citealt{GoldbergOtto2017,KaferPannell2017}). Several evolutionary pathways can lead from hermaphroditism to dioecy, but in all cases at least two steps are required: invasion of male- or female-sterile unisexuals (resulting in gyno- and androdioecy respectively), and the subsequent invasion of one or more mutations causing sterility for the-sex function not rendered sterile by the first mutation \citep{Westergaard1958,Charlesworth1978a,Charlesworth1978b,Charlesworth2006,Charlesworth2009,KaferPannell2017}. Several lines of empirical evidence support the notion that two or more loci are involved in plant sex determination. These include crosses between dioecious species and hermaphroditic or monoecious sister taxa \citep{Westergaard1958}, \hl{as well as direct cytological and cromosome mapping studies}, and patterns of recombination supression in species with sex chromosomes \citep{Charlesworth2002,Charlesworth2006,Renner2014,Ashman2015}.

Several classic theoretical studies have formalized the logic underpinning the evolutionary pathway from hermaphroditism to dioecy. Three key predictions regarding the initial invasion of unisexuals are that (1) invasion of a sterility mutation requires accompanying reproductive compensation through the remaining sex-function in unisexuals relative to hermaphrodites, (2) invasion of male-steriliy mutations (resulting in gynodioecy) is facilitated by higher self-fertilization and stronger inbreeding depression, and (3) the conditions necessary for invasion of male-sterility mutations are more permissive than those for female-sterility mutations (resulting in androdioecy) (\citealt{Lewis1941,Lloyd1975,Lloyd1976,Charlesworth1978a}; see \citealt{Charlesworth1999,Charlesworth2006} for comprehensive reviews of theory). Additionally, the subsequent invasion of complete or partial sterility mutations resulting in dioecy is facilitated by, and may often require, tight linkage with the first sterility mutation \citep{Charlesworth1978a}. The invasion of tightly linked complimentary sterility mutations sets the stage for further sex-chromosome differentiation via the accumulation of sexually antagonistic genes linked to the new sex-determining region, and the evolution of reduced recombination between the new pair of sex-chromosomes \citep{Rice1987,Bachtrog2006,Charlesworth2002,Qiuetal2013}. 

Sex-specific sterility mutations are sexually antagonistic (SA) alleles. When they cause complete sterility, they are a specific class of SA allele causing reproductive failure for one sex, and possibly beneficial effects for the opposite sex (e.g., via reproductive compensation; \citealt{Lewis1941,Lloyd1975,Charlesworth1978a}). Recent theory indicates that the genetic architecture of SA genes can strongly influence the evolution of SA alleles. For dioecious species, linkage between two SA loci expands the parameter space where polymorphism is maintained by selection, and inflates SA fitness variance beyond the predictions of single-locus theory \citep{Kidwell1977,Patten2010,UbedaPatten2010}. For partially selfing hermaphrodites, the conditions permitting SA polymorphism become more restrictive, due in part to an increasing female-bias in the net direction of selection introduced by selfing \citep{JordanConnallon2014}. In a two-locus context, however, the simultaneous reduction in the effective rate of recombination caused by selfing significantly expands the opportunity for balancing selection to maintain SA polymorphism relative to single-locus models \citep{Olito2017}. This occurs primarily because linkage facilitates the invasion of male-beneficial alleles, partially compensating for the female-bias in selection created by selfing. Although definitively indentifying SA genes is notoriously difficult \citep{ConnallonClark2012,Barson2015}, there is some recent empirical support for these predictions. For example, a chomosomal inversion with apparently SA fitness effects has recently been identified, and appears to be segregating at non-trivial frequencies, in populations of partially selfing \textit{Mimulus guttatus} \citep{LeeKelly2015}.

Considering the evolution of sex-specific sterility alleles from the perspective of SA selection theory raises several important questions regarding the genomic architecture of SA loci. Does linkage with another SA locus facilitate the invasion of sterility alleles, for example, by promoting selection on linked allelic combinations (e.g., female-beneficial with male-sterility, and male-beneficial with female-sterility alleles)? Can unisexuals more easily invade hermaphrodite populations in which a pre-existing SA polymorphism is segregating? Does linkage lead to elevated equilibrium frequencies of unisexuals in the resulting gyno- or androdoiecious populations? In many respects, these questions echo the premise of a 'linkage constraint' for the evolution of dioecy described by \citet{Charlesworth1978a}, but focus instead on the initial invasion of unisexuals rather than the subsequent evolution of dioecy. 

Here, we develop a series of population genetic models for the evolution of gyno- and androdioecy from simulatenous hermaphroditism via the invasion of sex-specific sterility mutations. We explore the theoretical consequences of linkage between one locus, experiencing SA selection, for the invasion of a sex-specific sterility mutation at a second locus. We show that linkage results in significantly more permissive parameter conditions for the invasion of unisexuals, as well as elevated equilibrium frequencies of unisexuals relative to the classic single-locus predictions. Overall, our results indicate that existing SA polymorphisms facilitate the evolution of dimorphic sexual systems via the invasion of linked sex-specific sterility mutations, suggesting a previously unrecognized role for SA genetic variation during the process of early sex-chromosome evolution.


\section{Models} \label{sec:methods}

Following previous population genetic models for the evolution of dimorphic sexual systems \citep{Charlesworth1978a}, we model the evolution of gyno- and androdioecy from an ancestral population of simultaneous hermaphrodites via the invasion of alleles causing complete male- or female-sterility. We restrict our analyses to four scenarios of biological interest: the evolution of gynodioecy via invasion of a (1) completely dominant, or (2) completely recessive male-sterility allele; and the evolution of androdioecy via the invasion of a (3) completely dominant or (4) completely recessive female-sterility allele. These scenarios represent 'corner cases' for both the phenotypic effect and the dominance of the sex-specific sterility mutations. Alternative models allowing for partial sterility or dominance will likely yield evolutionary dynamics that are intermediate to, and can inferred from, our results. Here we briefly describe the simplest case of a dominant male-sterility mutations (scenario 1), and highlight only essential differences for each of the other three models (see Appendix A in the Online Supporting Information for a full development of the models). 

\subsection{Gynodioecy}
Consider a genetic system involving two diallelic autosomal loci, $\mathbf{A}$ (with alleles $A$, $a$) and $\mathbf{M}$ (with alleles $M_1$, $M_2$), that recombine at rate $r$ in a large population of simultaneous hermaphrodites. Let us assume that the $\mathbf{A}$ locus is under sexually antagonistic selection, with the $A$ allele having female-beneficial (male-deleterious) fitness effects, and the $a$ allele having male-beneficial (female-deleterious) fitness effects. Let us also assume that the $M_1$ allele at the $\mathbf{M}$ locus is neutral, while the $M_2$ allele causes sterility through the male sex function (e.g., via the production of no, or inviable, pollen), and is completely dominant to the $M_1$ allele. In this genetic system, the successful invasion of the $M_2$ allele, either alone or coupled with an allele at the $\mathbf{A}$ locus as a haplotype, represents the evolution of gynodioecy from simultaneous hermaphroditism. 

The population rate of self-fertilization, $C$, is assumed to be independent of the genotype at the $\mathbf{A}$ locus. By definition, the genotype at the $\mathbf{M}$ locus determines whether or not self-fertilization is possible (after \citealt{Charlesworth1978a}). Let us also assume a constant population level of inbreeding depression, $\delta$, defined as the decrease in the probability of survival of zygotes formed by self-fertilization relative to those produced by outcrossing. As noted by \citet{Charlesworth1978a}, this assumption only holds for populations at equilibrium, but probably represents a conservative estimate of the effects of inbreeding depression. Generations are assumed to be non-overlapping, and the life-cycle proceeds as follows: fertilization $\rightarrow$ inbreeding depression $\rightarrow$ selection $\rightarrow$ fertilization.

Let $x_i$ and $y_i$ denote the frequencies of the four possible haplotypes $[AM_1,AM_2,aM_1,aM_2]$ in female and male gametes respectively. Under some circumstances, female unisexuals carring the $M_2$ allele may be able to re-allocate resources towards ovule production that would otherwise have been used for pollen production (\citealt{Lloyd1975,Lloyd1976,Charlesworth1978a}). To account for such reproductive compensation, let $k$ describe the proportional increase in ovule production in females relative to hermaphrodites. The fitness of offspring resulting from the union of the $i^{th}$ female and $j^{th}$ male gametic haplotype, are denoted $w^f_{ij}$ and $w^m_{ij}$ respectively, and are assumed to be the product of the fitness expressions for $\mathbf{A}$ and $\mathbf{M}$ (Table~\ref{tab:fitness}).

\begin{table*}[htbp]
\centering
\caption{\bf Fitness expressions for diploid adults prior to reproduction for the model of a dominant male-sterility mutation ($w^f_{ij}$ denotes fitness effects through the female sex-function , $w^m_{ij}$ for male sex-function).}
\begin{tableminipage}{\textwidth}
\begin{tabularx}{\textwidth}{XXXXX}
\hline
Haplotype & $y_1 = AM_1$ & $y_2 = AM_2$ & $y_3 = aM_1$ & $y_4 = aM_2$ \\
\hline
Female sex-function & & & & \\
$x_1 = AM_1$ & $1$ & $(1 + k)$ & $(1 - h_f s_f)$        & $(1 - h_f s_f)(1 + k)$ \\
$x_2 = AM_2$ & $-$ & $(1 + k)$ & $(1 - h_f s_f)(1 + k)$ & $(1 - h_f s_f)(1 + k)$ \\
$x_3 = aM_1$ & $-$ & $-$       & $(1 - s_f)$            & $(1 - s_f)(1 + k)$ \\
$x_4 = aM_2$ & $-$ & $-$       & $-$                    & $(1 - s_f)(1 + k)$ \\
Male sex-function & & & & \\
$x_1 = AM_1$ & $(1 - s_m)$ & $0$ & $(1 - h_m s_m)$ & $0$ \\
$x_2 = AM_2$ & $-$         & $0$ & $0$             & $0$ \\
$x_3 = aM_1$ & $-$         & $-$ & $1$             & $0$ \\
$x_4 = aM_2$ & $-$         & $-$ & $-$             & $0$ \\
\hline
\end{tabularx}
  \label{tab:fitness}
\end{tableminipage}
{\footnotesize Note: Rows and columns indicate the haplotype inherited from mothers and fathers respectively. The lower triangle of each matrix is the reflection of the upper triangle, and is omitted for simplicity and consistency with the $ij$ row/column indexing used throughout the article.}
\end{table*}


Non-independence between an individual's genotype at the $\mathbf{M}$ locus and the ability to self-fertilize complicates the derivation of recursion equations describing evolutionary change in genotype frequencies. We therefore model the change in frequency of each genotype given the mode of transmission (either self-fertilization or outcrossing). This approach yields a system of $20$ recursion equations ($10$ genotypes $\times$ two modes of transmission). However, when $M_2$ is completely dominant, the system of recursions reduces considerably. Let $F_{ij}$ equal the frequency among zygotes of the genotype formed by the union of the \textit{i}th female and \textit{j}th male gametic haplotype via outcrossing, and $G_{ij}$ be the same for zygotes formed via self-fertilization. The genotypic frequencies in the next generation among zygotes formed by outcrossing are then:
\begin{linenomath}\begin{align*} \label{eq:FprGyn}
    F'_{11} &= (1 - S) (x_1 y_1)  \\
    F'_{12} &= (1 - S) (x_2 y_1)  \\
    F'_{13} &= (1 - S) (x_1 y_3 + x_3 y_1)  \\
    F'_{14} &= (1 - S) (x_4 y_1)  \\
    F'_{22} &= 0 \\
    F'_{23} &= (1 - S) (x_2 y_3)  \\
    F'_{24} &= 0 \\
    F'_{33} &= (1 - S) (x_3 y_3)  \\
    F'_{34} &= (1 - S) (x_4 y_3)  \\
    F'_{44} &= 0, \numberthis
\end{align*}\end{linenomath}

\noindent where $x_{i}$ and $y_{i}$ are functions $x_i=f(F_{ij},G_{ij},w^f_{ij},C,\delta,r)$ and $y_i=f(F_{ij},G_{ij},w^m_{ij},C,\delta)$ describing the haplotype frequencies among ovules and pollen, and $S$ is the proportion of all ovules produced by the population that are self-fertilized. Note that $y_2=y_4=0$ because these male gametic haplotypes cannot be produced when $M_2$ is dominant. The genotypic frequencies in the next generation among zygotes formed by self-fertilization are 
\begin{linenomath}\begin{align*} \label{eq:GprGyn}
    G'_{11} &= S (o^S_{11} + o^S_{13}/4) \\
    G'_{13} &= S (o^S_{13}/2) \\
    G'_{33} &= S (o^S_{33} + o^S_{13}/4), \numberthis
\end{align*} \end{linenomath}

\noindent where $o^S_{ij}$ are functions $f(F_{ij},G_{ij},w^f_{ij},C,\delta)$ describing the proportional contribution of each genotype to self-fertilized ovules. All $G'_{ij} = 0$ where $ij \neq [11,13,33]$ (see Appendix A in the Online Supporting Information for a full derivation of all recursions). 

The basic form of the recursions does not change when $M_2$ is completely recessive, but there two notable differences. Because only $M_2M_2$ homozygotes are phenotypically female, none of the recursions reduce to zero, and the recombination rate enters into the nonzero $G'_{ij}$ recursions. For the same reason, all $x_i$ and $y_i$ describing the haplotype frequencies among ovules and pollen become nonzero functions $f(F_{ij},G_{ij},w^m_{ij},C,\delta,r)$ including the recombination rate.


\subsection{Androdioecy}

The successful invasion of a dominant $M_2$ allele causing complete female-sterility (e.g., production of no, or inviable, ovules) represents the evolution of androdioecy. In contrast to the models of gynodioecy described above, the reproductive compensation term, $k$, now describes the proportional increase in pollen production by males relative to hermaphrodites. The fitness expressions, $w^f_{ij}$ and $w^m_{ij}$ resemble those described in Table~\ref{tab:fitness}, except the fitness effects of the $\mathbf{M}$ locus apply to the female rather than male sex-function. 

The genotype $\times$ transmission mode recursions for the models of androdioecy are very similar to those described for the models of gynodioecy, with a few key differences. When $M_2$ is dominant, $x_{i}$ and $y_{i}$ are again functions describing the haplotype frequencies among ovules and pollen, but now the recombination rate drops out of the expressions for $x_i=f(F_{ij},G_{ij},w^f_{ij},C,\delta)$, comes into the expressions for $y_i=f(F_{ij},G_{ij},w^m_{ij},C,\delta,r)$, and $x_2=x_4=0$. The genotypic frequencies in the next generation among zygotes formed by outcross fertilization are:
\begin{linenomath}\begin{align*} \label{eq:FprAnd}
    F'_{11} &= (1 - S) (x_1 y_1) \\
    F'_{12} &= (1 - S) (x_1 y_2) \\
    F'_{13} &= (1 - S) (x_1 y_3 + x_3 y_1) \\
    F'_{14} &= (1 - S) (x_1 y_4) \\
    F'_{22} &= 0 \\
    F'_{23} &= (1 - S) (x_3 y_2) \\
    F'_{24} &= 0 \\
    F'_{33} &= (1 - S) (x_3 y_3) \\
    F'_{34} &= (1 - S) (x_3 y_4) \\
    F'_{44} &= 0, \numberthis
\end{align*} \end{linenomath}

\noindent where $S$ is again equal to the proportion of all ovules produced by the population that are self-fertilized, but now accounts for the fact that heterozygotes at the $\mathbf{M}$ locus do not produce ovules. The form of the $G'_{ij}$ recursions remain unchanged from Eq(\ref{eq:GprGyn}).

When the $M_2$ female-sterility allele is completely recessive, only $M_2M_2$ homozygotes are phenotypically male. The form of the recursions is very similar to the case of gynodioecy with a recessive sterility allele. The fitness effects arising from the $\mathbf{M}$ locus now only affect the female sex-function, and all $x_i$ and $y_i$ describing the haplotype frequencies among outcrossed ovules and pollen become nonzero functions $f(F_{ij},G_{ij},w^f_{ij},C,\delta,r)$ including the recombination rate.


\subsection{Analyses} \label{subsec:analyses}

Our analyses address three main theoretical questions: (1) Under what parameter conditions can unisexuals invade a population of hermaphrodites that is initially monomorphic at $\mathbf{A}$, and when will stable polymorphism be maintained at the $\mathbf{A}$ locus? (2) Does pre-existing SA polymorphism at the $\mathbf{A}$ locus facilitate the invasion of unisexuals? (3) How do the equilibrium frequencies of SA alleles ($A,a$), sterility alleles ($M_2$), and females differ from single-locus predictions for the evolution of gynodioecy? As in previous models of sexually antagonistic selection (e.g., \citealp{Kidwell1977,Prout2000,JordanConnallon2014}), we limit our analyses to the representative, and biologically interesting cases of additive fitness effects ($h_m = h_f = 1/2$), and dominance reversal ($h_m, h_f < 1/2$) at the $\mathbf{A}$ locus. These scenarios are of particular interests because additive fitness effects are commonly observed for alleles with small to intermediate fitness effects \citep{Agrawal2011}, and dominance reversals are often predicted by fitness landscape models of dominance \citep{Manna2011, ConnallonClark2014}.

To identify the parameter conditions where balancing selection is predicted to maintain stable polymorphism at the $\mathbf{A}$ locus and unisexuals are able to invade, we evaluate the stability of the system of recursions for a populations initially fixed for the female-beneficial or male-beneficial allele at the $\mathbf{A}$ locus, and the $M_1$ allele at the $\mathbf{M}$ locus. These conditions correspond to initial equilibrium genotypic frequencies among outcrossed and selfed zygotes of $F_{11} = (1 - C) [AAM_1M_1] = 1$, $F_{33} = (1 - C) [aaM_1M_1] = 1$, and $G_{11} = C [AAM_1M_1] = 1$, and $G_{33} = C [aaM_1M_1] = 1$. Under these initial conditions, the fate of new mutations is determined by the rate of change of the frequencies of rare genotypes, which can be approximated by one minus the leading eigenvalue of the Jacobian matrix of the system of recursions, $1 - \lambda_L$ \citep{OttoDay2007}. Balancing selection is predicted to maintain polymorphism when $\lambda_L > 1$ for both boundary equilibria \citep{Prout1968,OttoDay2007}. For the models involving dominant male- and female-sterility alleles, this analysis yields three analytically tractable candidate leading eigenvalues describing the invasion of a new mutation at each locus independently ($\lambda_\mathbf{A}$ and $\lambda_\mathbf{M}$), and the joint invasion of a double-mutant haplotype ($\lambda_\mathbf{AM}$). When sterility alleles are recessive, analysis of the eigenvalues yield inconclusive results for invasion of unisexuals because heterozygotes at the $\mathbf{M}$ locus are entirely sheltered from selection. We therefore present analytic results based on the eigenvalue analyses for the models of dominant sterility alleles only.

To evaluate whether pre-existing SA polymorphism influences the invasion of unisexuals into populations of hermaphrodites, we evaluated whether the $M_2$ allele could invade populations initially at single-locus equilibrium for $\mathbf{A}$ and fixed for $M_1$ at $\mathbf{M}$. Specifically, we evaluated whether $\lambda_\mathbf{M} \rvert_{F^{\ast}_{ij},G^{\ast}_{ij}} > 1$ and $\lambda_\mathbf{AM} \rvert_{F^{\ast}_{ij},G^{\ast}_{ij}} > 1$, where $F^{\ast}_{ij}$ correspond to the single-locus equilibrium genotypic frequencies
\begin{linenomath}\begin{align*} \label{eq:SA1locEQ}
	F^{\ast}_{11} &= (1 - C)[AAM_1M_1],& G^{\ast}_{11} &= C[AAM_1M_1]  \\
	F^{\ast}_{13} &= (1 - C)[AaM_1M_1],& G^{\ast}_{13} &= C[AaM_1M_1] \\
	F^{\ast}_{33} &= (1 - C)[aaM_1M_1],& G^{\ast}_{33} &= C[aaM_1M_1] \numberthis \\
\end{align*}\end{linenomath}

\noindent and all other $F^{\ast}_{ij} = G^{\ast}_{ij} =0$. Under obligate outcrossing, analytic solutions are possible for the the single-locus allele frequencies at $\mathbf{A}$ among males and females ($\hat{p}_f,\hat{p}_m$; \citealt{Kidwell1977}). This is not possible for partial selfing, or in the case of dominance reversal. However, under these conditions, the single-locus equilibrium frequencies can be estimated using a weak selection approximation for $\hat{p}$ ($s_m,s_f \ll 1$; after \citealt{JordanConnallon2014}), which we then use to calculate $F^{\ast}_{ij}$ and $G^{\ast}_{ij}$. In both cases, the approximations perform reasonably well when selection is not especially strong ($s_f,s_m \leq 0.5$; \citealt{JordanConnallon2014,Olito2017}). 

To determine the equilibrium frequencies of SA alleles, $M_2$, and unisexuals invading polymorphic populations, we performed deterministic simulations of the recursions $F'_{ij},~G'_{ij}$ using the initial genotypic frequencies described by Eq(\ref{eq:SA1locEQ}) under a variety of parameter conditions (see \hl{Appendix X} in the Online Supporting Information). We focus our analysis on the comparison of equilbrium frequencies of unisexuals from our models with the corresponding exact single-locus equilibrium frequencies, $\hat{Z}$, given by \citet{Charlesworth1978a} for cases of relatively tight linkage between the $\mathbf{A}$ and $\mathbf{M}$ ($r\leq 0.1$). We emphasize tight linkage for two reasons. First, this represents the parameter conditions where our model predictions should differ most from those of single-locus models; with higher recombination, the predictions from the one- and two-locus models converge. Second, tight or complete linkage should approximate the biologically plausible scenario where SA loci, including a sex-specific sterility locus, are involved in a chromosomal inversion \citep{LeeKelly2015}.

In single-locus models of gynodioecy and androdioecy, the equilibrium frequency of unisexuals is determined by the reproductive compensation term, $k$, and the compound parameter $C \delta$ \citep{Charlesworth1978a}. However, if inbreeding depression is caused primarily by recessive deleterious mutations, as current data suggest \citep{Charlesworth2009}, there should be strong negative covariance between $C$ and $\delta$ as increased selfing more effectively purges deleterious recessives. In the simplest case, the mutation load due to deleterious recessive mutations at a single locus in a completely selfing population should be roughly half that of a randomly mating outcrossing population ($\mu$ versus $2 \mu$, where $\mu$ is the genome-wide mutation rate; \citealt{OhtaCockerham1974}). To account for negative covariance between $C$ and $\delta$, and thereby explore more biologically meaningful parameter space, we constrain inbreeding depression for our simulations to be a function of the selfing rate: $\delta = \delta^\ast(1 - C/2)$, where $\delta^\ast$ represents the hypothetical severity of inbreeding depression if selfing were enforced on a completely outcrossing population ($\delta^\ast \in [0,1]$). More complex expressions for $\delta$ yielded qualitatively similar results (See \hl{Appendix X} in the Online Supporting Information). 

Finally, it was important to account for the fact that $C$ and $\delta$ also influence the maintenance of SA polymorphism and the equilibrium frequencies of the SA alleles at $\mathbf{A}$ \citep{JordanConnallon2014,Olito2017}. We therefore ran simulations using values of $s_f$ and $s_m$ corresponding to single-locus equilibrium frequencies of $1/2$ for the two SA alleles ($p = [A]$, $q = [a]$; $p=q=1/2$). Thus, our simulations explore the invasion of the male- or female-sterility allele, $M_2$, into hermaphroditic populations initially at highly polymorphic single-locus equilibrium at $\mathbf{A}$.


\subsection{Data availability}
A full development of all models can be found in \hl{Appendix A}, and all code necessary to reproduce the analyses are available in the Online Supporting Information, and at \url{https://github.com/colin-olito/dioecySA}.


\section{Results}

\subsection{Invasion into monomorphic populations}

\subsubsection{Gynodioecy:} We begin with the evolution of gynodioecy by the invasion of a dominant male-sterility allele, $M_2$, into populations initially fixed for the $AAM_1M_1$ and $aaM_1M_1$ genotypes. Solving for the conditions where $\lambda_{\mathbf{A}} > 1$, we were able to recover the well known single-locus invasion criteria for SA alleles under obligate outcrossing \citep{Kidwell1977} and partial selfing \citep{JordanConnallon2014,Olito2017}, with either additive fitness effects, or dominance reversal. Solving $\lambda_{\mathbf{M}} > 1$ for $k$ also yields the classic single-locus critera for the invasion of females into a population of hermaphrodites (Eq(4) in \citealt{Charlesworth1978a}):

\begin{equation}\label{eq:1LocGyn}
	\hat{k} > 1 - 2 C \delta.
\end{equation}

\noindent The invasion conditions for a mutant haplotype bearing $M_2$ depend on whether the population is initially fixed for the female-beneficial ($A$) or male-beneficial ($a$) allele at $\mathbf{A}$. In both cases, solving $\lambda_{\mathbf{AM}} > 1$ for $k$ yield expressions $f(C,\delta,s_f,r)$ of the same basic form as Eq(\ref{eq:1LocGyn}), in which $k$ is a decreasing function of $C \delta$. For populations initially fixed for $A$, the haplotype invasion conditions are always more restrictive (requiring larger $k$) than those described by Eq(\ref{eq:1LocGyn}), except in the limit of complete linkage and no selection against $a$ ($r \rightarrow 0$ and $s_f \rightarrow  0$). In contrast, for populations initially fixed for the male-beneficial allele, $a$, the invasion conditions become more permissive with either lower recombination, or stronger selection against $a$. Specifically, the conditions for the spread of mutant $AM_2$ haplotypes become more permissive than Eq(\ref{eq:1LocGyn}) when 

\begin{equation}\label{eq:2LocGyn}
	\frac{2 r}{1 + r} < s_f.
\end{equation}

\noindent Eq(\ref{eq:2LocGyn}) shows that the invasion of females into a population of hermaphrodites can be quite sensitive to linkage with a nearby SA locus. When selection is relatively weak ($s_f = 0.1$), even modest linkage ($r\approx0.05$) can reduce the amount of reproductive compensation necessary for females to invade relative to the single-locus prediction. 

Linkage with a male-sterility locus also influences the invasion of female-beneficial SA alleles. For example, under obligate outcrossing and additive fitness effects, the single-locus invasion condition for the female-beneficial allele, $A$, is $s_f < s_m / (1+s_m)$ (\citealt{Kidwell1977}; also from $\lambda_{\mathbf{A}}$). Substituting for $s_f$ in Eq(\ref{eq:2LocGyn}) and solving for $r$ yields

\begin{equation}\label{eq:2LocGynSA}
	r < \frac{s_m}{2 + s_m}.
\end{equation}

\noindent By the same method, the criteria for the invasion of a female-beneficial allele under partial selfing and additive fitness effects will become more permissive than the single-locus criteria when 

\begin{equation}\label{eq:2LocGynSApartSelf}
	r < \frac{s_m (1 - C)}{2 + s_m +C (2 - s_m - 4 \delta)}.
\end{equation}

\noindent Inspection of Eq(\ref{eq:2LocGynSApartSelf}) shows that the scope for linkage between the SA locus and the sterility locus to expand the parameter space where the female-beneficial allele can invade is reduced with higher selfing (larger $C$), but that this can be compensated somewhat by inbreeding depression (larger $\delta$). This outcome arises directly from the increasing "female bias" in selection caused by selfing. With higher selfing, there is simply less parameter space where linkage with a male-sterility locus can facilitate the invasion of the female-beneficial allele \citep{JordanConnallon2014,Olito2017}.

\subsubsection{Androdioecy:} As for the previous model, we performed a stability analysis for the model for the evolution of androdioecy via the invasion of a dominant female-sterility allele, $M_2$, into populations initially fixed for the $AAM_1M_1$ and $aaM_1M_1$ genotypes. As before, analysis of $\lambda_{\mathbf{A}}$ recovered the single-locus invasion criteria for SA alleles under obligate outcrossing and partial selfing under both additive fitness effects and dominance reversal conditions. Likewise, solving $\lambda_{\mathbf{M}} > 1$ for $k$ yielded the familiar invasion criteria for males into a population of hermaphrodites (Eq(8) in \citealt{Charlesworth1978a}):

\begin{equation}\label{eq:1LocAndro}
	\hat{k} > \frac{1 + C (1 - 2 \delta)}{(1 - C)}.
\end{equation}

\noindent Not surprisingly, the invasion conditions for mutant haplotypes bearing $M_2$ again depended on the initial fixed genotype of the hermaphrodite population, and solving $\lambda_{\mathbf{AM}} > 1$ for $k$ yields expressions $f(C,\delta,s_m,r)$ of the same form as Eq(\ref{eq:1LocAndro}). Mirroring the results of previous model, the conditions satisfying $\lambda_{\mathbf{AM}} > 1$ for the invasion of a mutant haplotype were always more restrictive than Eq(\ref{eq:1LocAndro}) when the population was initially fixed for the male-beneficial allele ($a$; except when $s_m \rightarrow 0$ and $r \rightarrow 0$), and more permissive for populations initially fixed for the female-beneficial allele ($A$). In fact, replacing $s_f$ with $s_m$ in Eq(\ref{eq:2LocGyn}) gives the conditions under which invasion of males via the spread of mutant $aM_1$ haplotypes become more permissive than the single locus criteria described by Eq(\ref{eq:1LocAndro}). Thus, although the conditions necessary for the evolution of androdioecy are more restrictive than for gynodioecy overall, the scope for linkage with another SA locus to facilitate the invasion of female-sterility alleles is the same as for male-sterility alleles. 

The effect of linkage with a female-sterility locus on the invasion of male-beneficial SA alleles is nearly identical for the case of obligate outcrossing due to the symmetry of the single-locus SA invasion conditions. In this case, substituting $s_m$ in for $s_f$ in Eq(\ref{eq:2LocGynSA}) gives the conditions where invasion male-beneficial alleles becomes more permissive than the single-locus predictions. However, the case is altered subtantially for partially selfing populations. For example, assuming additive fitness effects, invasion of male-beneficial alleles at an SA locus linked to a female-sterility locus becomes more permissive than Eq(\ref{eq:1LocAndro}) when

\begin{equation}\label{eq:2LocAndroSApartSelf}
	r < \frac{s_f + s_f C (1 - 2 \delta)}{2 + s_f - 2 C + s_f C (1 - 2 \delta)}.
\end{equation}

\noindent Eq(\ref{eq:2LocAndroSApartSelf}) increases with both $C$ and $s_f$ such that, for $C > 1/3$, there is always some expansion of the parameter space where male-beneficial alleles can invade beyond the single-locus expectation, even under free-recombination with the female-sterility locus ($r = 1/2$). 


\begin{figure}[htbp]
\centering
\includegraphics[width=\linewidth]{Fig1}
\caption{Invasion of unisexuals into populations with pre-existing SA polymorphism. Plots show the fraction of parameter space (defined by $s_f \times s_m | F^{\ast}_{ij},G^{\ast}_{ij} > 0$ and $0 < s_f,s_m \leq 0.5$) where a dominant sex-specific sterility allele at $\mathbf{M}$, can invade populations initially at single-locus equilibrium frequencies for $\mathbf{A}$ with additive fitness effects ($h_f=h_m=1/2$), plotted as a function of the recombination rate $r$. Panels A--C show results from the model of gynodioecy via invasion of a male-sterility allele, while planels D--F show results for the model of androdioecy via invasion of a female-sterility allele. For each panel, results are shown for different values of reproductive compensation, $k$, chosen as a fraction of the single-locus invasion critera for $M_2$ defined by Eq(\ref{eq:1LocGyn}) and Eq(\ref{eq:1LocAndro}) for the models of gynodioecy and androdioecy respectively. Hence, the orange, green, and dark blue lines show scenarios where unisexuals experience a decrease in gamete production relative to hermaphrodites of $1$, $5$, and $10\%$. Note the different scale for the x-axis in panels C and F. Results were obtained by evaluating the three candidate leading eigenvalues ($\lambda_{\mathbf{A}}$,$\lambda_{\mathbf{M}}$,$\lambda_{\mathbf{AM}}$) of the Jacobian matrix of the genotype $\times$ transmission mode recursions for populations at the above initial conditions for $1000$ points uniformly distributed throughout the relevant $s_f \times s_m$ paramter space.}
\label{fig:PrInv}
\end{figure}

\subsection{Invasion of unisexuals into polymorphic populations}

The fate of new sterility mutations in populations that are initially polymorphic at $\mathbf{A}$ is influenced by the degree of reproductive compensation ($k$), the rate of recombination ($r$), and the selfing rate ($C$). Unisexuals are always able to invade polymorphic populations if either Eq(\ref{eq:1LocGyn}) (for gynodioecy) or Eq(\ref{eq:1LocAndro}) (for androdioecy) are satisfied (fig.~\ref{fig:PrInv}; light blue lines). Unisexuals can still invade when $k < \hat{k}$, provided there is some linkage between $\mathbf{A}$ and $\mathbf{M}$. For example, under obligate outcrossing and tight linkage, unisexuals can invade a polymorphic population \hl{across} $\approx 69\%$ of relevant parameter space (defined by $s_f \times s_m$ where $0 < s_f,s_m \leq 0.5$, and $F^{\ast}_{ij} > 0$), despite a $10\%$ reduction in gamete production relative to hermaphrodites (fig.~\ref{fig:PrInv}A,D). With smaller reductions in relative gamete production, unisexuals can invade across a greater fraction of parameter space, even when linkage is quite weak (\hl{e.g.,} $\approx 38\%$ when $k = \hat{k} \times 0.95 $ and $r = 0.2$). 

Unisexuals can invade under similar conditions if the population selfing rate is relatively low and inbreeding depression is relatively high ($C = 1/4,~\delta = 4/5$) (fig.~\ref{fig:PrInv}B,E). This makes sense because under these conditions the majority of offspring are produced through outcrossing rather than selfing. For populations with relatively high selfing rates and low inbreeding depression ($C = 3/4,~\delta = 1/5$), major differences between the models of gyno- and androdioecy emerge. Most notably, when $k < \hat{k}$ significanlty tighter linkage is required for the male-sterility mutation to spread. Even then, invasion will only occur if the reduction in ovule production by females relative to hermaphrodites is quite small (fig.~\ref{fig:PrInv}C,F). Conversely, in the model of androdioecy, the invasion of female-sterility mutations does not become sensitive to $r$, and males can invade over similar fractions of parameter space as in predominantly outcrossing populations. This contrast between the models in selfing populations is a consequence of the increasing female-bias in the net direction of SA selection caused by increased selfing \citep{Charlesworth1978a,JordanConnallon2014,Olito2017}. Analogous to previous results for linked SA loci \citep{Olito2017}, as the female-bias in SA selection increases with selfing, there is greater scope for linkage to a male-beneficial SA allele to facilitate invasion of a female-sterility mutation.

For both the model of gyno- and androdioecy, the loss of parameter space where unisexuals are able to invade is not random. With weaker linkage between $\mathbf{A}$ and $\mathbf{M}$, invasion of unisexuals requires stronger SA selection coefficients that must also be increasingly biased toward the gender of the invading unisexuals. Hence, the invasion of male-sterility alleles requires strong female-baised selection, while the invasion of female-sterility requires male-biased selection (see Figures \hl{S1--S6} in the Online Supporting Information). \hl{The conditions for unisexual invasion into polymorphic populations are not qualitatively different under dominance reversal conditions}. However, as in previous SA models, a dominance reversal expands the region of parameter space where polymorphism is maintained at $\mathbf{A}$, and simultaneously expands the parameter space where unisexual are able to invade in our models (see figs.~S1--S6 in the Online Supporting Information).


\subsection{Equilibrium frequency of unisexuals}

When sterility mutations occur in initially polymorphic populations, linkage with another SA locus generally results in elevated equilibrium frequencies of unisexuals relative to single-locus predictions. If reproductive compensation by unisexuals satisfies the conditions for single-locus invasion of $M_2$ (i.e., $k \geq \hat{k}$), the increase in the equilibrium frequency of unisexuals relative to single-locus predictions can be quite large (fig.~\ref{fig:eqFreq}, light blue lines). Linkage also allows unisexuals to reach non-trivial equilibrium frequencies when reproductive compensation falls below than the single-locus threshold for invasion ($k < \hat{k}$; fig.~\ref{fig:eqFreq}). When sterility mutations are dominant, the relative increase in unisexual frequencies is largest for primarily outcrossing populations, and diminishes as the population selfing rate increases (with concomitant decrease in inbreeding depression), and as linkage decreases. 

\hl{Incomplete} Explain dip below $\hat{Z}$ when $k > \hat{k}$ ... Results for model of recessive male-sterility mutation, ...


\begin{figure}[htbp]
\centering
\includegraphics[width=\linewidth]{Fig2}
\caption{Equilibrium frequencies of unisexuals compared with single-locus predictions. Results are shown for the models of the evolution of gyno- and androdioecy via invasion of dominant sex-specific sterility mutations, using a male selection coefficient of $s_m = 0.4$, and inbreeding depression that follows $\delta = \delta^\ast(1 - C/2)$ (see \nameref{sec:methods}). Plots show the deviation of the equilibrium frequency of unisexuals predicted by our models from the corresponding exact single-locus equilibrium frequencies ($\hat{Z}$ in \citealt{Charlesworth1978a}). Results are plotted for three different levels of reproductive compensation, calculated as a fraction of the single-locus invasion criteria for $M_2$ defined by Eq(\ref{eq:1LocGyn}) and Eq(\ref{eq:1LocAndro}). The single-locus equilibrium frequency of unisexuals is always $0$ when $k < \hat{k}$; thus the results for $k = \hat{k} \times 0.95,~0.9$ (dark blue and orange lines) also represent absolute frequencies of females.}
\label{fig:eqFreq}
\end{figure}


\section{Discussion}

Four main insights emerge from our results: (1) When linkage exists between an SA locus and the site of a sex-specific sterility mutation, the conditions permitting invasion of unisexuals become far more permissive than predicted by single-locus models; (2) segregating SA polymorphism facilitates the invasion of linked sterility mutations, especially when the net direction of SA selection is biased towards the gender of invading unisexuals; and \hl{(3)} linkage facilitates invasion, but also elevates the equilibrium frequency of unisexuals relative to single-locus predictions; and (4) the effect of linkage with an SA locus is almost always greatest for predominantly outcrossing populations. Overall, these predictions suggest that the sterility mutations driving the evolution of dimorphic sexual systems are likely to evolve in genomic regions harboring polymorphic SA loci. When this occurs, elevated frequencies of unisexuals in the resulting gyno- and androdioecious populations are more likely to evolve in predominantly outcrossing populations than previously predicted. Below, we discuss the implications of our findings, and suggest emprical tests of our predictions, in three main contexts: SA polymorphism and the evolution of gyno- and andro-dioecy, hermaphrodite mating systems and the evolution of dioecy, and the population genetic basis of the evolution of separate sexes.


\subsection{SA Polymorphism and the evolution of dimorphic sexual systems}

Although dioecy is relatively uncommon among angiosperms (represented in $\approx 7\%$ of genera), the fantastic diversity and repeated evolution of dimorphic sexual systems in flowering plants begs a satisfactory genetical explanation \citep{Renner2014,KaferPannell2017}. Several evolutionary pathways, and a variety of genetic mechanisms, lead from hermaphroditism to dioecy, but all ultimatley involve the evolutionary invasion of at least two sex-specific sterility mutations \citep{Charlesworth1978a,Charlesworth1978b,Renner2014,Ashman2015}. Sterility mutations are an important class of SA allele, but previous theory has not considered the potential influence of standing SA genetic variation, or genetic architecture, on their initial invasion and establishment in hermaphroditic populations. Our theoretical results suggest that the genetic architecture of SA genes can strongly influence the initial evolution of gyno- or androdioecy from hermaphroditism, expanding the conditions under which we should expect dimorphic sexual systems, and ultimately separate sexes, to evolve. 

One major explanation for the relative rarity of dioecy among angiosperms rests on the classic theoretical prediction that the conditions under which unisexuals can invade a population of hermaphrodites are quite stringent \citep{Lloyd1975,Lloyd1976,Charlesworth1978a,KaferPannell2017}. For sterility mutations to invade, the resulting unisexuals must adequately compensate for the loss of a sex function through increased gamete production. Our results suggest that the conditions for the spread of mutant haplotypes bearing an SA allele that is beneficial for the same sex-function as the invading unisexuals are very permissive -- requiring only modest linkage for biologically plausible selection coefficients (e.g., $s_f,s_m \leq 0.1$). When linkage does exist, the fitness effects of the complimentary SA allele help offset the loss of a sex function, and reduce the amount of reproductive compensation required by unisexuals compared to single-locus predictions \citep{Charlesworth1978a}. An important corollary of this result is that the conditions for the maintenance of SA polymorphism are also expanded by linkage with a sex-specific sterility mutation. Hence, new sterility mutations underpinning dimorphic sexual systems are most likely to evolve in tight linkage with other SA loci, and should simultaneously promote the maintenance of SA polymorphism at linked loci. 

Despite the generally favourable conditions for the evolution of dimorphic sexual systems in monomorphic populations, the waiting time for double mutants with the necessary haplotype (e.g., female-beneficial--male-sterile) to appear will likely be very long \hl{CITE}. However, we also find that the invasion conditions for sterility mutations are still quite permissive when there is standing SA genetic variation in hermaphroditic populations. Thus, the amount of standing SA genetic variation should directly influence the potential for hermaphroditic populations to evolve dimorphic sexual systems. Although it is not yet clear how much SA genetic variation for fitness is harbored by hermaphroditic species, three features of SA selection suggest that this is not an unlikely scenario, particularly in large populations. First, balancing selection is predicted to maintain SA polymorphism in partially selfing populations over a broad spectrum of parameter conditions, particularly when SA loci are linked \citep{Patten2010,JordanConnallon2014,Olito2017}. Second, net directional selection under SA is predicted to be small, even when fitness effects in each sex are large, resulting in relatively slow evolutionary change in SA allele frequencies, and strong sensitivity to effective population size \citep{ConnallonClark2012}. Third, although definitively indentifying SA loci is difficult, there is empirical evidence suggesting that segregating SA allele frequencies can be non-trivial, even in partially selfing hermaphroditic populations \citep{Barson2015,LeeKelly2015}. Additional studies attempting to quantify the degree of SA genetic variation in hermaphroditic species would help clarify the potential for dimorphic sexual systems to evolve from hermaphroditism, especially if they were to target species exhibiting intraspecific variation in the degree or frequency of dimorphic sexual systems (e.g., \hl{cite}).



\subsection{Mating systems and the evolution of dioecy}

effect of selfing rate on:
\begin{itemize}
	\item invasion of unisexuals
	\item opportunities for SA polymorphism
	\item equilibrium frequency of unisexuals
	\item testing hypotheses: phylogenetic comparitive methods, testing association of selfing rate with sexual system.
\end{itemize}

The interplay between hermaphrodite mating systems and reproductive compensation is a key factor influencing the evolution of separate sexes in flowering plants, especially via the gynodieocy pathway \citep{Darwin1877,Charlesworth1978a}. Prior theory predicts that the evolution of gynodioecy is driven by the combination of reproductive compensation by unisexuals, and avoidance of inbreeding depression, and is therefore most likely to occur in partially selfing populations \citep{Lewis1942,Lloyd1975,Charlesworth1978a,KaferPannell2017}. This follows directly from the structure of Eq(\ref{eq:1LocGyn}) where $\hat{k}$ is determined entirely by the product of the selfing rate and inbreeding depression ($C \delta$). In contrast, the evolution of androdioecy is predicted to require significantly higher reproductive compensation, especially in partially selfing populations, because invading males must still compete with selfing hermaphrodites to fertilize ovules \citep{Charlesworth1978b,KaferPannell2017}. 

The population selfing rate plays a similarly critical role in our models. We begin with the evolution of gynodioecy via invasion of a male-sterility mutation, in which the population selfing rate has two important consequences. First, as the selfing rate increases (and inbreeding depression decreases), tighter linkage is required to expand the parameter space where a male-sterility mutation is predicted to invade compared to single-locus predictions. Second, the increase in the equilibrium frequency of females relative to single-locus predictions decays with the selfing rate. The evolution of higher equilibrium frequencies of females is particularly important as this should facilitate the subsequent evolution of dioecy via invasion of partial sterility mutations \citep{Charlesworth1978a,Charlesworth1978b,Charlesworth1999,Charlesworth2006}. Two implications follow from these results. First, male-sterility mutations are most likely to evolve in tight linkage with another SA locus in highly selfing populations. Second, when male-sterility mutations do arise in linkage with other SA loci, single-locus theory may significantly underestimate the potential for gynodieocy, and subsequently dioecy, to evolve in predominantly outcrossing species. 

For the evolution of androdioecy, tighter linkage is not required to facilitate invasion of female-sterility mutations into highly selfing populations beyond single-locus predictions, but the increase in the equilibrium frequency of males is still greatest for outcrossing populations. Thus, when linkage exists between a female-steriliy mutation and another SA locus, single-locus theory may also underpredict the potential for androdioecy, and dioecy, to evolve. However, our models agree with previous theory that the conditions for invasion of female-sterility mutations in partially selfing populations are still quite stringent, requiring very high reproductive compensation (\citealt{Charlesworth1978a}, Eq(\ref{eq:1LocAndro})). Overall, linkage with an SA locus will always faciliate the invasion of female-sterility mutations, but the evolution of androdioecy, and dioecy via the androdioecy pathway, is still expected to be quite rare relative to gynodioecy \citep{Charlesworth1978a,Charlesworth2006,KaferPannell2017,Renner2014}.

Despite the longstanding theoretical prediction of a strong correlation between the hermaphrodite mating system and dioecy, empirical evidence for this association remains equivocal \citep{Charlesworth1985,Charlesworth2006,Renner2014}. However, this pattern is entirely consistent with our predictions if linkage with SA loci increases the likelihood that dimorphic sexual systems evolve in outcrossing populations. In light of our results, a re-examination of the evolutionary association between angiosperm mating and sexual systems using modern phylogenetic comparative methods would be interesting, and would help identify species where our proposed mechanism for the evolution of gynodioecy is most likely to have played a role (e.g., species with dimorphic sexual systems that appear to have evolved from predominantly outcrossing ancestors). Finally, the potential for selection at a linked SA locus to influence the evolutionary dynamics of sterility mutations underlying dimorphic sexual systems adds to the list of possible mechanisms explaining the apparently high frequency of reversions from dioecy back to hermaphroditism \citep{GoldbergOtto2017,KaferPannell2017}


\subsection{The population genetic basis of the evolution of separate sexes}

\begin{itemize}
	\item linkage, classic dioecy models, 
	\item higher eq.~freq.~of unisexuals should promote subsequent evolution of full dioecy
	\item evolution of sex chromosomes and the process of differentiation
\end{itemize}

The now well established view of early sex-chromosome evolution proceeds as follows. After an ordinary pair of autosomes have aquired a sex-determining locus, the accumulation of SA genes linked to the sex-determining locus drives the evolution suppressed recombination between the neo sex-chromosomes, generally via chromosomal inversions \citep{Rice1987,Charlesworth2002,Bachtrog2006,Qiuetal2013,Bachtrog2014}. The tacit assumption that the emergence of a sex-determining locus is the first step in this process traces back (again) to classic population genetic theory of sex-specific sterility mutations. In particular, the linkage constraint of \citet{Charlesworth1978a} presents the clear and intuitive prediction that, after the invasion of a recessive male-sterility gene, the subsequent invasion of dominant gender modifiers leading to full dioecy require tight linkage to the male-sterility locus. This process leads to a tightly coupled pair of sex-determining loci, effectively a single sex determining locus -- and further differentiation can result in heteromorphic sex chromosomes.

Our theoretical results suggest a previously unrecognized, or at least underappreciated, role for SA genetic variation during these early stages of sex-chromosome evolution. Our finding that linkage with a polymorphic SA locus facilitates the initial invasion of sex-specific sterility mutations strongly suggests that the accumulation of SA genetic variation is probably the first step in sex chromosome evolution. In addition, the evolution of elevated equilibrium frequencies of females in mostly outcrossing populations should also facilitate the evolution of dioecy via tightly linked gender modifiers \citep{Charlesworth1978a}. The resulting complex of linked SA and sterility loci effectively establishes a nascent sex-determining-region on the neo sex-chromosome pair. If pre-existing SA polymorhism at the linked SA locus survives the transition to dioecy, the process of accumulating SA variation prior to recombination suppression is already begun. However, even if this doesn't happen,  the stage is set for the subsequent accumulation of linked SA genetic variation and recombination suppression leading to sex-chromosomes differentiation \citep{Charlesworth1978a,Rice1987,Bachtrog2006,Qiuetal2013}. Given the crucial role of linkage in each of these steps, it seems plausible that the degree of SA genetic variation present when an initial male-sterility mutation arises may help explain the large variation in the rate and extent of sex-chromosome differentiation in angiosperms \citep{Charlesworth2002,Renner2014,Bachtrog2014}.


\bibliography{dioecySA-bibliography}

\end{document}