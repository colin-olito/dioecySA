\documentclass[9pt,twocolumn,twoside,lineno]{gsajnl}
% Use the documentclass option 'lineno' to view line numbers

% Other useful packages
\usepackage{color,soul}

% Equation numbering
\newcommand\numberthis{\addtocounter{equation}{1}\tag{\theequation}}

\articletype{inv} % article type
% {inv} Investigation 
% {gs} Genomic Selection	
% {goi} Genetics of Immunity 
% {gos} Genetics of Sex 
% {mp} Multiparental Populations

\title{Sexually antagonistic polymorphism and the evolution of dimorphic sexual systems from hermaphroditism}

\author[$\ast$,1]{Colin Olito}
\author[$\ast$]{Tim Connallon}
% \author[$\dagger$]{Author Two}
% \author[$\ddagger$]{Author Three}
% \author[$\S$]{Author Four}
% \author[$\ast\ast$]{Author Five}

\affil[$\ast$]{Department of Biological Sciences, Monash University, Melbourne, VIC 3800, Australia}
% \affil[$\dagger$]{Author two affiliation}
% \affil[$\ddagger$]{Author three affiliation}
% \affil[$\S$]{Author four affiliation}
% \affil[$\ast\ast$]{Author five affiliation}

\keywords{Androdioecy; Dioecy; Gynodioecy; Intralocus sexual conflict; Linkage disequilibrium; Recombination; Sexual system; Sexual dimorphism}

\runningtitle{GENETICS Journal Template on Overleaf} % For use in the footer 

\correspondingauthor{Colin Olito}

\begin{abstract}
Most flowering plants are hermaphroditic, with each individual expressing both male and female sex functions. Nevertheless, the diversity of plant sexual systems falls along a wide spectrum -– from simultaneous hermaphroditism, to physically separate sexes (dioecy), and nearly everything in between. Identifying the evolutionary mechanisms that facilitate transitions between different sexual systems remains a major unresolved question. One evolutionary pathway from hermaphroditism towards separate sexes is via the sequental invasion of "unisexual" sterility alleles that each eliminate female or male sex functions in hermaphrodites, and thereby yield discrete female and male individuals. Classic population genetics theory identifies two key preconditions that facilitate this pathway to dioecy: (1) genetically based allocation trade-offs between female and male sex-functions (a form of "sexual antagonism"); and (2) modest-to-high rates of self-fertilization and inbreeding depression in the ancestral hermaphrodite population. This theory also predicts that transitions to dioecy are most likely to begin by invasion of a male-sterility rather than a female-sterility allele. We extend recent population genetics theory of sexually antagonistic selection to show that physical linkage to a sexually antagonistic locus enhances the invasion of female- and male-sterility alleles, and significantly alters predictions about the role of self-fertilization in the evolution of dioecy. The new model shows that linkage to a sexually antagonistic locus has three consequences for the evolution of sexual systems. First, linkage broadens the conditions for invasion of unisexual sterility alleles, promoting transitions from hermaphroditism to gyno- and andro-dioecy. Second, linkage elevates the equilibrium frequencies of unisexual females or males in mixed-mating populations, thereby promoting subsequent transitions to full dioecy. Third, linkage greatly diminishes effects of ancestral selfing and inbreeding depression on transitions to gyno- and androdioecy. Overall, our results predict that sterility mutations that initiate the evolution of dioecy will evolve in tight linkage with sexually antagonistic loci. We discuss implications of these new results for the evolution of plant mating systems and the early dynamics of sex chromosome evolution.
\end{abstract}

\setboolean{displaycopyright}{true}

\begin{document}

\maketitle
\thispagestyle{firststyle}
\marginmark
\firstpagefootnote
\correspondingauthoraffiliation{Department of Biological Sciences, Monash University, Melbourne, VIC 3800, Australia. Email: colin.olito@gmail.com.}
\vspace{-11pt}%

\lettrine[lines=2]{\color{color2}A}{}lthough the majority of flowering plants are hermaphroditic, where individuals perform both male and female sex-functions, plant sexual systems form a diverse spectrum ranging from simultaneous hermaphroditism, to physically separate sexes (dioecy) \citep{Darwin1877,Westergaard1958,Bachtrog2014}. Moreover, this diversity of sexual dimorphism encompasses nearly every possible intermediate state, from hermaphrodites bearing various combinations of perfect and/or single-sex flowers (monoecy), to mixed populations of hermaphrodite and unisexual individuals (subdioecy) \citep{Bawa1980,SakaiWeller1999}.

Dioecious sexual systems have evolved from hermaphrodite ancestors repeatedly, and independently, in numerous plant lineages (\citealt{Westergaard1958,SakaiWeller1999,Charlesworth2006,Bachtrog2014,Renner2014,GoldbergOtto2017, KaferPannell2017}). Two main evolutionary pathways leading from hermaphroditism to dioecy involve the initial invasion of unisexual females or males into an ancestral hermaphrodite population (gyno- and androdioecy respectively) \citep{Charlesworth1978a,Charlesworth1978b}. Unisexuals might arise through invasion of nuclear sterility alleles causing complete loss of one sex-function, or more gradually through gender modifier, or partial sterility alleles \citep{Charlesworth1978a,Charlesworth1978b,Charlesworth1999}. In all cases, at least two mutational steps are required: invasion of male- or female-sterility alleles, resulting in gyno- or androdioecy (respectively), and the subsequent invasion of one or more mutations causing sterility for the sex function not rendered sterile by the first invading mutation \citep{Westergaard1958,Charlesworth1978a,Charlesworth1978b,Charlesworth2006,Charlesworth2009,KaferPannell2017}. Consistent with this two-step pathway to separate sexes, several lines of empirical evidence support the notion that two or more loci are often involved in nuclear plant sex determination systems \citep{Westergaard1958,Charlesworth2002,Charlesworth2006,Renner2014,Ashman2015}.

Several classic theoretical studies have addressed evolutionary transitions from hermaphroditism to dioecy. Some key predictions are that such transitions are favoured by allocation trade-offs between male and female sex-functions, and by the combination of modest-to-high rates of self-fertilization and inbreeding depression \citep{Lewis1941,Lloyd1975,Lloyd1976,Charlesworth1978a}. Theory also predicts that the invasion of a male sterility allele is more likely to initiate transitions to dioecy than the invasion of a female sterility allele (see \citealt{Charlesworth1999,Charlesworth2006} for comprehensive reviews of relevant theory). Additionally, the subsequent invasion of complete or partial sterility mutations resulting in dioecy may often require tight linkage with the first sterility mutation (the "linkage constraint" of \citealt{Charlesworth1978a}). The invasion of tightly linked complimentary sterility mutations then sets the stage for the evolution of hetermorphic sex chromosomes, such as a gene-rich X and degenerate Y \citep{Rice1987,Bachtrog2006,Charlesworth2002,Qiuetal2013}. 

The allocation trade-offs between sex-functions favoring transitions to dieocy represent a form of sexual antagonism (SA), and can lead to intra-locus sexual conflict when alleles have pleiotropic effects that are deleterious via one sex-function (i.e., cause decreased pollen production) and beneficial for the other (i.e., cause increased ovule production) \citep{JordanConnallon2014,Olito2016}. When they cause such reproductive compensation, unisexual sterility alleles can be viewed as a special case of SA alleles that cause partial to complete loss of one sex-function, and possibly increased allocation to the other (e.g., via reproductive compensation; \citealt{Lewis1941,Lloyd1975,Charlesworth1978a}). In the broader context of SA selection, we might therefore expect that unisexual sterility alleles should behave similarly to SA alleles. For example, their invasion requires adequate allocation to the remaining sex-function to offset the loss of fitness due to sterility and out-compete self-compatible hermaphrodites \citep{Charlesworth1978a}. Similarly, the invasion of SA alleles requires that the beneficial effects through one sex-function are not overpowered by negative selection through the other, and SA alleles can be maintained as polymorphisms when selection through each sex-function is evenly balanced or very strong \citep{Kidwell1977,JordanConnallon2014}. 

The invasion conditions for male and female unisexual sterility alleles have been determined in the single-locus context \citep{Charlesworth1978a}, yet recent work on SA alleles suggest that multilocus dynamics, including tight linkage between SA loci, can substantially alter conditions for invasion and polymorphism at SA loci. In both dioecious and partially selfing hermaphrodite species, linkage between two SA loci facilitates the invasion of SA alleles, and expands the parameter space where polymorphism is maintained by balancing selection beyond the predictions of single-locus models \citep{Kidwell1977,Patten2010,UbedaPatten2010,Olito2016}. Despite the inherent challenges in indentifying SA genes \citep{ConnallonClark2012,Barson2015}, there is empirical evidence that such multilocus dynamics may be important in natural populations of hermaphrodites (e.g., chromosomal inversions with SA fitness effects; \citealt{LeeKelly2015}).

Considering the evolution of sex-specific sterility alleles from the perspective of SA selection theory raises several important questions regarding the role of genomic architecture for the evolution of separate sexes. Does linkage with another SA locus facilitate the invasion of sterility alleles, for example, by promoting selection on linked allelic combinations (e.g., female-beneficial with male-sterility, and male-beneficial with female-sterility alleles)? Does the presence of segregating SA variation expand conditions for the invasion of unisexuals in hermaphrodite populations? Does linkage to SA alleles lead to elevated equilibrium frequencies of unisexuals in the resulting gyno- or androdoiecious populations? In some ways, these questions echo the premise of a 'linkage constraint' for the evolution of dioecy described by \citet{Charlesworth1978a}, but focus instead on the initial invasion of unisexuals rather than the subsequent evolution of dioecy. 

Here, we extend population genetic models for the evolution of gyno- and androdioecy from an ancestral population of simulatenous hermaphrodites, and consider the role of linkage between SA loci on transitions away from hermaphroditic sexual systems. We specifically explore the consequences of physical linkage between SA loci for the invasion and equilibrium frequencies of alleles that cause sex-specific sterility. We show that linkage results in significantly more permissive parameter conditions for the invasion of unisexuals, as well as elevated equilibrium frequencies of unisexuals relative to the predictions of classical, single-locus models. Overall, our results indicate that multi-locus consequences of SA selection facilitate the evolution of dimorphic sexual systems, and suggest a hitherto unrecognized role for SA genetic variation during the initial stages of early sex-chromosome evolution.

%%%%%%%%%%%%%%%%%%%%%%%%
\section{Models} \label{sec:methods}
%%%%%%%%%%%%%%%%%%%%%%%%

Following previous population genetic theory \citep{Charlesworth1978a}, we model the evolution of gyno- and androdioecy from an ancestral hermaphroditic population via the invasion of unisexual sterility alleles causing complete male or female sterility. We restrict our analyses to four scenarios of biological interest: the evolution of gynodioecy via invasion of a (1) completely dominant, or (2) completely recessive male-sterility allele; and the evolution of androdioecy via the invasion of a (3) completely dominant or (4) completely recessive female-sterility allele. These scenarios represent 'corner cases' for both the phenotypic effect and the dominance of the sex-specific sterility mutations. Alternative models allowing for partial sterility or dominance will presumably yield evolutionary dynamics that are intermediate between these corner cases. Here we describe in full the simplest model, involving a dominant male-sterility mutation (scenario 1). We subsequently highlight, in brief, the essential differences arising in each of the other three models. Further information on each model, including a development of the full recursions, can be found in Appendix A of the Online Supporting Information. 

%%%%%%%%%%%%%%%%%%%%%%%%
\subsection{Gynodioecy}

Consider a genetic system involving two diallelic autosomal loci, $\mathbf{A}$ (with alleles $A$, $a$) and $\mathbf{M}$ (with alleles $M_1$, $M_2$), that recombine at rate $r$ per meiosis, in a large population that is initially hermaphroditic. Assume that the $\mathbf{A}$ locus is under sexually antagonistic selection, with the $A$ allele having female-beneficial (male-deleterious) fitness effects, and the $a$ allele having male-beneficial (female-deleterious) fitness effects. Also assume that the $M_1$ allele at the $\mathbf{M}$ locus has a relative fitness of $1$ in both sexes, while the $M_2$ allele causes sterility through the male sex-function (e.g., via the production of no, or inviable, pollen), and is completely dominant relative to the $M_1$ allele. Female unisexuals carring the $M_2$ allele may be able to re-allocate resources towards ovule production that would otherwise have been used for pollen production (\citealt{Lloyd1975,Lloyd1976,Charlesworth1978a}). To account for such reproductive compensation, let $k$ describe the proportional increase in ovule production in females relative to hermaphrodites. In this genetic system, the successful invasion of the $M_2$ allele represents the evolution of gynodioecy from simultaneous hermaphroditism. 

The rate of self-fertilization among hermaphrodite individuals, $C$, is not directly influenced by the genotype at the $\mathbf{A}$ locus. By definition, the genotype at the $\mathbf{M}$ locus determines whether or not self-fertilization is possible (after \citealt{Charlesworth1978a}). Let us also assume a constant population level of inbreeding depression, $\delta$, defined as the decrease in the probability of survival of zygotes formed by self-fertilization relative to those produced by outcrossing. As noted by \citet{Charlesworth1978a}, this assumption only holds for populations at equilibrium, but probably represents a conservative estimate of the effects of inbreeding depression. Generations are assumed to be non-overlapping, and the life-cycle proceeds as follows: fertilization $\rightarrow$ differential survival due to inbreeding depression $\rightarrow$ selection $\rightarrow$ fertilization.

Let $x_i$ and $y_i$ denote the frequencies of the four possible haplotypes $[AM_1,AM_2,aM_1,aM_2]$ in female and male gametes respectively. The fitness of offspring resulting from the union of the $i^{th}$ and $j^{th}$ gametic haplotypes, are denoted $w^f_{ij}$ and $w^m_{ij}$ respectively, and are assumed to be the product of the fitness expressions for $\mathbf{A}$ and $\mathbf{M}$ (Table~\ref{tab:fitness}). We assume no parent-of-origin effects on fitness, and so index the ten possible pairs of gametic haplotypes where $i \geq j$ hereafter.

\begin{table*}[htbp]
\centering
\caption{\bf Fitness expressions for diploid adults prior to reproduction for the model of a dominant male-sterility mutation ($w^f_{ij}$ denotes fitness effects through the female sex-function , $w^m_{ij}$ for male sex-function).}
\begin{tableminipage}{\textwidth}
\begin{tabularx}{\textwidth}{XXXXX}
\hline
Haplotype & $ AM_1$ & $ AM_2$ & $ aM_1$ & $ aM_2$ \\
\hline
Female sex-function & & & & \\
$ AM_1$ & $1$ & $(1 + k)$ & $(1 - h_f s_f)$        & $(1 - h_f s_f)(1 + k)$ \\
$ AM_2$ & $-$ & $(1 + k)$ & $(1 - h_f s_f)(1 + k)$ & $(1 - h_f s_f)(1 + k)$ \\
$ aM_1$ & $-$ & $-$       & $(1 - s_f)$            & $(1 - s_f)(1 + k)$ \\
$ aM_2$ & $-$ & $-$       & $-$                    & $(1 - s_f)(1 + k)$ \\
Male sex-function & & & & \\
$ AM_1$ & $(1 - s_m)$ & $0$ & $(1 - h_m s_m)$ & $0$ \\
$ AM_2$ & $-$         & $0$ & $0$             & $0$ \\
$ aM_1$ & $-$         & $-$ & $1$             & $0$ \\
$ aM_2$ & $-$         & $-$ & $-$             & $0$ \\
\hline
\end{tabularx}
  \label{tab:fitness}
\end{tableminipage}
{\footnotesize Note: Rows and columns indicate the \textit{i}th and \textit{j}th gametic haplotype respectively. The lower triangle of each matrix is the reflection of the upper triangle, and is omitted for simplicity and consistency with the $i \geq j$ row/column indexing used throughout the article.}
\end{table*}


The fact that an individual's genotype at the $\mathbf{M}$ locus determines whether they are able to self-fertilize complicates the derivation of recursion equations describing evolutionary change in genotype frequencies. We therefore model the change in frequency of each genotype given the mode of transmission (either self-fertilization or outcrossing). This approach yields a system of $20$ general recursion equations ($10$ genotypes $\times$ two modes of transmission; see Appendix A in the Online Supplementary Supporting Information for a full derivation of all recursions). Although more compact approaches have been used for generating recursions in previous models for the evolution of self-fertilization and dioecy (e.g., \citealt{Charlesworth1978a,Charlesworth2010}), we used the following, expanded expressions as a means of clarifying the pathway and assumptions that underlie each of our models. As shown below, our models reproduce the classical results in special cases where there is only selection at loci segregating for sterility alleles.

When $M_2$ is completely dominant, the system of recursions reduces considerably. Let $F_{ij}$ equal the frequency among zygotes of the genotype formed by the union of the \textit{i}th and \textit{j}th haplotypes via outcrossing, and $G_{ij}$ be the same for zygotes formed via self-fertilization. The genotypic frequencies in the next generation among zygotes formed by outcrossing are then:
\begin{linenomath}\begin{align*} \label{eq:FprGyn}
    F'_{11} &= (1 - S) (x_1 y_1)  \\
    F'_{12} &= (1 - S) (x_2 y_1)  \\
    F'_{13} &= (1 - S) (x_1 y_3 + x_3 y_1)  \\
    F'_{14} &= (1 - S) (x_4 y_1)  \\
    F'_{22} &= 0 \\
    F'_{23} &= (1 - S) (x_2 y_3)  \\
    F'_{24} &= 0 \\
    F'_{33} &= (1 - S) (x_3 y_3)  \\
    F'_{34} &= (1 - S) (x_4 y_3)  \\
    F'_{44} &= 0, \numberthis
\end{align*}\end{linenomath}

\noindent where $x_{i}$ and $y_{i}$ are functions $x_i=f(F_{ij},G_{ij},w^f_{ij},C,\delta,r)$ and $y_i=f(F_{ij},G_{ij},w^m_{ij},C,\delta)$ describing the haplotype frequencies among ovules and pollen, and $S$ is the proportion of all ovules produced by the population that are self-fertilized. Note that $y_2=y_4=0$ because these male gametic haplotypes cannot be produced when $M_2$ is dominant. The genotypic frequencies in the next generation among zygotes formed by self-fertilization are 
\begin{linenomath}\begin{align*} \label{eq:GprGyn}
    G'_{11} &= S (o^S_{11} + o^S_{13}/4) \\
    G'_{13} &= S (o^S_{13}/2) \\
    G'_{33} &= S (o^S_{33} + o^S_{13}/4), \numberthis
\end{align*} \end{linenomath}

\noindent where $o^S_{ij}$ are functions $f(F_{ij},G_{ij},w^f_{ij},C,\delta)$ describing the proportional contribution of each genotype to self-fertilized ovules. All $G'_{ij} = 0$ where $ij \neq [11,13,33]$. 

The basic form of the recursions does not change when $M_2$ is completely recessive, but there two notable differences. Because only $M_2M_2$ homozygotes are unisexual females, none of the recursions reduce to zero, and the recombination rate enters into the nonzero $G'_{ij}$ recursions. For the same reason, all $x_i$ and $y_i$ describing the haplotype frequencies among ovules and pollen become nonzero functions $f(F_{ij},G_{ij},w^m_{ij},C,\delta,r)$ including the recombination rate.


%%%%%%%%%%%%%%%%%%%%%%%%
\subsection{Androdioecy}

The successful invasion of a dominant $M_2$ allele causing complete female-sterility (e.g., production of no, or inviable, ovules) represents the evolution of androdioecy. In contrast to the models of gynodioecy described above, the reproductive compensation term, $k$, now describes the proportional increase in pollen production by males relative to hermaphrodites. The fitness expressions, $w^f_{ij}$ and $w^m_{ij}$ resemble those described in Table~\ref{tab:fitness}, except the fitness effects of the $\mathbf{M}$ locus apply to the female rather than male sex-function. 

The genotype $\times$ transmission mode recursions for the models of androdioecy are very similar to those described for the models of gynodioecy, with a few key differences. When $M_2$ is dominant, $x_{i}$ and $y_{i}$ are again functions describing the haplotype frequencies among ovules and pollen, but now the recombination rate drops out of the expressions for $x_i=f(F_{ij},G_{ij},w^f_{ij},C,\delta)$, comes into the expressions for $y_i=f(F_{ij},G_{ij},w^m_{ij},C,\delta,r)$, and $x_2=x_4=0$. The genotypic frequencies in the next generation among zygotes formed by outcross fertilization are:
\begin{linenomath}\begin{align*} \label{eq:FprAnd}
    F'_{11} &= (1 - S) (x_1 y_1) \\
    F'_{12} &= (1 - S) (x_1 y_2) \\
    F'_{13} &= (1 - S) (x_1 y_3 + x_3 y_1) \\
    F'_{14} &= (1 - S) (x_1 y_4) \\
    F'_{22} &= 0 \\
    F'_{23} &= (1 - S) (x_3 y_2) \\
    F'_{24} &= 0 \\
    F'_{33} &= (1 - S) (x_3 y_3) \\
    F'_{34} &= (1 - S) (x_3 y_4) \\
    F'_{44} &= 0, \numberthis
\end{align*} \end{linenomath}

\noindent where $S$ is again equal to the proportion of all ovules produced by the population that are self-fertilized, but now accounts for the fact that heterozygotes at the $\mathbf{M}$ locus do not produce ovules. The form of the $G'_{ij}$ recursions remain unchanged from Eq(\ref{eq:GprGyn}).

When the $M_2$ female-sterility allele is completely recessive, only $M_2M_2$ homozygotes are unisexual males. The form of the recursions is very similar to the case of gynodioecy with a recessive sterility allele. The fitness effects arising from the $\mathbf{M}$ locus now only affect the female sex-function, and all $x_i$ and $y_i$ describing the haplotype frequencies among outcrossed ovules and pollen become nonzero functions $f(F_{ij},G_{ij},w^f_{ij},C,\delta,r)$ including the recombination rate.


%%%%%%%%%%%%%%%%%%%%%%%%
\subsection{Analyses} \label{subsec:analyses}

Our analyses address three main theoretical questions: (1) Does linkage to SA loci expand conditions for the evolutionary invasion of unisexual sterility? (2) How does linkage to SA loci affect the frequency of unisexual individuals in andro- and gynodioecious populations? (3) How does the invasion of sterility alleles impact evolutionary dynamics, including conditions for polymorphism at SA loci? As in previous models of sexually antagonistic selection (e.g., \citealp{Kidwell1977,Prout2000,JordanConnallon2014}), we limit our analyses to the representative, and biologically interesting cases of additive fitness effects ($h_m = h_f = 1/2$), and dominance reversal ($h_f,h_m < 1/2$) at the $\mathbf{A}$ locus. These scenarios are of particular interest because additive fitness effects are commonly observed for alleles with small to intermediate fitness effects \citep{Agrawal2011}, and dominance reversals are often predicted by fitness landscape models of dominance \citep{Manna2011, ConnallonClark2014}.

We first examine the case where $\mathbf{A}$ is monomorphic, and identify the parameter conditions where balancing selection is predicted to maintain stable polymorphism at $\mathbf{A}$ and unisexuals are able to invade. We evaluate the stability of the system of recursions for a populations initially fixed for the female-beneficial or male-beneficial allele at the $\mathbf{A}$ locus, and the $M_1$ allele at the $\mathbf{M}$ locus. These conditions correspond to initial equilibrium genotypic frequencies among outcrossed and selfed zygotes of $F_{11} = (1 - C) [AAM_1M_1] = 1$, $F_{33} = (1 - C) [aaM_1M_1] = 1$, and $G_{11} = C [AAM_1M_1] = 1$, and $G_{33} = C [aaM_1M_1] = 1$. Under these initial conditions, the fate of new mutations is determined by the rate of change of the frequencies of rare genotypes, which can be approximated by one minus the leading eigenvalue of the Jacobian matrix of the system of recursions, $\lambda_L - 1$ \citep{OttoDay2007}. Balancing selection is predicted to maintain polymorphism when $\lambda_L > 1$ for both boundary equilibria \citep{Prout1968,OttoDay2007}. For the models involving dominant male- and female-sterility alleles, this analysis yields three analytically tractable candidate leading eigenvalues describing the invasion of a new mutation at each locus independently ($\lambda_\mathbf{A}$ and $\lambda_\mathbf{M}$), and the joint invasion of a double-mutant haplotype ($\lambda_\mathbf{AM}$). When sterility alleles are recessive, analysis of the eigenvalues yield inconclusive results for invasion of unisexuals because heterozygotes at the $\mathbf{M}$ locus are entirely sheltered from selection. We therefore present analytic results based on the eigenvalue analyses for the models of dominant sterility alleles only.

To explore how segregating SA variation influences the invasion of unisexuals into populations of hermaphrodites, we evaluated whether the $M_2$ allele could invade populations initially at polymorphic single-locus equilibrium for $\mathbf{A}$ and fixed for $M_1$ at $\mathbf{M}$. Specifically, we evaluated whether $\lambda_\mathbf{M} \rvert_{F^{\ast}_{ij},G^{\ast}_{ij}} > 1$ and $\lambda_\mathbf{AM} \rvert_{F^{\ast}_{ij},G^{\ast}_{ij}} > 1$, where $F^{\ast}_{ij}$ correspond to the single-locus equilibrium genotypic frequencies
\begin{linenomath}\begin{align*} \label{eq:SA1locEQ}
	F^{\ast}_{11} &= (1 - C)[AAM_1M_1],& G^{\ast}_{11} &= C[AAM_1M_1]  \\
	F^{\ast}_{13} &= (1 - C)[AaM_1M_1],& G^{\ast}_{13} &= C[AaM_1M_1] \\
	F^{\ast}_{33} &= (1 - C)[aaM_1M_1],& G^{\ast}_{33} &= C[aaM_1M_1] \numberthis \\
\end{align*}\end{linenomath}

\noindent and all other $F^{\ast}_{ij} = G^{\ast}_{ij} =0$. Assuming obligate outcrossing and additive effects, analytic solutions are possible for the the single-locus allele frequencies at $\mathbf{A}$ among males and females ($\hat{p}_f,\hat{p}_m$; \citealt{Kidwell1977}). This is not possible for partial selfing, or in the case of dominance reversal. However, under these conditions, the single-locus equilibrium frequencies can be estimated using weak selection approximations for $\hat{p}$ ($s_m,s_f \ll 1$), which we then use to calculate $F^{\ast}_{ij}$ and $G^{\ast}_{ij}$. For example, under complete dominance reversal ($h_f=h_m=0$), the single-locus equilibrium frequency of a male-beneficial allele under weak selection can be approximated as $\hat{p}_{DR} \approx s_m/(s_f + s_m)$ \citep{ConnallonJordan2016}. More complex expressions can be found for the case of incomplete dominance reversal (e.g., $h_f=h_m=1/4$; \citealt{JordanConnallon2014,Olito2016}). In both cases, the approximations perform reasonably well when selection is not especially strong ($s_f,s_m \leq 0.5$; \citealt{JordanConnallon2014,ConnallonJordan2016,Olito2016}). 

To determine the equilibrium frequencies of SA alleles, $M_2$, and unisexuals invading polymorphic populations, we performed deterministic simulations of the recursions $F'_{ij},~G'_{ij}$ using the initial genotypic frequencies described by Eq(\ref{eq:SA1locEQ}). We focus our analysis on the comparison of equilbrium frequencies of unisexuals from our models with the corresponding exact single-locus equilibrium frequencies, $\hat{Z}$, given by \citet{Charlesworth1978a} for cases of relatively tight linkage between the $\mathbf{A}$ and $\mathbf{M}$ ($r\leq 0.1$). We emphasize tight linkage for two reasons. First, this represents the parameter conditions where our model predictions should differ most from those of single-locus models; with higher recombination, the predictions from the one- and two-locus models converge. Second, tight or complete linkage should approximate the biologically plausible scenario where SA loci, including a sex-specific sterility locus, are involved in a chromosomal inversion \citep{LeeKelly2015}.

In single-locus models of gynodioecy and androdioecy, the equilibrium frequency of unisexuals is determined by the reproductive compensation term, $k$, and the compound parameter $C \delta$ \citep{Charlesworth1978a}. However, if inbreeding depression is caused primarily by recessive deleterious mutations, as current data suggest \citep{Charlesworth2009}, there should be strong negative covariance between $C$ and $\delta$ as increased selfing more effectively purges deleterious recessives. In the simplest case, the mutation load due to deleterious recessive mutations at a single locus in a completely selfing population should be roughly half that of a randomly mating outcrossing population ($\mu$ versus $2 \mu$, where $\mu$ is the genome-wide mutation rate; \citealt{OhtaCockerham1974}). To account for negative covariance between $C$ and $\delta$, and thereby explore more biologically meaningful parameter space, we constrain inbreeding depression for our simulations to be a linear declining function of the selfing rate: $\delta = \delta^\ast(1 - C/2)$, where $\delta^\ast$ represents the hypothetical severity of inbreeding depression if selfing were enforced on a completely outcrossing population ($\delta^\ast \in [0,1]$). More complex expressions for $\delta$ yielded qualitatively similar results (See Appendix C in the Online Supporting Information). 

Finally, it was important to account for the fact that $C$ and $\delta$ also influence the maintenance of SA polymorphism and the equilibrium frequencies of the SA alleles at $\mathbf{A}$ \citep{JordanConnallon2014,Olito2016}. We therefore ran simulations using values of $s_f$ and $s_m$ corresponding to single-locus equilibrium frequencies of $1/2$ for the two SA alleles ($p = [A]$, $q = [a]$; $p=q=1/2$). Thus, our simulations explore the invasion of the male- or female-sterility allele, $M_2$, into hermaphroditic populations initially at highly polymorphic single-locus equilibrium at $\mathbf{A}$.


\subsection{Data availability}
A full development of all models can be found in Appendix A of the Online Supporting Information, and all code necessary to reproduce the analyses are available in the Online Supporting Information, and at \url{https://github.com/colin-olito/dioecySA}.

%%%%%%%%%%%%%%%%%%%%%%%%
\section{Results}
%%%%%%%%%%%%%%%%%%%%%%%%

%%%%%%%%%%%%%%%%%%%%%%%%
\subsection{Invasion into monomorphic populations}

%%%%%%%%%%%%%%%%%%%%%%%%
\subsubsection{Gynodioecy:} We begin with the evolution of gynodioecy by the invasion of a dominant male-sterility allele, $M_2$, into populations initially fixed for the $AAM_1M_1$ and $aaM_1M_1$ genotypes. We focus on results for additive fitness effects at the $\mathbf{A}$ locus, and discuss the effect of dominance reversals at the end of each section of the results. Solving for the conditions where $\lambda_{\mathbf{A}} > 1$, we were able to recover the well known single-locus invasion criteria for SA alleles under obligate outcrossing \citep{Kidwell1977} and partial selfing \citep{JordanConnallon2014,Olito2016}, with either additive fitness effects, or dominance reversal. Solving $\lambda_{\mathbf{M}} > 1$ for $k$ also yields the classic single-locus criterion for the invasion of females into a population of hermaphrodites (Eq(4) in \citealt{Charlesworth1978a}):

\begin{equation}\label{eq:1LocGyn}
	\hat{k} > 1 - 2 C \delta.
\end{equation}

\noindent The invasion conditions for a mutant haplotype bearing $M_2$ depend on whether the population is initially fixed for the female-beneficial ($A$) or male-beneficial ($a$) allele at $\mathbf{A}$. In both cases, solving $\lambda_{\mathbf{AM}} > 1$ for $k$ yield expressions $f(C,\delta,s_f,r)$ of the same basic form as Eq(\ref{eq:1LocGyn}), in which $k$ is a decreasing function of $C \delta$. For populations initially fixed for $A$, the haplotype invasion conditions are always more restrictive (requiring larger $k$) than those described by Eq(\ref{eq:1LocGyn}), except in the limit of complete linkage and no selection against $a$ ($r \rightarrow 0$ and $s_f \rightarrow  0$). In contrast, for populations initially fixed for the male-beneficial allele, $a$, the invasion conditions become more permissive with either lower recombination, or stronger selection against $a$. Specifically, the condition for the spread of mutant $AM_2$ haplotypes becomes more permissive than Eq(\ref{eq:1LocGyn}) when 

\begin{equation}\label{eq:2LocGyn}
	\frac{2 r}{1 + r} < s_f.
\end{equation}

\noindent Eq(\ref{eq:2LocGyn}) shows that the invasion of females into a population of hermaphrodites can be quite sensitive to linkage with a nearby SA locus. When selection is relatively weak ($s_f = 0.1$), even modest linkage ($r\approx0.05$) can reduce the amount of reproductive compensation necessary for females to invade relative to the single-locus prediction. 

Linkage with a male-sterility locus also influences the invasion of female-beneficial SA alleles. For example, under obligate outcrossing and additive fitness effects, the single-locus invasion condition for the female-beneficial allele, $A$, is $s_f > s_m / (1+s_m)$ (\citealt{Kidwell1977}; also from $\lambda_{\mathbf{A}}$). Substituting for $s_f$ in Eq(\ref{eq:2LocGyn}) and solving for $r$ yields the degree of linkage necessary to expand the invasion conditions for a female-benefit allele beyond the single-locus criterion:

\begin{equation}\label{eq:2LocGynSA}
	r < \frac{s_m}{2 + s_m}.
\end{equation}

\noindent By the same method, the criterion for the invasion of a female-beneficial allele under partial selfing and additive fitness effects will become more permissive than the single-locus criterion when

\begin{equation}\label{eq:2LocGynSApartSelf}
	r < \frac{s_m (1 - C)}{2 + s_m +C (2 - s_m - 4 \delta)}.
\end{equation}

\noindent Eq(\ref{eq:2LocGynSApartSelf}) shows that the scope for linkage between SA loci to expand the parameter space where female-beneficial alleles can invade is reduced with higher selfing (larger $C$), but inbreeding depression (larger $\delta$) can compensate for this effect by 'enforcing' outcross reproduction. This outcome arises directly from the increasing "female bias" in selection caused by selfing. With higher selfing, there is simply less parameter space where linkage with a male-sterility locus can facilitate the invasion of the female-beneficial allele \citep{JordanConnallon2014,Olito2016}.


%%%%%%%%%%%%%%%%%%%%%%%%
\subsubsection{Androdioecy:} We performed a stability analysis for the model for the evolution of androdioecy via the invasion of a dominant female-sterility allele, $M_2$, into populations initially fixed for the $AAM_1M_1$ genotype or the $aaM_1M_1$ genotype. As before, analysis of $\lambda_{\mathbf{A}}$ recovered the single-locus invasion criteria for SA alleles under obligate outcrossing and partial selfing under both additive fitness effects and dominance reversal conditions. Likewise, solving $\lambda_{\mathbf{M}} > 1$ for $k$ yielded the familiar invasion criterion for males into a population of hermaphrodites (Eq(8) in \citealt{Charlesworth1978a}),

\begin{equation}\label{eq:1LocAndro}
	\hat{k} > \frac{1 + C (1 - 2 \delta)}{(1 - C)}.
\end{equation}

\noindent Not surprisingly, the invasion conditions for mutant haplotypes bearing $M_2$ again depended on the initial fixed genotype of the hermaphrodite population, and solving $\lambda_{\mathbf{AM}} > 1$ for $k$ yields expressions $f(C,\delta,s_m,r)$ of the same form as Eq(\ref{eq:1LocAndro}). Mirroring the results of the previous model, the conditions satisfying $\lambda_{\mathbf{AM}} > 1$ for the invasion of a mutant haplotype were always more restrictive than Eq(\ref{eq:1LocAndro}) when the population was initially fixed for the male-beneficial allele ($a$; except when $s_m \rightarrow 0$ and $r \rightarrow 0$), and more permissive for populations initially fixed for the female-beneficial allele ($A$). In fact, replacing $s_f$ with $s_m$ in Eq(\ref{eq:2LocGyn}) gives the conditions under which invasion of males via the spread of mutant $aM_1$ haplotypes become more permissive than the single locus criteria described by Eq(\ref{eq:1LocAndro}). Thus, although the conditions necessary for the evolution of androdioecy are more restrictive than for gynodioecy overall, linkage to a SA locus similarly impacts the invasion conditions for female-sterility alleles and for male-sterility alleles. 

The effect of linkage with a female-sterility locus on the invasion of male-beneficial SA alleles is nearly identical for the case of obligate outcrossing. This is due to the symmetry of the single-locus SA invasion conditions \citep{Kidwell1977}. In this case, substituting $s_m$ in for $s_f$ in Eq(\ref{eq:2LocGynSA}) gives the conditions where invasion of male-beneficial alleles becomes more permissive than the single-locus predictions. However, the case is altered subtantially in partially selfing populations. For example, assuming additive fitness effects, linkage to a female-sterility locus expands the invasion conditions of male-beneficial alleles at an SA locus when

\begin{equation}\label{eq:2LocAndroSApartSelf}
	r < \frac{s_f + s_f C (1 - 2 \delta)}{2 + s_f - 2 C + s_f C (1 - 2 \delta)}.
\end{equation}

\noindent The right hand side of Eq(\ref{eq:2LocAndroSApartSelf}) increases with both $C$ and $s_f$ such that, for $C > 1/3$, there is always some expansion of the parameter space where male-beneficial alleles can invade beyond the single-locus expectation, even under free-recombination with the female-sterility locus ($r = 1/2$). 


\begin{figure}[htbp]
\centering
\includegraphics[width=\linewidth]{Fig1}
\caption{Invasion of unisexuals into populations with pre-existing SA polymorphism. Plots show the fraction of parameter space (defined by $s_f \times s_m | F^{\ast}_{ij},G^{\ast}_{ij} > 0$ and $0 < s_f,s_m \leq 0.5$) where a dominant sex-specific sterility allele at $\mathbf{M}$, can invade populations initially at single-locus equilibrium frequencies for $\mathbf{A}$ with additive fitness effects ($h_f=h_m=1/2$), plotted as a function of the recombination rate $r$. Panels A--C show results from the model of gynodioecy via invasion of a male-sterility allele, while planels D--F show results for the model of androdioecy via invasion of a female-sterility allele. For each panel, results are shown for different values of reproductive compensation, $k$, chosen as a fraction of the single-locus invasion criterion for $M_2$ defined by Eq(\ref{eq:1LocGyn}) and Eq(\ref{eq:1LocAndro}) for the models of gynodioecy and androdioecy respectively. Hence, the orange, green, and dark blue lines show scenarios where unisexuals experience a decrease in gamete production relative to hermaphrodites of $1$, $5$, and $10\%$. Note the different scale for the x-axis in panels C and F. Results were obtained by evaluating the three candidate leading eigenvalues ($\lambda_{\mathbf{A}}$,$\lambda_{\mathbf{M}}$,$\lambda_{\mathbf{AM}}$) of the Jacobian matrix of the genotype $\times$ transmission mode recursions for populations at the above initial conditions for $1000$ points uniformly distributed throughout the relevant $s_f \times s_m$ parameter space.}
\label{fig:PrInv}
\end{figure}

%%%%%%%%%%%%%%%%%%%%%%%%
\subsubsection{Dominance reversals $(h_f,h_m < 1/2)$:} Under dominance reversal scenarios at the SA locus, the invasion criteria for SA alleles become more permissive than when SA fitness effects are additive \citep{Kidwell1977,ConnallonClark2012,JordanConnallon2014,Olito2016}. In our models, when linkage exists between the SA locus and the sterlity locus, dominance reversal scenarios at the SA locus expand the conditions where the $M_2$ sterility allele is predicted to invade. For the model of gynodioecy via dominant male-sterility allele under obligate outcrossing ($C=0$), an incomplete dominance reversal ($h_f=h_m=1/4$) alters the conditions where the spread of mutant $AM_2$ haplotypes in populations fixed for female-beneficial SA alleles becomes more permissive than Eq(\ref{eq:1LocGyn}) to be

\begin{equation}\label{eq:2LocGynDomRev}
	\frac{4 r}{3 + r} < s_f.
\end{equation}

\noindent Under relatively weak selection ($s_f=0.1$), this reduces the degree of linkage required to for unisexual females to invade from $r \approx 0.05$ (for the additive case), to $r \approx 0.07$. Similarly, linkage with a male-sterility locus expands the parameter conditions for a female-benefit allele beyond the single-locus criterion when

\begin{equation}\label{eq:2LocGynDomRevSA}
	r < \frac{s_m}{4 + s_m}.
\end{equation}


%%%%%%%%%%%%%%%%%%%%%%%%
\subsection{Invasion of unisexuals into polymorphic populations}

Three factors determine the fate of new sterility mutations in populations that are initially polymorphic at $\mathbf{A}$: the degree of reproductive compensation ($k$), the rate of recombination ($r$), and the selfing rate ($C$). Unisexuals are always able to invade populations with segregating SA variation if either Eq(\ref{eq:1LocGyn}) (for gynodioecy) or Eq(\ref{eq:1LocAndro}) (for androdioecy) are satisfied (fig.~\ref{fig:PrInv}; light blue lines). Unisexuals can still invade when $k < \hat{k}$, provided there is some linkage between $\mathbf{A}$ and $\mathbf{M}$. For example, under obligate outcrossing and tight linkage, unisexuals can invade a polymorphic population across $\approx 69\%$ of relevant parameter space (defined by $s_f \times s_m$ where $0 < s_f,s_m \leq 0.5$, and $F^{\ast}_{ij} > 0$), despite a $10\%$ reduction in gamete production relative to hermaphrodites (fig.~\ref{fig:PrInv}A,D). With smaller reductions in relative gamete production, unisexuals can invade across a greater fraction of parameter space, even when linkage is quite weak (e.g., $\approx 38\%$ when $k = \hat{k} \times 0.95 $ and $r = 0.2$). 

Unisexuals can invade under similar conditions if the selfing rate among hermaphrodites is relatively low and inbreeding depression is relatively high ($C = 1/4,~\delta = 4/5$) (fig.~\ref{fig:PrInv}B,E). This because under these conditions the majority of offspring are produced through outcrossing rather than selfing. For populations with relatively high selfing rates among hermaphrodites and low inbreeding depression ($C = 3/4,~\delta = 1/5$), major differences between the models of gyno- and androdioecy emerge. Most notably, much tighter linkage is required for the male-sterility mutation to spread when $k < \hat{k}$. Even then, unisexual females will only invade if the reduction in ovule production by females relative to hermaphrodites is quite small (fig.~\ref{fig:PrInv}C,F). In contrast, the invasion of female-sterility mutations does not become sensitive to $r$, and unisexual males can invade over similar fractions of parameter space as in predominantly outcrossing populations. This contrast in the effect of selfing between the models of gynodioecy and androdioecy is a consequence of the increasing female-bias in the net direction of SA selection caused by increased selfing among hermaphrodites \citep{Charlesworth1978a,JordanConnallon2014,Olito2016}. Analogous to previous results for linked SA loci \citep{Olito2016}, as the female-bias in SA selection increases with selfing, there is greater scope for linkage to a male-beneficial SA allele to facilitate invasion of a female-sterility mutation.

In the models of both gyno- and androdioecy, the loss of parameter space where unisexuals can invade is determined by the rate of recombination, the degree of reproductive compensation, and the relative strength of seleciton through each sex-function. With weaker linkage between $\mathbf{A}$ and $\mathbf{M}$, invasion of unisexuals requires stronger SA selection coefficients that must also be increasingly biased toward the gender of the invading unisexuals. Hence, the invasion of male-sterility alleles requires female-baised selection, while the invasion of female-sterility requires male-biased selection (see figs.~S1--S6 in the Online Supporting Information). \hl{The conditions for unisexual invasion into polymorphic populations are not qualitatively different under dominance reversal conditions}. However, as in previous SA models, a dominance reversal expands the region of parameter space where polymorphism is maintained at $\mathbf{A}$, and simultaneously expands the parameter space where unisexual are able to invade in our models (see figs.~S1--S6 in the Online Supporting Information).

%%%%%%%%%%%%%%%%%%%%%%%%
\subsection{Equilibrium frequencies of unisexuals}

When unisexual sterility alleles arise in linkage with another polymorphic SA locus, it can result in elevated equilibrium frequencies of unisexuals relative to single-locus predictions. If reproductive compensation by unisexuals satisfies the conditions for single-locus invasion of $M_2$ (i.e., $k \geq \hat{k}$), the increase in the equilibrium frequency of unisexuals relative to single-locus predictions can be quite large (fig.~\ref{fig:eqFreq2v1Loc}, greyscale lines). As long as linkage among the SA loci remains relatively tight ($r < 0.05$), the effect is strongest in predominantly outcrossing populations, and becomes weaker with increasing selfing rates among hermaphrodite individuals (with concomitant decrease in inbreeding depression; see \nameref{sec:methods}). With weaker linkage, the two-locus predictions converge on those of the single-locus models. Although the single-locus invasion condition for the $M_2$ allele ($\hat{k}$) is significantly higher for androdioecy than gynodioecy (Eq.\ref{eq:1LocGyn} vs.~Eq.\ref{eq:1LocAndro}), linkage among SA loci results in similarly elevated equilibrium frequencies of unisexuals in either model.

\begin{figure}[htbp]
\centering
\includegraphics[width=\linewidth]{Fig2Alt}
\caption{Equilibrium frequencies of unisexual (A) females, and (B) males, compared with single-locus predictions when reproductive compensation is above the single-locus threshold (i.e., $k > \hat{k}$). Results are shown for the models of gyno- and androdioecy via invasion of recessive unisexual sterility alleles, additive fitness effects at $\mathbf{A}$ ($h_f = h_m = 0.5$, using selection coefficients of $s_m = 0.1$ (for the model of gynodioecy) and $s_f = 0.1$ (for the model of androdioecy), and inbreeding depression that follows $\delta = \delta^\ast(1 - C/2)$ (see \nameref{sec:methods}). Plots illustrate the increase in equilibrium frequencies of unisexuals predicted by our two-locus models (dashed greyscale lines) relative to the corresponding exact single-locus equilibrium frequencies (solid black line; $\hat{Z}$ predicted by \citealt{Charlesworth1978a}). Results are shown for four different levels of recombination, highlighting that with weaker linkage, the two-locus predictions converge on those of the single-locus model.}
\label{fig:eqFreq2v1Loc}
\end{figure}

When linkage exists among SA loci, unisexual sterility alleles can still invade when reproductive compensation falls below the single-locus threshold for invasion ($k < \hat{k}$; fig.~\ref{fig:eqFreq}). Under tight linkage, the increase in equilibrium unisexual frequencies is again greatest for predominantly outcrossing populations, and decreases as the rate of self-fertilizaiton among hermaphrodite indiviuals increases (fig.~\ref{fig:eqFreq}A,C). This effect diminishes with weaker linkage, where equilibrium unisexual frequencies drop to $0$ with smaller and smaller decreases in reproductive compensation relative to the single locus invasion criterion (fig.~\ref{fig:eqFreq}B,D). In this case, the decrease in equilibrium male frequencies in the model of androdioecy was more sensitive to reductions in $k$ relative to $\hat{k}$ than for the model of gynodioecy. Under weaker linkage or lower reproductive compensation, equilibrium frequencies of unisexuals are predicted to be highest for populations with intermediate selfing rates. This pattern results from three counterbalancing effects arising from our assumption that inbreeding depression, $\delta$, decreases with the selfing rate, $C$, in our simulations. On one hand, higher selfing rates favour the invasion of unisexuals by increasing the rate at which $M_2 M_2$ homozygotes (unisexuals in the recessive models) are formed from rare $M_2$ mutants increases, and lowers inbreeding depression. On the other hand, unisexuals can only reproduce by outcrossing, and higher selfing eventually limits their contribution to the genetic composition of the next generation. 


\begin{figure}[htbp]
\centering
\includegraphics[width=\linewidth]{Fig3Alt}
\caption{Equilibrium frequencies of unisexual (A--B) females and (C--D) males across a gradient of reproductive compensation. Results are shown for the models of gyno- and androdioecy via invasion of recessive unisexual sterility alleles, additive fitness effects at $\mathbf{A}$ ($h_f = h_m = 0.5$, using selection coefficients of $s_m = 0.1$ (for the model of gynodioecy) and $s_f = 0.1$ (for the model of androdioecy), and inbreeding depression that decreases linearly with the selfing rate: $\delta = \delta^\ast(1 - C/2)$ (see \nameref{sec:methods}). Plots show the equilibrium frequencies of unisexuals predicted by our models for five different levels of reproductive compensation, calculated as a fraction of the single-locus invasion criterion for $M_2$ defined by Eq(\ref{eq:1LocGyn}) and Eq(\ref{eq:1LocAndro}). Note that the single-locus equilibrium frequency of unisexuals always equals $0$ when $k < \hat{k}$. Hence, the lines corresponding to $k < \hat{k}$ illustrate how linkage among SA loci expands the parameter space where unisexual sterility alleles can invade beyond the predictions of the single-locus models.}
\label{fig:eqFreq}
\end{figure}


%%%%%%%%%%%%%%%%%%%%%%%%
%%%%%%%%%%%%%%%%%%%%%%%%
\section{Discussion}

The first theoretical insight provided by our models is that linkage among SA loci facilitates the invasion of unisexual sterility alleles, and elevates the equilibrium frequencies of unisexuals relative to single-locus predictions. The second is that when linkage is relatively tight, this effect is greatest for predominantly outcrossing populations -- suggesting that the ancestral hermaphrodite selfing rate may play a different role in the evolution of separate sexes than predicted by the classic models of \citep{Charlesworth1978a}. Overall, these predictions suggest that the unisexual sterility alleles driving the evolution of dimorphic sexual systems are likely to evolve in genomic regions harboring polymorphic SA loci. When this occurs, elevated frequencies of unisexuals in the resulting gyno- and androdioecious populations are more likely to evolve than previously predicted, facilitating subsequent transitions to separate sexes \citep{Charlesworth1978a}. Below, we discuss the implications of our findings, and suggest empirical tests of our predictions, in three main contexts: SA polymorphism and the evolution of gyno- and androdioecy, hermaphrodite mating systems and the evolution of dioecy, and the population genetic basis of the evolution of separate sexes.

%%%%%%%%%%%%%%%%%%%%%%%%
\subsection{SA Polymorphism and the evolution of dimorphic sexual systems}

Although dioecy is relatively uncommon among angiosperms (represented in $\approx 7\%$ of genera), the fantastic diversity and repeated evolution of dimorphic sexual systems in flowering plants begs a genetical explanation \citep{Renner2014,KaferPannell2017}. Several evolutionary pathways, and a variety of genetic mechanisms, lead from hermaphroditism to separate sexes, but all ultimately involve the evolutionary invasion of at least two unisexual sterility alleles \citep{Charlesworth1978a,Charlesworth1978b,Renner2014,Ashman2015}. When accompanied by allocation trade-offs between sex-functions, unisexual sterility alleles are an important class of SA allele, yet previous theory has not considered the potential influence of standing SA genetic variation, or genetic architecture, on their invasion and establishment in hermaphroditic populations. Our theoretical results suggest that physical linkage among SA loci can strongly influence the evolution of gyno- or androdioecy from hermaphroditism, expanding the conditions under which dimorphic sexual systems, and ultimately separate sexes, are predicted to evolve. 

Classical theoretical predictions, based on single-locus models, suggest that conditions for the evolution of dioecy may be quite stringent \citep{Lloyd1975,Lloyd1976,Charlesworth1978a,KaferPannell2017}. For unisexual sterility alleles to invade, the resulting unisexuals must adequately compensate for the loss of a sex function through increased gamete production. On the other hand, we find that the conditions for the spread of unisexual sterility alleles can be quite permissive -- especially if the sterility mutation arises on a haplotype bearing a complimentary SA allele (e.g., one that is beneficial for the same sex-function as the invading unisexuals). In this case, invasion requires only modest linkage for biologically plausible selection coefficients (e.g., $s_f,s_m \leq 0.1$). When linkage does exist, the fitness effects of the complimentary SA allele help offset the loss of a sex function, and reduce the amount of reproductive compensation required by unisexuals to invade the population (i.e., relative to single-locus predictions; \citealt{Charlesworth1978a}). These results are similar to other multilocus models of sex-specific selection and mutations, which show that predominantly female-harming mutations tend to accumulate on haplotypes carrying male-benefit alleles \citep{ConnallonJordan2016}. An important corollary of this result is that the conditions for the maintenance of SA polymorphism are also expanded by linkage to sex-specific sterility mutations. Hence, new unisexual sterility mutations underpinning dimorphic sexual systems are most likely to evolve in tight linkage with other SA loci, and should simultaneously promote the maintenance of SA polymorphism at linked loci. 

The amount of standing SA genetic variation should directly influence the potential for hermaphroditic populations to evolve dimorphic sexual systems. Although the invasion conditions for unisexual sterility alleles into populations without segregating SA variation are generally favourable, the waiting time for double mutants with the necessary haplotype (e.g., female-beneficial--male-sterile) to appear could be long \citep{WeinreichChao2005,ConnallonClark2010}. However, we find that the invasion conditions for unisexual sterility alleles are still quite permissive when there is standing SA genetic variation in hermaphroditic populations. Although it is not yet clear how much SA genetic variation for fitness is harbored by hermaphroditic species, three features of SA selection suggest that this is not an unlikely scenario, particularly in large populations. First, balancing selection is predicted to maintain SA polymorphism in partially selfing populations over a broad spectrum of parameter conditions, particularly when SA loci are linked \citep{Patten2010,JordanConnallon2014,Olito2016}. Second, net directional selection under SA is predicted to be small, even when fitness effects in each sex are large, resulting in relatively long persistence times of SA alleles \citep{ConnallonClark2012}. This facilitates the formation of double-mutant haplotypes (e.g., \citealt{WeinreichChao2005}), and leads to and elevated contribution of SA alleles to fitness variance compared to other classes of mutations (e.g., unconditionally beneficial or deleterious alleles; \citealt{ConnallonClark2012}). Third, although definitively indentifying SA loci is challenging, there is empirical evidence suggesting that segregating SA allele frequencies can be non-trivial, even in partially selfing hermaphroditic populations \citep{Barson2015,LeeKelly2015}. Additional studies attempting to quantify the degree of SA genetic variation in hermaphroditic species would help clarify the potential for dimorphic sexual systems to evolve from hermaphroditism, especially if they were to target species exhibiting intraspecific variation in the degree or frequency of dimorphic sexual systems (e.g., many of the species reviewed in \citealt{SakaiWeller1999,Barrett2010,Renner2014}).


%%%%%%%%%%%%%%%%%%%%%%%%
\subsection{Mating systems and the evolution of dioecy}

The interplay between hermaphrodite mating systems and reproductive compensation is a key factor influencing the evolution of separate sexes in flowering plants, especially via the gynodieocy pathway \citep{Darwin1877,Charlesworth1978a}. Prior theory predicts that the evolution of gynodioecy is driven by the combination of reproductive compensation by unisexuals, and avoidance of inbreeding depression, and is therefore most likely to occur in partially selfing populations \citep{Lewis1942,Lloyd1975,Charlesworth1978a,KaferPannell2017}. This follows directly from the structure of Eq(\ref{eq:1LocGyn}) where $\hat{k}$ is determined entirely by the product of the selfing rate and inbreeding depression ($C \delta$). In contrast, the evolution of androdioecy is predicted to require significantly higher reproductive compensation, especially in partially selfing populations, because invading males must still compete with selfing hermaphrodites to fertilize ovules \citep{Charlesworth1978b,KaferPannell2017}. 

The population selfing rate plays a similarly critical role in our models. Linkage to a SA locus both expands conditions for invasion, and elevates the equilibrium frequencies, of unisexual indiviuals. Yet the effect of linkage, relative to single-locus models, is particularly pronounced in hermaphrodite populations with low rates of self-fertilization, and least pronounced in those with high selfing rates. The evolution of higher equilibrium frequencies of females in the gynodioecy model is particularly important as this should facilitate the subsequent evolution of dioecy via invasion of partial female-sterility mutations \citep{Charlesworth1978a,Charlesworth1978b,Charlesworth1999,Charlesworth2006}. The major implication is that when male-sterility mutations do arise in linkage with other SA loci, single-locus theory may significantly underestimate the potential for gynodieocy, and subsequently dioecy, to evolve in predominantly outcrossing species. Moreover, outcrossing hermaphrodite populations are more likely to harbor segregating SA genetic variation than selfing ones \citep{JordanConnallon2014,Olito2016}. The ancestral mating systems in which linkage to a polymorphic SA locus is predicted to have the largest effect on evolutionary transitions to dioecy are also the ones that are most likely to harbor segregating SA variation. 

For the evolution of androdioecy, modest linkage to an SA locus broadens the conditions for invasion of female-sterility alleles, even for highly selfing populations. Linkage also increases equilibrium conditions frequencies of unisexual males (given invasion), particularly in populations with low selfing-rates. Thus, when linkage exists between a female-steriliy allele and another SA locus, single-locus theory may underestimate the potential for the evolution of androdioecy, and subsequently, of dioecy. However, our models agree with previous theory that the conditions for invasion of female-sterility alleles in partially selfing populations are still quite stringent, requiring very high reproductive compensation (large $k$;\citealt{Charlesworth1978a}, Eq(\ref{eq:1LocAndro})). Overall, linkage with an SA locus will always faciliate the invasion of female-sterility mutations, but the evolution of androdioecy, and dioecy via the androdioecy pathway, is still expected to be quite rare relative to gynodioecy \citep{Charlesworth1978a,Charlesworth2006,KaferPannell2017,Renner2014}.

Despite the longstanding theoretical prediction of a strong correlation between the hermaphrodite mating system and dioecy, empirical evidence for this association remains equivocal \citep{Charlesworth1985,Charlesworth2006,Renner2014}. Our predictions suggest a partial explanation for this weak association. In species where unisexual sterility alleles arise in linkage with another SA locus, evolutionary transitions to dimorphic sexual systems should be elevated in taxa with predominantly outcrossing hermaphrodite ancestors. However, not all transitions to dioecy will have occurred by our proposed mechanism, and the net effect will be a weakened correlation between the ancestral hermaphrodite selfing rate, and dieocy. In light of our results, a re-examination of the evolutionary association between angiosperm mating and sexual systems using modern phylogenetic comparative methods would be most interesting, and would help identify species where our proposed mechanism for the evolution of gynodioecy is most likely to have played a role (e.g., species with dimorphic sexual systems that appear to have evolved from predominantly outcrossing ancestors). 


%%%%%%%%%%%%%%%%%%%%%%%%
\subsection{The population genetic basis of the evolution of separate sexes}

Our results also have implications for the role of SA alleles during the early stages of sex chromosome evolution. The process of sex chromosome origin and differentiation into distinct X and Y (or Z and W) chromosomes is thought to proceed by the following series of steps. First, a sex-determining locus originates on an ordinary pair of autosomes. Second, SA genes linked to the sex-determining locus accumulate over time. Third, this linked SA variation drives the evolution suppressed recombination between the neo sex-chromosomes, generally via chromosomal inversions. Fourth, the non-recombining sex chromosome degenerates (usually). Finally, mechanisms of dosage compensation evolve in response to the loss of coding regions on the degenerate sex chromosome (sometimes) \citep{Rice1987,Charlesworth2002,Bachtrog2006,Qiuetal2013,Bachtrog2014}. 

A key assumption that the emergence of a sex-determining locus is the first step in this process traces back (again) to classic population genetic theory of unisexual sterility alleles. The key prediction is that after the invasion of a recessive male-sterility allele (yielding gynodioecy), the subsequent invasion of dominant gender modifiers leading to full dioecy require tight linkage to the first male-sterility locus (the linkage constraint of \citealt{Charlesworth1978a}). This process leads to a tightly coupled pair of sex-determining loci, effectively a single sex determining locus. Once this sex determining locus arises, further differentiation following the steps outlined above can result in heteromorphic sex chromosomes.

Our theoretical results suggest a previously unrecognized, or at least underappreciated, role for SA genetic variation in hermaphroditic taxa during these initial stages of sex-chromosome evolution. Our finding that linkage with a polymorphic SA locus facilitates the invasion of unisexual sterility alleles strongly suggests that the accumulation of SA genetic variation may actually precede the origin of sex-determination loci. In other words, SA may represent the first (rather than a subsequent) step in the origin and evolution of new sex chromosome systems. This process is similar to prior theory regarding sex chromosome turnover \citep{vanDoornKirkpatrick2007,vanDoornKirkpatrick2010}, but is distinct in that it begins with SA variation in hermaphroditic populations, and deals with the invasion of unisexual sterility alleles rather than master sex-determining loci. In addition, our finding that linkage drives the evolution of elevated equilibrium frequencies of unisexual females should also facilitate the subsequent evolution of dioecy via tightly linked gender modifiers yielding separate sexes \citep{Charlesworth1978a}. The resulting complex of linked SA and sterility loci effectively establishes a nascent sex-determining-region on the neo sex-chromosome pair. If pre-existing SA polymorhism at the linked SA locus survives the transition to dioecy, the process of accumulating SA variation prior to recombination suppression will have already begun -- setting the stage for the subsequent accumulation of linked SA genetic variation and recombination suppression leading to sex-chromosome differentiation \citep{Charlesworth1978a,Rice1987,Bachtrog2006,Qiuetal2013}. Given the crucial role of linkage in each of these steps, it seems plausible that the degree of SA genetic variation present when an initial male-sterility mutation arises may help explain the large variation in the rate and extent of sex-chromosome differentiation in angiosperms \citep{Charlesworth2002,Renner2014,Bachtrog2014}.


\bibliography{dioecySA-bibliography}

\end{document}