%%%%%%%%%%%%%%%%%%%%%%%%%%%%%
%Preamble
\documentclass{article}

%Dependencies
\usepackage[left]{lineno}
\usepackage{titlesec}
\usepackage{color,soul}
\usepackage{ogonek}
\usepackage{float}


% Other Packages
%\usepackage{times}
\RequirePackage{fullpage}
\linespread{1.5}
\RequirePackage[colorlinks=true, allcolors=blue]{hyperref}
\RequirePackage[english]{babel}
\RequirePackage{amsmath,amsfonts,amssymb}
\RequirePackage[sc]{mathpazo}
\RequirePackage[T1]{fontenc}
\RequirePackage{url}

% Bibliography
%\usepackage[authoryear,sectionbib,sort]{natbib}
\usepackage{natbib} \bibpunct{(}{)}{;}{author-year}{}{,}
\bibliographystyle{amnatnat}
\addto{\captionsenglish}{\renewcommand{\refname}{Literature Cited}}
\setlength{\bibsep}{0.0pt}

% Graphics package
\usepackage{graphicx}
\graphicspath{{../output/figures/}.pdf}

% New commands: fonts
%\newcommand{\code}{\fontfamily{pcr}\selectfont}
%\newcommand*\chem[1]{\ensuremath{\mathrm{#1}}}
\newcommand\numberthis{\addtocounter{equation}{1}\tag{\theequation}}
\titleformat{\subsubsection}[runin]{\bfseries\itshape}{\thesubsubsection.}{0.5em}{}


%%%%%%%%%%%%%%%%%%%%%%%%%%%%%
% Title Page

\title{Sexually antagonistic polymorphism and the evolution of dimorphic sexual systems in hermaphrodites}
\author{Colin Olito$^{\ast,1,2}$ \& Tim Connallon$^{1}$}
\date{\today}

\begin{document}
\maketitle


\noindent{} $^{1}$ Centre for Geometric Biology, School of Biological Sciences, Monash University, Victoria 3800, Australia.

\noindent{} $^{2}$ \textit{Present address}: Department of Biology, Section for Evolutionary Ecology, Lund University, Lund 223 62, Sweden.

\noindent{} $^{\ast}$ Corresponding author e-mail: \url{colin.olito@gmail.com}

\bigskip

\noindent{} \textit{Manuscript elements}: Figure~1, Figure~2, Figure~3, Table~1; Online Supplementary Material: Appendix A -- Development of the recursions; Appendix B -- Supplementary figures; Appendix C -- Alternative relations between the selfing rate and inbreeding depression.

\bigskip
\noindent{} \textit{Running Head}: SA polymorphism and dimorphic sexual systems

\bigskip

\noindent{} \textit{Keywords}: Androdioecy; Dioecy; Gynodioecy; Intralocus sexual conflict; Linkage disequilibrium; Recombination; Sexual system; Sexual dimorphism

\bigskip

\noindent{} \textit{Manuscript type}: Article

\bigskip


% Set line number options
\linenumbers
\modulolinenumbers[1]
\renewcommand\linenumberfont{\normalfont\small}

%%%%%%%%%%%%%%%%%%%%%%%%%%%%%
% Main Text

\newpage{}
\section*{Abstract}

\noindent{} Multicellular Eukaryotes employ a broad spectrum of sexual reproduction strategies, ranging from simultaneous hermaphroditism to complete dioecy (separate sexes). The evolutionary pathway from hermaphroditism to dioecy involves the spread of "sterility alleles" that eliminate female or male reproductive functions, producing unisexual individuals. Classical theory predicts that evolutionary transitions to dioecy are feasible when female and male sex functions genetically trade-off with one another (allocation to sex functions is "sexually antagonistic"), and rates of self-fertilization and inbreeding depression are high within the ancestral hermaphrodite population. We show that physical linkage between sterility alleles and loci under sexually antagonistic selection significantly alters these classical predictions. We identify three specific consequences of linkage for the evolution of dimorphic sexual systems. First, linkage broadens conditions for the invasion of unisexual sterility alleles, facilitating transitions to sexual systems that are intermediate between hermaphroditism and dioecy (gyno- and androdioecious populations). Second, linkage elevates the equilibrium frequencies of unisexual individuals within gyno- and androdioecious populations, which promotes subsequent transitions to full dioecy. Third, linkage dampens the role of inbreeding during transitions to gyno- and androdioecy; such transitions become feasible in outbred populations. We discuss implications of these results for the evolution of reproductive systems and sex chromosomes.
\newpage{}


%%%%%%%%%%%%%%%%%%%%%%%%
\section*{Introduction}
%%%%%%%%%%%%%%%%%%%%%%%%
Multicellular Eukaryotes employ a diverse array of strategies for sexual reproduction \citep{Bachtrog2014}. At one end of this spectrum, simultaneous hermaphrodites express both female and male reproductive structures (sex functions) and may reproduce by outcrossing, self-fertilization, or a combination of both (mixed mating). At the other end of the spectrum, dioecious (or gonochoristic) species are comprised of physically discrete females and males -- typically at similar population frequencies, and often distinguishable by their strikingly different morphologies, behaviors, physiologies and life-histories \citep{Andersson1994}. Within Eukaryotes, flowering plants are particularly diverse in the range of reproductive tactics that they use. The range of plant sexual systems includes hermaphroditism, dioecy, and nearly every possible state in between \citep{Darwin1877, Westergaard1958, Bawa1980, SakaiWeller1999, Bachtrog2014}. These intermediate states include gynodioecy (hermaphrodites and females), androdioecy (hermaphrodites and males), and sub-dioecy (mixtures of hermaphrodites and unisexual males and females; e.g., citealt{SakaiWeller1999, Renner2014}). 

In plants, dioecy is thought to have evolved repeatedly and independently from ancestral populations of hermaphrodites -- an observation that has inspired much research into the evolutionary mechanisms that might drive these transitions \citep{Westergaard1958, SakaiWeller1999, Charlesworth2006, Bachtrog2014, Renner2014, GoldbergOtto2017, KaferPannell2017}. Both theory and data suggest that transitions from hermaphroditism to dioecy are likely to follow a two-step evolutionary pathway. In the first step, the spread of unisexual females or males within an ancestral population of hermaphrodites gives rise to a mixed-mating population of hermaphrodites and unisexuals of a single sex (i.e., gyno- or androdioecious populations of hermaphrodites and either females or males, respectively \citep{Charlesworth1978a, Charlesworth1978b}). Unisexuality can arise through the invasion of a single "sterility allele" that causes complete loss of one of the two ancestral hermaphrodite sex-functions, or through the invasion of combinations of alleles with cumulative gender-modifying effects \citep{Charlesworth1978a, Charlesworth1978b, Charlesworth1999}. During the second step, the spread of the second type of unisexual completes the transition to full dioecy. The second step may involve the spread of a female-sterility allele in a gynodioecious population, or the spread of a male-sterile allele in an androdioecious population \citep{Westergaard1958, Charlesworth1978a, Charlesworth1978b, Charlesworth2006, Charlesworth2009, KaferPannell2017}. Current data on the evolution of plant sex determination systems are largely consistent with this two-step scenario \citep{Westergaard1958, Charlesworth2002, Charlesworth2006, Renner2014, Ashman2015}.

Although the two-step evolutionary path to dioecy is feasible, such transitions are far from inevitable. This raises the question: what are the population genetic conditions that promote or constrain transitions to dioecy through this pathway? Classical theory has shown that transitions to dioecy are most likely to occur when two conditions are met \citep{Lewis1941, Lloyd1975, Lloyd1976, Charlesworth1978a}. First, female and male sex-functions in hermaphrodites must genetically trade-off against one another (e.g., through allocation trade-offs). Second, modest-to-high rates of self-fertilization and inbreeding depression in the ancestral hermaphrodite population promote the spread of unisexual sterility alleles that initiate the transition to dioecy. The theory also predicts that the first sterility allele to spread is more likely to cause male than female sterility, so that gynodioecy is the most likely path from hermaphroditism to dioecy (\citealt{Lloyd1975, Charlesworth1978a, KaferPannell2017}; see \citealt{Charlesworth1999, Charlesworth2006} for comprehensive reviews of relevant theory). Finally, the second sterility mutation to invade the population, which completes the transition to dioecy, is predicted to establish in tight linkage with the locus segregating for the first sterility allele (the "linkage constraint" of \citealt{Charlesworth1978a}). The establishment of complimentary sterility mutations, under tight linkage and in repulsion phase, sets the stage for the subsequent evolution of heteromorphic sex chromosomes (e.g., a gene-rich X and degenerate Y \citealt{Rice1987, Bachtrog2006, Charlesworth2002, Qiuetal2013}). 

Genetically-based trade-offs between sex-functions are special cases of a much broader phenomenon of sexual antagonism (hereafter, "SA" selection, or "intralocus" sexual conflict; \citealt{Rice1992,BondurianskyChenoweth2009, Mank2017}), where alleles that benefit one sex are deleterious for the other. In a hermaphrodite population, an SA allele could, for example, increase ovule production yet decrease pollen produciton or viability \citep{JordanConnallon2014, Olito2016}. During the transition to dioecy, male-sterility alleles that improve female sex-function, and female-sterility alleles that improve male sex-function, are essentially SA alleles, with SA effectspotentially arising via reproductive compensation \citep{Lewis1941, Lloyd1975, Charlesworth1978a}. The evolutionary dynamics of unisexual sterility alleles can therefore be understood through the broader perspective of SA selection theory. For example, SA alleles spread within a population when the benefit to one sex offsets the cost to the other; the critical balance between these benefits and costs is determined by the level of inbreeding of the population \citep{Kidwell1977, JordanConnallon2014}. Similarly, unisexual sterility alleles -- which initiate the transition to dioecy -- can spread when the loss of fitness from unisexual sterility is offset by fitness gains through the remaining sex-function, and once again, the balance between these factors is mediated by inbreeding \citep{Charlesworth1978a}. 

Recent multi-locus population genetics theory has shown that a locus under SA selection can substantially impact the evolutionary dynamics at linked loci that segregate for deleterious mutations, meiotic drive elements, or SA alleles \citep{ConnallonClark2010, Patten2010, UbedaPatten2010, ConnallonJordan2016, Olito2016}. For example, strong SA selection at one locus facilitates the spread of physically linked SA alleles that would otherwise be removed from a population (i.e., in the absence of such linkage). This consequence of linkage to an SA locus applies in both dioecious populations \citep{Patten2010,ConnallonClark2010} and hermaphrodite populations with fixed rates of self-fertilization \citep{Olito2016}. These multi-locus models suggest a mechanism that might promote transitions between hermaphrodite and gyno- or androdioecious sexual systems. Linkage of unisexual sterility alleles to an SA locus could, in principle, promote the spread of unisexuals within hermaphrodite populations, and relax evolutionary constraints that would otherwise hinder evolutionary transitions between reproductive systems.

Classical theories for the evolution of dioecy model the initiating step of the transition (see above) using the single-locus invasion conditions for unisexual sterility alleles in a hermaphrodite population \citep{Charlesworth1978a}. Here, we extend this theory by considering the role of linkage between SA loci during transitions to gyno- and androdioecy. We focus on three key questions in our analysis. First, does linkage to an SA locus facilitate the invasion of sterility alleles, for example, by promoting selection of multilocus haplotypes that combine female-beneficial SA alleles with male-sterility alleles, or male-beneficial SA alleles with female-sterility alleles? Second, how does self-fertilization and inbreeding depression in the ancestral hermaphrodite population impact transitions to dioecy in the extended theory? Third, how does linkage to an SA locus impact the equilibrium frequencies of unisexuals in derived gyno- and androdoiecious populations? Our results show that linkage to an SA locus significantly expands the parameter conditions for the invasion of unisexuals, dampens the constraints associated with inbreeding on the invasion of sterility alleles, and elevates the equilibrium frequencies of unisexuals in andro- and gynodioecious populations. Overall, our results indicate that multi-locus consequences of SA selection facilitate the evolution of dimorphic sexual systems, and suggest a hitherto unrecognized role for SA genetic variation during the initial stages of early sex-chromosome evolution.


%%%%%%%%%%%%%%%%%%%%%%%%
\section*{Models} \label{sec:Models}
%%%%%%%%%%%%%%%%%%%%%%%%

Following previous theory, we model evolutionary transitions to gyno- and androdioecy from an ancestral population of hermaphrodites. Each transition occurs via the invasion of unisexual sterility alleles that cause complete male or female sterility (for gyno- and androdioecy, respectively; \citep{Charlesworth1978a}. In our models, we separately consider the invasion of completely dominant sterility alleles, and completely recessive sterility alleles, during transitions to gyno- and androdioecy (leading to a total of four genetic models). These scenarios mirror classical theory for the invasion of unisexuals in hermaphrodite populations, providing clear points of contrast. Immediately below, we describe in full the simplest model: invasion of a dominant, male-sterility mutation during the evolution of gynodioecy. We then briefly highlight essential differences arising in each of the other three models. Additional details, including full recursions, can be found in Appendix A of the Online Supporting Information.

%%%%%%%%%%%%%%%%%%%%%%%%
\subsection*{Gynodioecy}

Consider a genetic system involving two diallelic autosomal loci, $\mathbf{A}$ (with alleles $A$, $a$) and $\mathbf{M}$ (with alleles $M_1$, $M_2$), that recombine at rate $r$ per meiosis. The $\mathbf{A}$ locus is under sexually antagonistic selection, with the $A$ allele female-beneficial/male-deleterious, and the $a$ allele male-beneficial/female-deleterious. At the $\mathbf{M}$ locus, the $M_1$ allele has a relative fitness of $1$ in both sexes, while the $M_2$ allele causes sterility through the male sex-function, and is completely dominant to the $M_1$ allele (we subsequently consider the case where $M_2$ is recessive to $M_1$). Female unisexuals carrying the $M_2$ allele potentially re-allocate resources towards ovule production that would otherwise have been used for pollen production (\citealt{Lloyd1975,Lloyd1976,Charlesworth1978a}), with $k$ representing the proportional increase in ovule production by unisexual females relative to hermaphrodites. The population is initially hermaphroditic (fixed for $M_1$); evolution of gynodioecy requires the invasion of the $M_2$ allele into the population.

Among offspring of hermaphrodites, a fixed proportion $C$ are produced by self-fertilization, and the remainder are produced by outcrossing. Unisexual females, which carry at least one $M_2$ allele, cannot self-fertilize. Individuals produced by self-fertilization suffer from inbreeding depression, with $\delta$ representing the decreased probability of survival of self-fertilized zygotes relative to zygotes produced by outcrossing. Generations are discrete, and the life-cycle proceeds as follows: birth $\rightarrow$ differential survival due to inbreeding depression $\rightarrow$ selection on reproductive success through male and female sex-functions $\rightarrow$ random fertilization.

Let $x_i$ and $y_i$ denote the frequencies of the four possible haplotypes $[AM_1,AM_2,aM_1,aM_2]$ in female and male gametes, respectively. An individual carrying haplotypes $i$ and $j$ has fitness of $w^f_{ij}$ through the female sex-function, and $w^m_{ij}$ through the male sex-function; fitness per sex-function is multiplicative between the $\mathbf{A}$ and $\mathbf{M}$ loci (Table~\ref{tab:fitness}). We assume there are no parent-of-origin effects on fitness.

To track evolutionary changes in genotype frequencies, we partitioned genotypes by whether they were produced by self-fertilization or by outcrossing. The approach yields a system of $20$ general recursion equations ($10$ genotypes $\times$ two modes of reproduction; see Appendix A). Although more compact approaches have been used for generating recursions in previous models for the evolution of self-fertilization and dioecy (e.g., \citealt{Charlesworth1978a, Charlesworth2010}), we used the following, expanded expressions as a means of clarifying the transmission pathway and assumptions that underlie each of our models. As shown below, our models reproduce the classical results in special cases where there is only selection at loci segregating for sterility alleles.

When $M_2$ is completely dominant to $M_1$, the system of recursions simplifies considerably. Let $F_{ij}$ represent the zygotic frequency the genotype produced by outcrossing and carrying haplotypes $i$ and $j$; $G_{ij}$ is the frequency of the same genotype among zygotes formed via self-fertilization. The recursions for outcrossed zygotes in the next generation are:
\begin{linenomath}\begin{align*} \label{eq:FprGyn}
    F'_{11} &= (1 - S) (x_1 y_1)  \\
    F'_{12} &= (1 - S) (x_2 y_1)  \\
    F'_{13} &= (1 - S) (x_1 y_3 + x_3 y_1)  \\
    F'_{14} &= (1 - S) (x_4 y_1)  \\
    F'_{22} &= 0 \\
    F'_{23} &= (1 - S) (x_2 y_3)  \\
    F'_{24} &= 0 \\
    F'_{33} &= (1 - S) (x_3 y_3)  \\
    F'_{34} &= (1 - S) (x_4 y_3)  \\
    F'_{44} &= 0, \numberthis
\end{align*}\end{linenomath}

\noindent where $x_{i}$ and $y_{i}$ are functions describing the haplotype frequencies among ovules and pollen respectively, and $S$ is the proportion of all ovules produced by the population that are self-fertilized (see Appendix A for details). Note that $y_2=y_4=0$ because these male gametic haplotypes cannot be produced when $M_2$ is a dominant male sterility allele. Among zygotes formed by self-fertilization, the genotypic frequencies in the next generation are: 
\begin{linenomath}\begin{align*} \label{eq:GprGyn}
    G'_{11} &= S (o^S_{11} + o^S_{13}/4) \\
    G'_{13} &= S (o^S_{13}/2) \\
    G'_{33} &= S (o^S_{33} + o^S_{13}/4), \numberthis
\end{align*} \end{linenomath}

\noindent where $o^S_{ij}$ are functions describing the proportional contribution of each genotype to self-fertilized ovules (see Appendix A). All $G'_{ij} = 0$ where $ij \neq [11,13,33]$. 

The basic form of the recursions does not change when $M_2$ is recessive, but there are two notable differences. First, because only $M_2M_2$ homozygotes are unisexual females, none of the recursions reduce to zero. Second, the recombination rate parameter ($r$) enters into the nonzero $G'_{ij}$ recursions, as well as the functions for $x_i$ and $y_i$; this was not the case for the gynodioecy model with dominant $M_2$ allele (see Appendix A).


%%%%%%%%%%%%%%%%%%%%%%%%
\subsection*{Androdioecy}

Androdioecy evolves when an $M_2$ allele causing female-sterility (e.g., $M_2$ carriers do not produce viable ovules) invades the hermaphrodite population. In contrast to the models of gynodioecy, described above, the reproductive compensation term, $k$, now describes the increase in pollen production by males relative to hermaphrodites. For the case of a dominant $M_2$ allele (i.e., $M_2$ carriers are male), the fitness expressions, $w^f_{ij}$ and $w^m_{ij}$ resemble those described in Table~\ref{tab:fitness}, except the fitness effects of the $\mathbf{M}$ locus apply to the female rather than the male sex-function. 

Recursions for the androdioecy models are similar to the gynodioecy recursions, with a few key differences. When $M_2$ is dominant, the recombination rate drops out of the expressions for haplotype frequencies in ovules ($x_i$; $x_2=x_4=0$, as in the gynodioecy model with dominant $M_2$ allele), whereas haplotype frequencies in pollen are partly dependent on recombination (functions for $y_i$ include $r$ terms). The genotypic frequencies in the next generation among zygotes formed by outcross fertilization, $F'_{ij}$, are otherwise identical to Eq(\ref{eq:FprGyn}), except that heterozygotes at the $\mathbf{M}$ locus do not produce ovules. The form of the $G'_{ij}$ recursions remains unchanged from Eq(\ref{eq:GprGyn}) (see Appendix A).

When the $M_2$ (female-sterility) allele is recessive, only $M_2M_2$ homozygotes are unisexual males. The form of the recursions is similar to the case of gynodioecy with a recessive $M_2$ allele, but with the $\mathbf{M}$ locus now affecting female sterility. Expressions for the ovule and pollen haplotype frequencies, $x_i$ and $y_i$, are functions that partially depend on the recombination rate (see Appendix A).


%%%%%%%%%%%%%%%%%%%%%%%%
\subsection*{Analyses} \label{subsec:analyses}

Our analyses address three questions: How does linkage between unisexual sterility alleles and SA loci affect (1) conditions for the evolutionary invasion of unisexual sterility alleles, (2) frequencies of unisexual individuals in andro- and gynodioecious populations, and (3) the evolutionary dynamics and conditions for polymorphism at SA loci? Because they are analytically tractable for linear stability analyses, we used the models of completely dominant unisexual sterility alleles to address questions (1) and (3). However, the evolutionary dynamics of recessive sterility alleles are of particular biological interest because recessive sterility alleles are probably more common than dominant ones (\citealt{Charlesworth1978a}). We therefore used numerical simulation of the models of recessive sterility mutations to address question (2). 

Following previous models of SA selection (e.g., \citealp{Kidwell1977, Prout2000, JordanConnallon2014}), we focus our analyses on the simplest scenario of allelic expression at the SA locus (locus $\mathbf{A}$): we assume that SA alleles have additive fitness effects on the sex-functions ($h_m = h_f = 1/2$). Additional results exploring the effects of dominance reversals at the $\mathbf{A}$ locus ($h_m, h_f < 1/2$) are presented in Appendix C of the Online Supporting Information. These additional results provide results that are qualitatively similar to theoretical predictions for the additive SA scenario. 

We first examine the case where both loci are initially monomorphic ($M_1$ is fixed at the $\mathbf{M}$ locus; $A$ or $a$ is fixed at locus $\mathbf{A}$ locus). To identify conditions for the invasion of derived alleles at each locus (individually or in combination), we carried out a linear stability analysis of each monomorphic condition. These initial conditions correspond to $F_{11} = (1 - C) [AAM_1M_1] = 1$ and $G_{11} = C [AAM_1M_1] = 1$ when $A$ is initially fixed, and to $F_{33} = (1 - C) [aaM_1M_1] = 1$ and $G_{33} = C [aaM_1M_1] = 1$ when $a$ is initially fixed, with terms in square brackets representing starting genotype frequencies. An equilibrium is unstable, and a derived haplotype will invade the population, when the leading eigenvalue of the Jacobian matrix of the system of recursions is greater than one ($\lambda_L > 1$) \citep{OttoDay2007}. For the models involving dominant unisexual sterility alleles, each analysis yields three analytically tractable candidate leading eigenvalues. The first two candidates describe invasion conditions for the derived allele at each locus, independent of the other locus ($\lambda_\mathbf{A}$ and $\lambda_\mathbf{M}$). The third candidate eigenvalue describes the invasion condition for a haplotype carrying both derived alleles ($\lambda_\mathbf{AM}$). We present analytic results for stability analyses of models of dominant sterility alleles.

We subsequently explore how segregating SA variation, maintained under balancing selection, influences the invasion of derived unisexual alleles ($M_2$) into a population of hermaphrodites. For this scenario, we evaluated whether the $M_2$ allele can invade a population at a deterministic polymorphic equilibrium at the $\mathbf{A}$ locus; $M_1$ is initially fixed at the $\mathbf{M}$ locus. Under obligate outcrossing and additive effects of SA alleles, we used exact analytical expressions provided in Kidwell et al. (1977) for the initial conditions at the polymorphic locus ($\hat{p}_f,\hat{p}_m$; \citealt{Kidwell1977}). Exact analytical expressions for a single-locus sexually antagonistic polymorphism cannot be obtained for cases of partial selfing and/or nonadditive effects of SA alleles. For these conditions, single-locus equilibrium frequencies can be approximated under sufficiently weak selection ($s_m,s_f \ll 1$); we use previously published approximations for polymorphic SA equilibrium in our stability analyses for partial selfing cases (see \citealt{JordanConnallon2014,ConnallonJordan2016}; see Appendix B in the Online Supporting Information  for the full list of expressions for single-locus SA polymorphism). The approximations compare well against exact numerical results when selection is moderate to weak ($s_f,s_m \leq 0.5$; \citealt{JordanConnallon2014, ConnallonJordan2016, Olito2016}). To identify conditions under which a dominant $M_2$ allele invades the population, we numerically evaluated whether $\lambda_{\mathbf{M}} > 1$, which corresponds to the condition for invasion of a unisexual.

Linear stability analysis is inconclusive when sterility alleles are recessive. To compliment our analyses of the invasion of dominant unisexual ($M_2$) alleles, we carried out deterministic simulations of the exact recursions to determine the full equilibrium haplotype frequencies for the models of recessive sterility alleles. Initial genotypic frequencies in these simulations are the same as described above. To assess the impact of linkage on the polymorphic frequencies of unisexuals and hermaphrodites, we compared equilbrium frequencies from our models with the corresponding single-locus equilibria for unisexuals (denoted $\hat{Z}$) provided by \citet{Charlesworth1978a}. For these comparisons, we focus on relatively tight linkage between $\mathbf{A}$ and $\mathbf{M}$ ($r\leq 0.1$), which represents the parameter range where our models and prior single-locus models differ the most; predictions from one- and two-locus models converge under loose linkage (high $r$). 

In single-locus models of gynodioecy and androdioecy, the equilibrium frequency of unisexuals is determined by the reproductive compensation term, $k$, and the compound parameter $C \delta$ \citep{Charlesworth1978a}, where $C$ and $delta$ refer to the rate of selfing and the severity of inbreeding depression, respectively. If inbreeding depression is caused primarily by recessive deleterious mutations, as current data suggest \citep{Charlesworth2009}, $C$ and $\delta$ should covary negatively as increased selfing purges deleterious recessives. For simplicity, we accounted for such negative covariance in our simulations by constraining inbreeding depression to follow a linear declining function of the selfing rate: $\delta = \delta^\ast(1 - C/2)$, where $\delta^\ast$ represents the hypothetical severity of inbreeding depression for an outcrossing population. We set $\delta^\ast = 0.8$ for all simulations, which resulted in levels of inbreeding depression that are consistent with empirical data (e.g., Fig. 2 in \citealt{HusbandSchemske1996}). We emphasize that our intention is not to formally model the evolution of inbreeding depression, but to capture effects of purging of deleterious recessives in partially selfing populations, and thereby explore biologically meaningful parameter space for the evolution of gyno- and androdioecy. In fact, a variety of detailed genetic models give rise to a nonlinear relation between inbreeding depression and selfing (e.g., \citealt{OhtaCockerham1974, LandeSchemske1985, Charlesworth1985, Garcia-Dorado2017, LandePorcher2017}). As we show in Appendix E in the Online Supporting Information, such non-linear expressions for $\delta$ as a function of the selfing rate yield results that are similar to those using the linear expression described above. 

To account for the fact that $C$ and $\delta$ also influence the maintenance of SA polymorphism \citep{JordanConnallon2014,Olito2016}, we ran simulations using values of $s_f$ and $s_m$ that correspond to single-locus equilibrium frequencies of $1/2$ for the two SA alleles ($p = [A]$, $q = [a]$; $p=q=1/2$). Thus, our simulations explore the invasion of the male- or female-sterility allele, $M_2$, into hermaphroditic populations at polymorphic equilibrium with equal frequencies of male- and female-beneficial SA alleles at the $\mathbf{A}$ locus.


\subsection*{Data availability}
A full development of all models can be found in Appendix A of the Online Supporting Information. Code necessary to reproduce the simulations is available at \url{https://github.com/colin-olito/dioecySA}.

%%%%%%%%%%%%%%%%%%%%%%%%
\section*{Results}
%%%%%%%%%%%%%%%%%%%%%%%%

%%%%%%%%%%%%%%%%%%%%%%%%
\subsection*{Invasion into monomorphic populations}

%%%%%%%%%%%%%%%%%%%%%%%%
\subsubsection*{Gynodioecy:} We begin with the simplest scenario for the evolution of gynodioecy: invasion of a dominant male-sterility allele, $M_2$, into a hermaphrodite population that is initially monomorphic for the SA locus (fixed for the $AAM_1M_1$ or the $aaM_1M_1$ genotype).  

Three conditions favour the invasion of a derived haplotype, each corresponding to evolutionary instability of the ancestral equilibrium (instability occurs when at least one of the three candidate eigenvalues, $\lambda_{\mathbf{A}}$, $\lambda_{\mathbf{M}}$ and $\lambda_{\mathbf{AM}}$, exceeds one). First, the derived SA allele can invade when $\lambda_{\mathbf{A}} > 1$, which corresponds to the single-locus invasion criteria for SA alleles, as identified by previous theory \citep{Kidwell1977, JordanConnallon2014, Olito2016}. This single-locus invasion condition applies broadly to cases of additive and non-additive effects of SA alleles. 

Second, a male-sterile allele can invade when $\lambda_{\mathbf{M}} > 1$, which corresponds to the classic single-locus criterion for the evolution of gynodioecy:

\begin{equation}\label{eq:1LocGyn}
	k > 1 - 2 C \delta.
\end{equation}

\noindent (see Eq(4) of \citealt{Charlesworth1978a}), where $k$ is the increased ovule production in females relative to hermaphrodites, and $\delta$ represents the strength of inbreeding depression in self-fertilized relative to outcrossed individuals. For convenience, we define $\hat{k}$ as the threshold level of reproductive compensation for invasion of a unisexual in the single-locus model; $\hat{k}$ is equal to the right-hand side of Eq(\ref{eq:1LocGyn}), and $k > \hat{k}$ is the single-locus condition for invasion.

Third, a mutant haplotype bearing both the male-sterile allele and the derived allele at the SA locus can invade when $\lambda_{\mathbf{AM}} > 1$. This condition for invasion takes the same basic form as Eq(\ref{eq:1LocGyn}), in which the minimum value of $k$ that is required for invasion is a decreasing function of $C \delta$. When the population is initially fixed for the female-beneficial SA allele ($A$), the haplotype invasion condition is more restrictive (requiring larger $k$) than the condition in Eq(\ref{eq:1LocGyn}). In contrast, when the population is initially fixed for the male-beneficial allele ($a$), invasion of the derived haplotype becomes more permissive when recombination between loci is sufficiently low relative to the strength of selection at the SA locus. Specifically, the condition for invasion of the $AM_2$ haplotype is more permissive than Eq(\ref{eq:1LocGyn}) when: 

\begin{equation}\label{eq:2LocGyn}
	\frac{2 r}{1 + r} < s_f.
\end{equation}

\noindent Eq(\ref{eq:2LocGyn}) shows that linkage to an SA locus promotes the invasion of unisexual females into a hermaphrodite population, relative to the single-locus condition for the evolution of gynodioecy (Eq(\ref{eq:1LocGyn})). With relatively weak SA selection ($s_f = 0.1$), conditions for the invasion of unisexual females expand when linkage to the SA locus is modest ($r \leq 0.05$); strong SA selection further relaxes the conditions of linkage that promote the evolution of gynodioecy. 

Linkage with a male-sterility allele also expands conditions for the invasion of female-beneficial SA alleles. Under obligate outcrossing and additive fitness effects, the single-locus invasion condition for a female-beneficial allele is $s_f > s_m / (1+s_m)$ (\citealt{Kidwell1977}, corresponding to $\lambda_{\mathbf{A}} > 1$ in our model). The conditions for invasion of a derived haplotype linking a female-benefit allele with a male-sterility allele ($AM_2$; corresponding to $\lambda_{\mathbf{AM}} > 1$ in our model) are more permissive, relative to the above single-locus predictions, when:

\begin{equation}\label{eq:2LocGynSA}
	r < \frac{s_m}{2 + s_m}.
\end{equation}

\noindent Under partial selfing, invasion of female-beneficial SA alleles is more permissive than in the single-locus case when:

\begin{equation}\label{eq:2LocGynSApartSelf}
	r < \frac{s_m (1 - C)}{2 + s_m +C (2 - s_m - 4 \delta)}.
\end{equation}

\noindent Eq(\ref{eq:2LocGynSApartSelf}) shows that selfing dampens the linkage-induced expansion of parameter space for invasion of female-beneficial SA alleles. Nevertheless, tight linkage can still aid invasion of female-beneficial alleles provided there is some degree of outcrossing among hermaphrodites. 


%%%%%%%%%%%%%%%%%%%%%%%%
\subsubsection*{Androdioecy:} We identified conditions for the evolution of androdioecy by analyzing stability of a population initially fixed for the $AAM_1M_1$ or the $aaM_1M_1$ genotype; as in the gynodioecy results presented above, we follow the evolution of a dominant sterility allele ($M_2$, causing female sterility) and SA alleles with additive effects on the sex functions. As before, three conditions allow for invasion of a derived haplotype, each defined by the three candidate leading eigenvalues for the equilibrium. Analysis of the first eigenvalue ($\lambda_{\mathbf{A}}$) recovers the general, single-locus invasion criterion for a derived SA allele (see above). Analysis of the second eigenvalue, $\lambda_{\mathbf{M}}$ likewise yields the familiar single-locus invasion criterion for unisexual males into a population of hermaphrodites:

\begin{equation}\label{eq:1LocAndro}
	k > \frac{1 + C (1 - 2 \delta)}{(1 - C)}.
\end{equation}

\noindent (see Eq(8) of \citealt{Charlesworth1978a}). As before, we define $\hat{k}$ as the threshold level of reproductive compensation for unisexual invasion in the single-locus model, with $\hat{k}$ equal to the right-hand side of Eq(\ref{eq:1LocGyn}) (for the gynodioecy model) or Eq(\ref{eq:1LocAndro}) (for the androdioecy model), as appropriate.

Haplotypes bearing derived alleles at both loci can invade when $\lambda_{\mathbf{AM}} > 1$, where invasion conditions are once again dependent upon the SA genotype that is initially fixed in the ancestral hermaphrodite population. Mirroring the results of the gynodioecy model, the conditions satisfying $\lambda_{\mathbf{AM}} > 1$ are always more restrictive than the classical invasion condition (Eq(\ref{eq:1LocAndro})) when the population is initially fixed for the male-beneficial allele ($a$). Conditions for invasion are more permissive for populations initially fixed for the female-beneficial allele ($A$). In fact, replacing $s_f$ with $s_m$ in Eq(\ref{eq:2LocGyn}) gives the condition under which the invasion of males is more permissive than the single locus criterion, Eq(\ref{eq:1LocAndro}). Thus, although conditions for the evolution of androdioecy are more restrictive than for gynodioecy (cf. Eq(\ref{eq:1LocAndro}), Eq(\ref{eq:1LocGyn})), the effect of linkage on the evolution of unisexual sterility alleles is similar for the two cases. 

Linkage to a female-sterility allele also facilitates the invasion of male-beneficial SA alleles. Under obligate outcrossing, Eq(\ref{eq:2LocGynSA}), with $s_m$ substituted for $s_f$, provides the condition for invasion of a male-beneficial allele that is linked to a female-sterility allele at the $\mathbf{M}$ locus. However, the parallels between gyno- and androdioecy break down somewhat under partial selfing. In the gynodioecy model, recall that selfing dampens the expansion of parameter space for the invasion of female-beneficial SA alleles that are linked to male-sterility alleles. For the andro-dioecy case, self-fertilization further expands conditions for invasion of male-beneficial SA alleles that are linked to a sterility locus. Under arbitrary selfing in hermaphrodites, linkage to a female-sterility locus expands the invasion condition of male-beneficial SA alleles when:

\begin{equation}\label{eq:2LocAndroSApartSelf}
	r < \frac{s_f + s_f C (1 - 2 \delta)}{2 + s_f - 2 C + s_f C (1 - 2 \delta)}.
\end{equation}

\noindent The right hand side of Eq(\ref{eq:2LocAndroSApartSelf}) increases with both $C$ and $s_f$ such that, for $C > 1/3$, there is always some expansion of the parameter space where male-beneficial alleles can invade (i.e., relative to the single-locus expectation), even under free-recombination ($r = 1/2$). 



%%%%%%%%%%%%%%%%%%%%%%%%
\subsection*{Invasion of unisexuals into polymorphic populations}

Three factors determine the fate of new sterility mutations in populations that are initially polymorphic at $\mathbf{A}$: the degree of reproductive compensation associated with unisexual sterility alleles ($k$), the rate of recombination ($r$), and the selfing rate ($C$). Unisexuals are always able to invade populations with segregating SA variation when the classical single-locus conditions for unisexual invasion are satisfied (fig.~\ref{fig:PrInv}; light blue lines; the single-locus conditions are: Eq(\ref{eq:1LocGyn}) and Eq(\ref{eq:1LocAndro})). When the degree of reproductive compensation falls below the single-locus threshold for invasion, unisexuals can still invade provided there is some linkage between loci $\mathbf{A}$ and $\mathbf{M}$. For example, under obligate outcrossing and tight linkage ($r = 0$), unisexuals can invade across $\approx 69\%$ of the parameter conditions maintaining SA polymorphism (within the range $0 < s_f,s_m \leq 0.5$), even if they suffer a  $10\%$ reduction in gamete production relative to the single-locus invasion criterion for unisexual sterility alleles (the right hand side of Eq(\ref{eq:1LocGyn}); see fig.~\ref{fig:PrInv}A,D). With smaller reductions in gamete production relative to the single locus invasion criterion, unisexuals can invade across a greater fraction of parameter space, even when linkage is relatively weak (e.g., $\approx 38\%$ of parameter conditions maintaining SA polymorphism when $k$ is $0.95$ times the minimum level of reproductive compensation for single-locus invasion and $r = 0.2$). 

With a small amount of selfing and high inbreeding depression in hermaphrodites ($C = 1/4,~\delta = 0.8$), invasion conditions for unisexuals remain similar to the obligate outcrossing scenario, with similar ranges of parameter space allowing for the evolution of gynodioecy and androdioecy (fig.~\ref{fig:PrInv}B,E). However, under high selfing rates and low inbreeding depression ($C = 3/4,~\delta = 0.2$), transitions to androdioecy become much more permissive than transitions to gynodioecy. Most notably, tighter linkage is required for the invasion of male-sterility alleles (for gynodioecy) than for the spread of female-sterility alleles (for androdioecy). Yet even under complete linkage, conditions are more amenable for transitions to andro- than to gynodioecy (fig.~\ref{fig:PrInv}C,F). This contrast in the effect of selfing between the models of gynodioecy and androdioecy arises because partial selfing causes asymmetry in the evolutionary dynamics of SA alleles, in favour of female-benefit alleles (\citealt{ Charlesworth1978a, JordanConnallon2014}; see Appendix D, figs.~D1--D6 for examples of how the selfing rate and inbreeding depression influence SA invasion conditions). Analogous to previous results for linked SA loci \citep{Olito2016}, this female-bias in selection increases the scope for linkage to a male-beneficial SA allele to facilitate invasion of a female-sterility allele (resulting in androdioecy), but has the opposite effect for a male-sterility allele linked to a female-beneficial allele (resulting in gynodioecy).

In the models of gyno- and androdioecy, the loss of parameter space where unisexuals can invade is determined by the rate of recombination, the degree of reproductive compensation, and the relative strength of selection through each sex-function. With weaker linkage between $\mathbf{A}$ and $\mathbf{M}$, invasion of unisexuals requires that selection is stronger at the SA locus (larger values of $s_m$ and $s_f$), and also increasingly biased toward the gender of the invading unisexuals. This result is intuitive: for the derived haplotype to successfully invade, weaker linkage between the SA and sterility loci must be compensated for by stronger selection favouring the SA allele (the female-benefit allele, $A$, for gynodioecy, and the male-benefit allele, $a$, for androdioecy). Hence, with incomplete linkage the invasion of male-sterility alleles requires female-biased selection, while the invasion of female-sterility requires male-biased selection (see Appendix D, figs.~D1--D6 in the Online Supporting Information). 

%%%%%%%%%%%%%%%%%%%%%%%%
\subsection*{Equilibrium frequencies of unisexuals when sterility alleles are recessive}

Linkage with an SA locus elevates the equilibrium frequencies of unisexuals in the population relative to single-locus predictions. For example, for models involving recessive sterility alleles with reproductive compensation ($k$) exactly matching or exceeding the threshold for unisexual invasion in the single-locus model (i.e., $k \geq \hat{k}$; where $\hat{k}$ is equal to the right-hand side of Eq(\ref{eq:1LocGyn}) or Eq(\ref{eq:1LocAndro}) as appropriate), linkage can lead to equilibrium frequencies of unisexuals that far exceed predictions of the single-locus models provided in \citet{Charlesworth1978a} (fig.~\ref{fig:eqFreq2v1Loc}, greyscale lines). The effect is strongest in predominantly outcrossing populations with relatively tight linkage ($r < 0.05$), and weakens with increased selfing among hermaphrodites (recall that inbreeding depression also declines with selfing in our models; see \hyperref[sec:Models]{Models}). As the strength of linkage declines, unisexual equilibria in our models converge to predictions for single-locus models. Finally, female and male unisexuals reach similar equilibrium frequencies in the gyno- and androdioecy models, despite more restrictive single-locus conditions for the evolution of androdioecy (i.e., Eq.\ref{eq:1LocGyn} vs.~Eq.\ref{eq:1LocAndro}).

Of course, linkage to an SA locus also permits invasion of unisexual sterility alleles when reproductive compensation falls below the single-locus threshold for invasion ($k < \hat{k}$; fig.~\ref{fig:eqFreq}). In cases where invasion occurs with linkage, but not in the single-locus context, the combination of tight linkage and high outcrossing rates once again heighten the equilibrium frequencies of unisexuals (fig.~\ref{fig:eqFreq}A,C). Equilibrium unisexual frequencies ultimately drop to $0$ as the degree of linkage, outcrossing, and reproductive compensation declines (fig.~\ref{fig:eqFreq}B,D). In these cases, the frequencies of males in the androdioecy model are more sensitive to reductions in $k$ than for the equilibrium frequencies of females in the gynodioecy model. In models of recessive sterility alleles, with incomplete linkage and low reproductive compensation, \hl{equilibrium frequencies of unisexuals are highest for populations with intermediate selfing rates}. These intermediate peaks reflect the balance between two key factors affecting the evolution of recessive unisexuals. First, selfing increases the expression, in $M_2 M_2$ homozygotes, of rare unisexual sterility alleles and lowers inbreeding depression, which promotes the spread of unisexuals in the population. Second, because unisexuals can only reproduce by outcrossing, high selfing rates ultimately limit the reproductive contributions of unisexuals to the next generation, which hinders the spread of unisexual alleles. 





%%%%%%%%%%%%%%%%%%%%%%%%
\section*{Discussion}
%%%%%%%%%%%%%%%%%%%%%%%%

The first theoretical insight provided by our models is that linkage among SA loci facilitates the invasion of unisexual sterility alleles, and elevates the equilibrium frequencies of unisexuals relative to single-locus predictions. The second is that when linkage is relatively tight, this effect is greatest for predominantly outcrossing populations -- suggesting that the ancestral hermaphrodite selfing rate may play a different role in the evolution of separate sexes than predicted by the classic models of \citet{Charlesworth1978a}. Overall, these predictions suggest that the unisexual sterility alleles driving the evolution of dimorphic sexual systems are likely to evolve in genomic regions harboring polymorphic SA loci. When this occurs, elevated frequencies of unisexuals in the resulting gyno- and androdioecious populations are more likely to evolve than previously predicted, facilitating subsequent transitions to separate sexes \citep{Charlesworth1978a}. Below, we discuss the implications of our findings, and suggest empirical tests of our predictions, in three main contexts: SA polymorphism and the evolution of gyno- and androdioecy, hermaphrodite mating systems and the evolution of dioecy, and the population genetic basis of the evolution of separate sexes.

%%%%%%%%%%%%%%%%%%%%%%%%
\subsection*{SA Polymorphism and the evolution of dimorphic sexual systems}

Although dioecy is relatively uncommon among angiosperms (represented in $\approx 7\%$ of genera), the fantastic diversity and repeated evolution of dimorphic sexual systems in flowering plants begs a genetical explanation \citep{Renner2014,KaferPannell2017}. Several evolutionary pathways, and a variety of genetic mechanisms, lead from hermaphroditism to separate sexes, but all ultimately involve the evolutionary invasion of at least two unisexual sterility alleles \citep{Charlesworth1978a,Charlesworth1978b,Renner2014,Ashman2015}. When accompanied by allocation trade-offs between sex-functions, unisexual sterility alleles are an important class of SA allele, yet previous theory has not considered the potential influence of standing SA genetic variation, or genetic architecture, on their invasion and establishment in hermaphroditic populations. Our theoretical results suggest that physical linkage among SA loci can strongly influence the evolution of gyno- or androdioecy from hermaphroditism, expanding the conditions under which dimorphic sexual systems, and ultimately separate sexes, are predicted to evolve. 

Classical theoretical predictions, based on single-locus models, suggest that conditions for the evolution of dioecy may be quite stringent \citep{Lloyd1975,Lloyd1976,Charlesworth1978a,KaferPannell2017}. For unisexual sterility alleles to invade, the resulting unisexuals must adequately compensate for the loss of a sex function through increased gamete production. On the other hand, we find that the conditions for the spread of unisexual sterility alleles can be quite permissive -- especially if the sterility mutation arises on a haplotype bearing a complimentary SA allele (e.g., one that is beneficial for the same sex-function as the invading unisexuals). In this case, invasion requires only modest linkage for biologically plausible selection coefficients (e.g., $s_f,s_m \leq 0.1$). When linkage does exist, the fitness effects of the complimentary SA allele help offset the loss of a sex function, and reduce the amount of reproductive compensation required by unisexuals to invade the population relative to single-locus predictions \citep{Charlesworth1978a}. These results are similar to other multilocus models of sex-specific selection and mutations, which show that predominantly female-harming mutations tend to accumulate on haplotypes carrying male-benefit alleles \citep{ConnallonJordan2016}. An important corollary of this result is that the conditions for the maintenance of SA polymorphism are also expanded by linkage to sex-specific sterility mutations. Hence, new unisexual sterility mutations underpinning dimorphic sexual systems are most likely to evolve in tight linkage with other SA loci, and should simultaneously promote the maintenance of SA polymorphism at linked loci. 

The amount of standing SA genetic variation should directly influence the potential for hermaphroditic populations to evolve dimorphic sexual systems. Although the invasion conditions for unisexual sterility alleles into populations without segregating SA variation are generally favourable, the waiting time for double mutants with the necessary haplotype (e.g., female-beneficial--male-sterile) to appear could be long \citep{WeinreichChao2005,ConnallonClark2010}. However, we find that the invasion conditions for unisexual sterility alleles are still quite permissive when there is standing SA genetic variation in hermaphroditic populations. Although it is not yet clear how much SA genetic variation for fitness is harbored by hermaphroditic species, three features of SA selection suggest that this is not an unlikely scenario, particularly in large populations. First, balancing selection is predicted to maintain SA polymorphism in partially selfing populations over a broad spectrum of parameter conditions, particularly when SA loci are linked \citep{Patten2010,JordanConnallon2014,Olito2016}. Second, net directional selection under SA is predicted to be small, even when fitness effects in each sex are large, resulting in relatively long persistence times of SA alleles \citep{ConnallonClark2012}. Long persistence times facilitate the formation of double-mutant haplotypes (e.g., \citealt{WeinreichChao2005}), and elevates the contribution of SA alleles to fitness variance compared to other classes of mutations (e.g., unconditionally beneficial or deleterious alleles; \citealt{ConnallonClark2012}). Third, although definitively indentifying SA loci is challenging, emerging data suggest that segregating SA allele frequencies can be non-trivial, even in partially selfing hermaphroditic populations \citep{Barson2015,LeeKelly2015}. Additional studies attempting to quantify the degree of SA genetic variation in hermaphroditic species would help clarify the potential for dimorphic sexual systems to evolve from hermaphroditism, especially if they were to target species exhibiting intraspecific variation in the degree or frequency of dimorphic sexual systems (e.g., many of the species reviewed in \citealt{SakaiWeller1999,Barrett2010,Renner2014}).


%%%%%%%%%%%%%%%%%%%%%%%%
\subsection*{Mating systems and the evolution of dioecy}

The interplay between hermaphrodite mating systems and reproductive compensation is a key factor influencing the evolution of separate sexes in flowering plants, especially via the gynodieocy pathway \citep{Darwin1877,Charlesworth1978a}. Previous theory predicts that the evolution of gynodioecy is driven by the combination of reproductive compensation by unisexuals, and inbreeding avoidance, and is therefore most likely to occur in partially selfing populations \citep{Lewis1942,Lloyd1975,Charlesworth1978a,KaferPannell2017}. This prediction follows directly from the structure of Eq(\ref{eq:1LocGyn}) where $\hat{k}$ depends entirely on the product of the selfing rate and inbreeding depression ($C \delta$). In contrast, the evolution of androdioecy is predicted to require significantly higher reproductive compensation, especially in partially selfing populations, because invading males must still compete with selfing hermaphrodites to fertilize ovules \citep{Charlesworth1978b,KaferPannell2017}. 

The population selfing rate plays a similarly critical role in our models. As noted above, linkage to an SA locus expands conditions for invasion and elevates the equilibrium frequencies of unisexual indiviuals. Yet, the effects of linkage are especially prominent under low rates of self-fertilization, and weaken with increased selfing. The evolution of higher equilibrium frequencies of females in the gynodioecy model is particularly important because this should facilitate the subsequent evolution of dioecy via invasion of partial female-sterility mutations \citep{Charlesworth1978a,Charlesworth1978b,Charlesworth1999,Charlesworth2006}. The major implication is that when male-sterility mutations do arise in linkage with other SA loci, single-locus theory may significantly underestimate the potential for gynodieocy, and subsequently dioecy, to evolve in predominantly outcrossing species. Moreover, outcrossing hermaphrodite populations are more likely to harbor segregating SA genetic variation than selfing ones \citep{JordanConnallon2014,Olito2016}. Hence, the ancestral mating systems in which linkage to a polymorphic SA locus is predicted to have the largest effect on evolutionary transitions to dioecy are also the ones that are most likely to harbor segregating SA variation. 

For the evolution of androdioecy, modest linkage to an SA locus broadens the conditions for invasion of female-sterility alleles, even for highly selfing populations. Linkage also increases equilibrium frequencies of unisexual males (given invasion), particularly in populations with low selfing-rates. Thus, when linkage exists between a female-steriliy allele and another SA locus, single-locus theory may underestimate the potential for the evolution of androdioecy, and subsequently, of dioecy. However, our models agree with previous theory that the conditions for invasion of female-sterility alleles in partially selfing populations are still quite stringent, requiring very high reproductive compensation (large $k$; \citealt{Charlesworth1978a}, Eq(\ref{eq:1LocAndro})). Overall, linkage with an SA locus will always faciliate the invasion of female-sterility mutations, but the evolution of androdioecy, and dioecy via the androdioecy pathway, is still expected to be quite rare relative to gynodioecy \citep{Charlesworth1978a,Charlesworth2006,KaferPannell2017,Renner2014}.

Despite the longstanding theoretical prediction of a strong correlation between the hermaphrodite mating system and dioecy, empirical evidence for this association remains equivocal \citep{Charlesworth1985,Charlesworth2006,Renner2014}. Our predictions suggest a partial explanation for this weak association. In species where unisexual sterility alleles arise in linkage with another SA locus, evolutionary transitions to dimorphic sexual systems should be elevated in taxa with predominantly outcrossing hermaphrodite ancestors. However, not all transitions to dioecy will have occurred by our proposed mechanism, and the net effect will be a weakened correlation between the ancestral hermaphrodite selfing rate, and dieocy. In light of our results, a re-examination of the evolutionary association between angiosperm mating and sexual systems using modern phylogenetic comparative methods would be most interesting, and would help identify species where our proposed mechanism for the evolution of gynodioecy is most likely to have played a role (e.g., species with dimorphic sexual systems that appear to have evolved from predominantly outcrossing ancestors). 


%%%%%%%%%%%%%%%%%%%%%%%%
\subsection*{The population genetic basis for the evolution of separate sexes}

Our results also have implications for the role of SA alleles during the early stages of sex chromosome evolution. The process of sex chromosome origin and differentiation into distinct X and Y (or Z and W) chromosomes is thought to proceed by the following series of steps. First, a sex-determining locus originates on an ordinary pair of autosomes. Second, SA genes linked to the sex-determining locus accumulate over time. Third, this linked SA variation drives the evolution of suppressed recombination between the neo sex-chromosomes, generally via chromosomal inversions. Fourth, the non-recombining sex chromosome degenerates (usually). Finally, mechanisms of dosage compensation evolve in response to the loss of coding regions on the degenerate sex chromosome (sometimes) \citep{Rice1987,Charlesworth2002,CharlesworthEtAl2005,Bachtrog2006,Qiuetal2013,Bachtrog2014}. 

A key assumption that the emergence of a sex-determining locus is the first step in this process traces back (again) to classic population genetic theory of unisexual sterility alleles. The key prediction is that after the invasion of a recessive male-sterility allele (yielding gynodioecy), the subsequent invasion of dominant gender modifiers leading to full dioecy require tight linkage to the first male-sterility locus (the linkage constraint of \citealt{Charlesworth1978a}). The sequential invasion of tightly linked gender modifiers leads to a tightly coupled pair of sex-determining loci -- effectively a single sex determining locus. Once this sex determining locus arises, further differentiation following the steps outlined above can result in heteromorphic sex chromosomes.

Our theoretical results suggest a previously unrecognized, or at least underappreciated, role for SA genetic variation in hermaphroditic taxa during these initial stages of sex-chromosome evolution. Our finding that linkage with a polymorphic SA locus facilitates the invasion of unisexual sterility alleles strongly suggests that the accumulation of SA genetic variation may often precede the origin of sex-determination loci. In other words, SA may represent the first (rather than an intermediate) step in the origin and evolution of new sex chromosome systems. This role of SA genetic variation during the origin of sex chromosomes is similar to prior theory regarding sex chromosome turnover \citep{vanDoornKirkpatrick2007,vanDoornKirkpatrick2010}, but is distinct in that it begins with SA variation in hermaphroditic populations, and deals with the invasion of unisexual sterility alleles rather than master sex-determining loci. In addition, our finding that linkage drives the evolution of elevated equilibrium frequencies of unisexuals should also facilitate the invasion of tightly linked gender modifiers yielding separate sexes \citep{Charlesworth1978a}. The resulting complex of linked SA and sterility loci effectively establishes a nascent sex-determining-region on the neo sex-chromosome pair. If pre-existing SA polymorhism at the linked SA locus survives the transition to dioecy, the process of accumulating SA variation prior to recombination suppression will have already begun -- setting the stage for the subsequent accumulation of linked SA genetic variation and recombination suppression leading to sex-chromosome differentiation \citep{Charlesworth1978a,Rice1987,Bachtrog2006,Qiuetal2013}. Given the crucial role of linkage in each of these steps, it seems plausible that the degree of SA genetic variation present when an initial male-sterility mutation arises may help explain the large variation in the rate and extent of sex-chromosome differentiation in angiosperms \citep{Charlesworth2002,Renner2014,Bachtrog2014}.


%%%%%%%%%%%%%%%%%%%%%%%%
\subsection*{Acknowledgements}
This research was supported by Monash University Dean's International Postgraduate Scholarship and Postgraduate Publication Awards to C.O., and Australian Research Council funds to T.C. Valuable feedback on earlier versions of the manuscript were provided by C.~Venables, M.~Patten, H.~Spencer, EDITOR, and XXX reviewers. C.~O. conceived the study, developed the recursions, performed the analyses, and wrote the manuscript. T.C. assisted in developing the recursions and writing the manuscript.


%%%%%%%%%%%%%%%%%%%%%
% Bibliography
%%%%%%%%%%%%%%%%%%%%%
\bibliography{dioecySA-bibliography}

\newpage


%%%%%%%%%%%%%%%%%%%%%%%%%%%%%%%%%%%%%%%%%%%%%%%%%%%%%%%%%%%%%%%%%%
%  Tables 

\begin{table}[htbp]
\centering
\caption{\bf Fitness expressions for diploid adults prior to reproduction for the model of a dominant male-sterility mutation ($w^f_{ij}$ denotes fitness effects through the female sex-function , $w^m_{ij}$ for male sex-function).}
\begin{tabular}{l c c c c} \hline
Haplotype & $ AM_1$ & $ AM_2$ & $ aM_1$ & $ aM_2$ \\
\hline
Female sex-function & & & & \\
$ AM_1$ & $1$ & $(1 + k)$ & $(1 - h_f s_f)$        & $(1 - h_f s_f)(1 + k)$ \\
$ AM_2$ & $-$ & $(1 + k)$ & $(1 - h_f s_f)(1 + k)$ & $(1 - h_f s_f)(1 + k)$ \\
$ aM_1$ & $-$ & $-$       & $(1 - s_f)$            & $(1 - s_f)(1 + k)$ \\
$ aM_2$ & $-$ & $-$       & $-$                    & $(1 - s_f)(1 + k)$ \\
Male sex-function & & & & \\
$ AM_1$ & $(1 - s_m)$ & $0$ & $(1 - h_m s_m)$ & $0$ \\
$ AM_2$ & $-$         & $0$ & $0$             & $0$ \\
$ aM_1$ & $-$         & $-$ & $1$             & $0$ \\
$ aM_2$ & $-$         & $-$ & $-$             & $0$ \\
\hline
\end{tabular}
\label{tab:fitness}\\
{\footnotesize Note: Rows and columns indicate the \textit{i}th and \textit{j}th gametic haplotype respectively. The lower triangle of each matrix is the reflection of the upper triangle, and is omitted for simplicity and consistency with the $i \geq j$ row/column indexing used throughout the article.}
\end{table}
\newpage{}



%%%%%%%%%%%%%%%%%%%%%%%%%%%%%%%%%%%%%%%%%%%%%%%%%%%%%%%%%%%%%%%%%%
%  Figures 

\begin{figure}[htbp]
\centering
\includegraphics[scale=0.57]{./Fig1}
\caption{Invasion of unisexuals into populations with pre-existing SA polymorphism. Plots show the fraction of parameter conditions maintaining single-locus SA polymorphism (within the range $0 < s_f,s_m \leq 0.5$) where a dominant sex-specific sterility allele at $\mathbf{M}$ can invade, assuming additive SA fitness effects at the SA $\mathbf{A}$ locus ($h_f=h_m=1/2$), plotted as a function of the recombination rate $r$. Panels A--C show results from the model of gynodioecy via invasion of a male-sterility allele, while planels D--F show results for the model of androdioecy via invasion of a female-sterility allele. For each panel, results are shown for different values of reproductive compensation, $k$, chosen as a fraction of the single-locus invasion threshold for the unisequal sterility allele ($\hat{k}$, which is equal to the right-hand side of Eq(\ref{eq:1LocGyn}) or Eq(\ref{eq:1LocAndro}) for the models of gynodioecy and androdioecy respectively). Hence, the orange, green, and dark blue lines show scenarios where unisexuals experience a decrease in gamete production relative to hermaphrodites of $1$, $5$, and $10\%$. Note the different scale for the x-axis in panels C and F. Results were obtained by evaluating the three candidate leading eigenvalues ($\lambda_{\mathbf{A}}$,$\lambda_{\mathbf{M}}$,$\lambda_{\mathbf{AM}}$) of the Jacobian matrix of the genotype $\times$ transmission mode recursions for populations at the above initial conditions for $1000$ points uniformly distributed throughout the relevant $s_f \times s_m$ parameter space.}
\label{fig:PrInv}
\end{figure}

\newpage{}

\begin{figure}[htbp]
\centering
\includegraphics[width=\linewidth]{./Fig2Alt}
\caption{Effect of linkage on equilibrium frequencies of unisexual (A) females, and (B) males when reproductive compensation is above the single-locus threshold for invasion of unisexual sterility allele (i.e., $k > \hat{k}$; where $\hat{k}$ is equal to the right-hand side of Eq(\ref{eq:1LocGyn}) or Eq(\ref{eq:1LocAndro}) as appropriate). Results are shown for the models of gyno- and androdioecy via invasion of recessive unisexual sterility alleles, additive fitness effects at $\mathbf{A}$ ($h_f = h_m = 0.5$, using selection coefficients of $s_m = 0.1$ (for the model of gynodioecy) and $s_f = 0.1$ (for the model of androdioecy), and inbreeding depression that follows $\delta = \delta^\ast(1 - C/2)$ (see \hyperref[sec:Models]{Models}). Plots illustrate the increase in equilibrium frequencies of unisexuals predicted by our two-locus models (dashed greyscale lines) relative to the corresponding exact single-locus equilibrium frequencies (solid black line; $\hat{Z}$ predicted by \citealt{Charlesworth1978a}). Results are shown for four different levels of recombination, highlighting that with weaker linkage, the two-locus predictions converge on those of the single-locus model.}
\label{fig:eqFreq2v1Loc}
\end{figure}
\newpage{}

\begin{figure}[htbp]
\centering
\includegraphics[scale=0.75]{./Fig3AltPts}
\caption{Equilibrium frequencies of unisexual (A--B) females and (C--D) males across a gradient of reproductive compensation (with values chosen as a fraction of the single-locus invasion threshold for $M_2$ ($\hat{k}$, which is equal to the right-hand side of Eq(\ref{eq:1LocGyn}) or Eq(\ref{eq:1LocAndro})() for the models of gynodioecy and androdioecy respectively). Results are shown for the models of gyno- and androdioecy via invasion of recessive unisexual sterility alleles, additive fitness effects at $\mathbf{A}$ ($h_f = h_m = 0.5$, using selection coefficients of $s_m = 0.1$ (for the model of gynodioecy) and $s_f = 0.1$ (for the model of androdioecy), and inbreeding depression that decreases linearly with the selfing rate: $\delta = \delta^\ast(1 - C/2)$ (see \hyperref[sec:Models]{Models}). Plots show the equilibrium frequencies of unisexuals predicted by our models for five different levels of reproductive compensation, calculated as a fraction of the single-locus invasion criterion for $M_2$ defined by Eq(\ref{eq:1LocGyn}) and Eq(\ref{eq:1LocAndro}). Note that the single-locus equilibrium frequency of unisexuals always equals $0$ when $k < \hat{k}$. Hence, the lines corresponding to $k < \hat{k}$ illustrate how linkage among SA loci expands the parameter space where unisexual sterility alleles can invade beyond the predictions of the single-locus models.}
\label{fig:eqFreq}
\end{figure}


\end{document}
